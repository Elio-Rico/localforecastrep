\documentclass[letterpaper,12pt]{article}
\usepackage{newtxtext}
\usepackage{newtxmath}
\usepackage{amsmath}
\let\openbox\relax
\usepackage{amsthm}              % See geometry.pdf to learn the layout options. There are lots.
\usepackage{geometry}
\usepackage[figuresleft]{rotating}
\usepackage{pdflscape}
\usepackage{afterpage}
\usepackage{graphicx}
%\usepackage{amssymb}
\usepackage{epstopdf}
\usepackage{multirow}
\usepackage{booktabs}
\usepackage{setspace}
\usepackage{float}
\setlength{\columnseprule}{.4pt}
%\doublespacing
\usepackage{natbib} % if you're using natbib
\setlength{\bibsep}{0pt} % removes the space between items
\usepackage[multiple]{footmisc}
\geometry{left=1in,right=1in,top=1in,bottom=1in}
%\makeatother \addtolength{\textwidth}{0.4in}
%\addtolength{\oddsidemargin}{-0.2in}
%\addtolength{\textheight}{0.8in} \addtolength{\topmargin}{-0.4in}
\newtheorem{prop}{Proposition}
\newtheorem{cor}{Corollary}
\newtheorem{lemma}{Lemma}
\newtheorem{assumption}{Assumption}
\usepackage{xcolor}
\setlength{\tabcolsep}{2pt}
\usepackage{xr-hyper}
\externaldocument{Appendix}

% datatool to  make ces in Latex to external values
\usepackage{datatool}
\DTLsetseparator{|}
\DTLloaddb{valuesData}{Sections/valuesDataUpdated.csv}
%\DTLloaddb{valuesData2}{Sections/valuesData2.csv}
%
% make tables from csv:
\usepackage{csvsimple}

\usepackage{longtable}
\usepackage{tabularx}

% subfigures
\usepackage{caption}
\usepackage{subcaption}

% inline notes or comments
%\usepackage{todonotes} % if want notes

% environment for figure notes:
\newenvironment{fignote}{\footnotesize \begin{singlespace} \noindent}{\end{singlespace} \par }

% environment for table notes:
\newenvironment{tabnote}{\footnotesize \begin{singlespace} \noindent }{\end{singlespace} \par}



% new date format:
\usepackage{datetime}

\newdateformat{monthyeardate}{%
	\monthname[\THEMONTH], \THEYEAR}


\renewcommand\qedsymbol{$\blacksquare$}


\usepackage{hyperref}

\renewcommand{\baselinestretch}{2}


\DeclareGraphicsRule{.tif}{png}{.png}{`convert #1 `dirname #1`/`basename #1 .tif`.png}

\title{Do Local Forecasters Have Better Information?\thanks{We are thankful to the editor, Olivier Coibion, three anonymous referees, Daron Acemoglu, Adrian Bruhin, Harald Hau, Beata Javorcik, Lorenz Kueng, Evgenia Passari, Jean-Paul Renne, David Thesmar, Qingyuan Yang, three anonymous referees and seminar and conference participants at the University of Lausanne's Macro Workshop, the University of Lausanne's Department of Economics Research Days, the EEA annual meeting 2022, the RCEA 2022, the Webinar Series for Graduate Economic Students, T2M 2022, the University of Lyon 2, the KOF Institute, the 2022 Annual meeting of the CEPR International Macro and Finance group, the SFI annual meeting 2022, the Annual Congress of the Swiss Society of Economics and Statistics 2022, the Durham University Business School CEMAP Workshop 2022, the University of Heidelberg, the Halle Institute for Economic Research, the 14th ifo Conference on Macroeconomics and Survey Data and the ECB for useful discussions. Navid Samadi provided outstanding research assistance. We acknowledge financial support by the SNSF grant number 100018\_182195.}}
\vspace{-1cm}
\author{Kenza Benhima\footnote{HEC-Lausanne (University of Lausanne) and CEPR, email: kenza.benhima@unil.ch.} and Elio Bolliger\footnote{Swiss Federal Department of Finance.} }
\date{}
\vspace{-1cm}
\date{May 2025}%\today
%\\
%\vspace{0.5cm}
%\href{https://www.dropbox.com/s/74qfltmzqvt3yei/Local_vs_foreign_forecasters_most_recent.pdf?dl=0}{\emph{Click here for the most recent version} }
% }

\begin{document}

\vspace{-1cm}
\maketitle


\vspace{-1cm}
\begin{abstract}
\begin{singlespace}
%Do local forecasters outperform foreign ones when forecasting macroeconomic fundamentals? If so, is this local advantage due to behavioral biases or information asymmetries?
Using inflation and growth forecasts for a panel of emerging and advanced economies, we provide evidence that foreign forecasters publish and update their forecasts less frequently than local forecasters and make larger errors. We provide tests that identify differences in information frictions across groups and show that the lower accuracy of foreigners is not due to more irrational expectations, but to less precise information. This informational advantage is linked both to barriers to information and to incentives and is typically stronger when uncertainty is lower. These results provide a basis for disciplining international finance and trade models with heterogeneous information.\\
\textbf{Keywords:} Information asymmetries, Expectation formation. \\
\textbf{JEL codes:} E3, E7, D82.
\end{singlespace}

\end{abstract}

%\newpage
%\tableofcontents

\newpage

\setcounter{page}{1}

\section{Introduction}

\label{sec:introduction}
The informational advantage of local forecasters over foreign forecasters regarding macroeconomic fundamentals has far-reaching consequences. Information asymmetries are a primary explanation for the tendency of investors to prefer domestic assets in their investment portfolios, known as the home bias in asset holdings, originally documented by  \citet{FrenchPoterba1991}.\footnote{On asymmetric information and the home bias, see for instance \citet{Admati1985}, \citet{Portesetal2001}, \citet{DeMarcoetal2021}.} Information asymmetries are also a potential source of capital flow volatility, since disagreement between foreign and domestic investors generates cross-border asset trade.\footnote{See \citet{BrennanCao1997},  \citet{TillevanWincoop2014}, \citet{BenhimaCordonier2022}.} Beyond their impact on international asset markets, they also constitute a barrier to the international trade in goods (potentially accounting for the ``missing trade''), as highlighted by \citet{AndersonvanWincoop2004}. Finally, recent papers underscore their role in international business cycle comovement.\footnote{See \citet{Buietal2021}.} However, there remains a lack of direct evidence regarding the existence of information asymmetry on macroeconomic fundamentals and quantitative estimates of the extent of this asymmetry.

We fill this gap by exploiting a unique dataset of inflation and GDP growth forecasts for the current and the next year provided by local and foreign forecasters. Unlike previous studies, the forecaster and country dimensions of the panel allows us to control for a rich set of fixed effects. We first show that foreign forecasters publish and update their forecasts about 10\% less frequently than local forecasters. Since forecasters in our dataset update their forecasts on average 6 times a year, this corresponds to about 0.6 fewer updates annually. They also make more mistakes than local forecasters, and foreign forecasters' absolute error is on average 6-9\% higher than local forecasters, which corresponds to 0.035-0.04 percentage points on average. We argue that these estimates are economically relevant.

We then investigate the role of information frictions and behavioral biases in explaining our results about forecast errors. We do this in two steps. First, we rule out behavioral biases such as over-reaction to new information as a key explanation of the foreigners' excess mistakes, by showing that the local and foreign biases do not differ systematically. Second, we show that local forecasters have more precise private information. To do so, we build on and extend the fast-growing literature that uses model-based tests to identify frictions in expectation formation. In particular, we provide tests of asymmetric information that are robust to the presence of public signals.

We then explore the determinants of the information asymmetry between local and foreign forecasters. First, we show that the location of subsidiaries plays an important role for the information produced by multinationals. Having a subsidiary located in a country is associated with the same informational advantage as having headquarters based there. Second, geography matters. Linguistic distance in particular appears to be a major barrier to information. Foreign forecasters also forecast better when the economic ties (trade and financial) between the country where their headquarter is located and the country they are forecasting are stronger. This evidence is consistent both with the existence of exogenous barriers to information and with smaller incentives to acquire information. While the limitations of our data prevents us from fully disentangling these two channels, we provide arguments that they are both at play.

Interestingly, we show that information asymmetries are inversely related to forecasting uncertainty. Indeed, the local advantage is higher for short horizons, for inflation (as opposed to GDP growth), and for large countries. In all these situations, the forecasting uncertainty (measured by the average forecast error) happens to be \emph{smaller}. This evidence suggests that when information is available, local forecasters are better at finding it. Interestingly, the information asymmetry \emph{increases} in the course of a year (the asymmetry is higher in December than in January). This is consistent with the idea that local forecasters are more aware of the regular releases of partial GDP growth and inflation figures and integrate this information faster. Consistently, inflation figures are typically available at a higher frequency and with a shorter lag than GDP, making the access to that information an even greater advantage. However, we find no evidence that the difference in forecast errors between local and foreign forecasters is higher for developing countries, when institutional quality is poor, or when macroeconomic volatility is high.

%Information advantages have been used to explain exchange rate variations (Evans and Lyons (2005), Bacchetta and van Wincoop (2006), international consumption correlation puzzle (Coval (2003)), international capital flows (Brennan and Cao (1997)), a bias towards investing in local firms (Coval and Moskowitz (2001)), and the own-company stock puzzle (B and Wang (2003)). Information asymmetry is also the basis for other home bias explanations, such as ambiguity aversion (Uppal and Wang). All these explanations are bolstered by our finding that information are not only sustainable when information is mobile, but by the fact that asymmetric information can be amplified when investors can choose what to learn.

%The canonical reference on asymmetric information with multiple assets is Admati (1985). Work on asymmetric information and the home bias, in particular, includes Pastor (2000), Brennan and Cao (1997), and Portes, Rey, and Oh (2001).

%Ahearne, Griever, and Warnock (2004)


%ChatGPT thrid version: Are our estimates economically significant? Consider the impact of information asymmetries on international trade. While there are no precise quantitative estimates provided by Rauch (2001) on the percentage difference in information precision required to fully explain trade barriers, his qualitative analysis emphasizes the role of networks in reducing information frictions and facilitating trade, particularly for differentiated products. This suggests that our estimates can be informative in contexts where detailed empirical estimates are lacking. \footnote{See \citet{Rauch2001} for an analysis of the role of networks and information asymmetries in trade. Although Rauch does not provide specific quantitative estimates, his work highlights the significant impact of these frictions on trade. See also \citet{AndersonvanWincoop2004}, \citet{NunnTrefler2013}, and \citet{HeadMayer2013} for discussions on the broader implications of information asymmetries in trade.} Our estimates are more in line with models that rely on amplification effects. For instance, \citet{Allen2014} shows that small differences in information precision can lead to significant trade imbalances when firms decide on market entry and investment based on their information sets. \footnote{See also \citet{Chaney2008}, who discusses how minor information differences can influence trade patterns and volumes through amplification effects.} Specifically, \citet{Chaney2008} finds that information frictions can significantly affect trade flows by influencing firms' decisions to enter foreign markets. Similarly, \citet{NunnTrefler2013} discuss how relational contracts and trust, which are closely tied to information asymmetries, impact international trade. Therefore, while our estimates of information asymmetries alone may not fully explain the observed trade barriers, they are consistent with theoretical models where small initial differences in information can be amplified through market dynamics and firm behaviors.

This paper contributes to the recent literature that uses professional forecasters' expectations to identify information frictions and behavioral biases. This literature has used reduced-form estimations as indicators of deviations from Full-Information Rational Expectations (FIRE). \citet{CoibionGorodnichenko2015} (CG henceforth) use the estimated coefficient in the regression of the consensus error on the consensus revision as an indicator of deviations from Full Information (FI). \citet{Bordaloetal2020} use the estimated coefficient in the individual pooled regression as an indicator of deviations from Rational Expectations (RE).\footnote{An earlier literature has previously identified deviations from rationality by studying the joint behavior of actual on predicted values, the auto-correlation of forecasts revisions and the predictability of errors. See, for example, \citet{MincerZarnowitz1969} and \citet{Nordhaus1987}.} We borrow this test to assess whether domestic and foreign deviations from RE differ.

However, CG's FI test, which has been commonly used in the literature, is not adapted to our purpose. In the presence of public information, the CG coefficient, which is a common measure of information frictions, is biased. Importantly, the bias is not a monotonic function of the precision of private signals. Comparing the CG coefficient across local and foreign forecasters cannot indicate which group faces more frictions. We thus provide two tests that are robust to the presence of public information. The first relies on individual regressions with country-time fixed effects to capture aggregate shocks and the public signals. This test is similar in spirit to \citet{Goldstein2021}, who proposes to use forecasters' deviations from the mean to measure information frictions robustly. The second test infers the relative precision of private information from the relative reaction of expectations to public signals.

This paper also belongs to the empirical literature documenting the informational advantage of locals. Many studies provide indirect evidence of asymmetric information between domestic and foreign investors by showing that location matters for portfolio composition and for portfolio returns.\footnote{See for instance \citet{KangStulz1997}, \citet{GrinblattKeloharju2001}, \citet{PortesRey2005}, \citet{Hau2001}, and \citet{Clemensetal2020}. Based on investor choices and returns, some papers find that foreign investors perform better than local investors (e.g. \citet{GrinblattKeloharju2000}). This could be explained by the specialization of some investors in some specific markets where they have an initial informational advantage. Location can be a source of this informational advantage, but it is not the only one. Therefore, information heterogeneity can also lead to specialization in either domestic or non-domestic assets (see \citet{VanNieuwerburghVeldkamp2010} and \citet{DeMarcoetal2021}).} Other studies document foreigners' lack of attention to domestic information.\footnote{See for instance \citet{Mondriaetal2010} and \citet{Czirakietal2021}.} In contrast to these studies, we investigate whether location affects the quality of forecasters' information, thus providing direct evidence of information asymmetries. Closest to our study is the paper by \citet{Baeetal2008}, which studies the performance of local and foreign analysts in forecasting earnings for firms. Our focus is different since we examine whether locals outperform foreigners in forecasting aggregate variables. Moreover, we investigate whether the foreigners' excess mistakes come from information frictions or behavioral biases.

The paper is structured as follows. Section \ref{sec:data} describes our dataset. Section \ref{sec:mistakes} documents the updating frequency of forecasts and foreign forecasters' excess mistakes. Section \ref{sec:model} lays down a model of expectation formation and tests for the sources of the foreigners' excess mistakes. Sections \ref{sec:geography} and \ref{sec:heterogeneity} investigate respectively the drivers of the foreign penalty and of its heterogeneity. Finally, Section  \ref{sec:conclusion} concludes.



\section{The Data}\label{sec:data}

\paragraph{Forecasts.} We use data from Consensus Economics, a survey firm polling individual economic forecasters on a monthly frequency. The survey covers 51 advanced and emerging countries and we focus on observations between 1998 and 2021.\footnote{For an overview of the data coverage of all advanced and emerging economies in our sample see Table \ref{tab:app_data_countryrange} in the online Appendix. Note that the survey provides forecasts as of 1989 for some countries. However, our sample period is limited by the GDP and inflation vintage series of realized outcomes provided by the IMF. The development status retained for the countries of our sample is available in Table \ref{tab:app_emerging_advanced} in the online Appendix.} Each month, forecasters provide estimates of several macroeconomic indicators for the current and the following year. %An advantage of this dataset is that it allows for  meaningful comparisons across both countries and forecasters.\footnote{Consensus Economics clearly defines each macroeconomic indicator surveyed. }
 In this paper, we focus on two indicators, namely CPI inflation and GDP growth. The dataset discloses the name of the individual forecasters. There are \DTLfetch{valuesData}{indicator}{N}{value} unique forecasters from which \DTLfetch{valuesData}{indicator}{Nctry}{value} conduct forecasts for at least 2 distinct countries.  For each forecaster-country pair, the average (median) number of observations is \DTLfetch{valuesData}{indicator}{meanIdct}{value} (\DTLfetch{valuesData}{indicator}{medianIdct}{value}), which corresponds to approximately \DTLfetch{valuesData}{indicator}{meanIdctYr}{value} (\DTLfetch{valuesData}{indicator}{medianIdctYr}{value}) years. This leads to an unbalanced panel dataset.


\paragraph{Realized Outcomes.} Following the literature, we use first-release data to compare forecast precision across forecasters. For each survey year, we use the realized outcome for yearly inflation and real GDP growth from the International Monetary Fund World Economic Outlook (IMF WEO) published in April of the subsequent year. This allows us to avoid forecast errors that are due to data revisions. For example, to assess the accuracy of the 2013 real GDP growth forecast for Brazil from the January 2013 survey, we use the yearly GDP growth reported in the April 2014 IMF WEO as realized outcome. To assess the accuracy of the 2014 real GDP growth forecast for Brazil from the same January 2013 survey, we use the yearly GDP growth reported in April 2015. %We conduct robustness checks with alternative vintages using IMF WEO published in September or in subsequent years.\footnote{Using alternative vintage series ensures that differences in forecasting precisions are not solely due to individual forecasters that anticipate revisions in actual GDP or inflation and therefore have a different forecasting target.}
 Archived IMF WEO vintage data are available from 1998 onwards.

As is common in the literature, we trim observations, removing forecasts that are more than 5 interquartile ranges away from the median. The quantiles are calculated first on the whole sample, but separately for emerging and advanced countries, and second, for each country and date. This trimming ensures that our results are not driven by extreme outcomes, such as periods of hyperinflation, or by typos. It reduces the number of forecasts for current inflation and GDP by \DTLfetch{valuesData}{indicator}{fcastTrimLosscpi}{value} and \DTLfetch{valuesData}{indicator}{fcastTrimLossgdp}{value} \% respectively.%\DTLfetch{valuesData2}{indicator}{fcastTrimLossgdp}{value} and \DTLfetch{valuesData2}{indicator}{fcastTrimLosscpi}{value}, respectively.

\paragraph{Forecast errors.} We use this information to construct forecast errors. The forecast errors with respect to the current year and the future year are defined as
$Error_{ijt,t}^m=x_{jt}-E_{ijt}^m(x_{jt})$ and $Error_{ijt,t-1}^m=x_{jt}-E_{ijt-1}^m(x_{jt})$, 
where $t$ refers to the year, $i$ is the forecaster, $j$ is the country, $m=1,..,12$ is the month of the year when the forecast is produced, and $x$ is either inflation or GDP growth.

\paragraph{Location of Forecasters.} Consensus Economics discloses the name of the forecasting institution. Eikon (Refinitiv) provides the company tree structure of most forecasters in our dataset. The tree structure includes information about the countries in which the headquarters, the subsidiaries and the affiliates are located. If the forecaster was not listed in the Eikon database, we manually searched for this information on the Internet. In the main analysis, we consider a forecaster to be foreign if neither its headquarter nor any of its subsidiaries are located in the country of the forecast.\footnote{Note that the location information is not time-varying and corresponds to the information accessed in 2021. This amounts to a measurement error that could bias the magnitude of the location effect downward.} %Later, we will examine whether the forecasts produced by a forecaster headquartered in the country are different from the forecasts produced by a forecaster with only a subsidiary located in the country.


\paragraph{Forecasters' Scope.} Furthermore, we identify the scope of the forecasters. We categorize forecasters with subsidiaries and headquarters all located in the same country as non-multinational forecasters. In contrast, we categorize forecasters with at least one subsidiary located in a country outside that of their headquarters as multinationals.\footnote{The scope variable is also based on information from 2021 and is therefore not time-varying.}

\paragraph{Descriptive Statistics} Some descriptive statistics, shown in the online Appendix, are particularly relevant for our analysis. According to Panel a) of Figure \ref{fig:hq_sub_obs}, almost two thirds of the forecasts come from multinational forecasters, and almost three quarters are made by ``local'' forecasters, that is, forecasters who have either their headquarters or a subsidiary located in the country. As compared to non-multinational forecasters, a higher proportion of forecasts by multinational forecasters is local, thanks to their subsidiaries. Since multinationals are also more likely to have well-endowed forecasting departments and to make smaller forecasting errors, it will be important to control for forecaster-level characteristics.

According to Figure \ref{fig:hq_sub_obs_by_cty}, the share of forecasts provided by foreign forecasters (that have neither their headquarters, nor a subsidiary located in the country) varies a lot across countries. Countries that have a lower share of foreign forecasts are large advanced economies where many national and multinational forecasters have their headquarters (United States, Japan, Germany...), or large emerging economies where many multinational forecasters have subsidiaries (China, South Korea, Brazil, Chile...). Countries with a high share of foreign forecasts tend to be small advanced or emerging economies, as they are less likely to host the headquarters or a subsidiary of a forecaster (e.g., Bulgaria, Latvia, Greece, Nigeria). Since smaller countries have more volatile business cycles that are more difficult to forecast, it will be important to control for country-level characteristics in our analysis.

Because we will be controlling for forecaster and country characteristics with fixed effects, the identification of the role of the foreign nature of forecasts will come from the forecasters that provide both local and foreign forecasts. Panel a) of Figure \ref{fig:loc_for} shows that these forecasters are a small minority of forecasters (11\%). Indeed, only 86 forecasters among 749 provide forecasts for both local and foreign countries. However, this minority of forecasters accounts for 60\% of observations in our sample, as Panel b) shows.\footnote{This is consistent with Figure \ref{fig:hist}, which shows that most forecasters provide forecasts for only one country, and only a small proportion of forecasters provide forecasts for 5 countries or more. Even though the remaining 40\% of observations will not contribute to the identification of the role of the forecaster's location, they will contribute to the identification of the role of the country and forecaster fixed effects and improve the precision and power of our estimates.} %Besides, most of these observations come from multinational firms.
%especially non-multinational firms,
%(22\% of multinational forecasters, 5\% of non-multinational forecasters). The difference in forecasting behavior or performance that we will identify will thus come from the most ``expert'' among forecasters: the multinational forecasters and those that cover many countries. The role of these characteristics will be explored in more details later.


\section{Forecast Errors and Updating}\label{sec:mistakes}

In this section, we analyze the forecasters' errors and forecast updating. We find that foreign forecasters make larger errors than local ones, and that they update their forecasts less often.

\subsection{Foreign Forecasters Make Larger Errors}

As preliminary descriptive evidence on forecast errors, Panels (a) and (b) of Figure \ref{fig:errors} in the Appendix show the density of forecast errors regarding the current year for each group of forecasters. The forecast errors are distributed around 0 for both local and foreign forecasters. However, the distribution of forecast errors for foreign forecasters is wider than for local forecasters, which indicates that foreign forecasts are less precise.\footnote{Panels (c) and (d) of Figure \ref{fig:errors} in the Appendix show similar distributions regarding forecasts about the future year.} Formal tests of variance equality are performed in Appendix \ref{app:sec:variance} and show that the variance of foreign forecasters' errors is indeed significantly larger.

Note, however, that this preliminary evidence does not control for country- and forecaster-specific characteristics. For instance, a higher proportion of forecasts by multinational forecasters is local. Given that multinationals are also more likely to have well-endowed forecasting departments, local forecasts could artificially appear more accurate if we do not control for forecasters' characteristics. Besides, small countries' forecasts are more likely to be produced by a foreign forecaster. Given that small countries typically have more volatile business cycles, foreign forecasts could misleadingly appear less accurate if we do not control for country characteristics. For this reason, we control for forecaster- and country-specific characteristics by exploiting the panel structure of our data.

As a first measure of the forecast error distribution, we estimate the standard deviation $\sigma^m_{\text{FE},i,j}$ of the forecast error for every forecaster-country-month triplet $(m,i,j)$ for current forecasts. We discard forecaster-country-month triplets with less than 10 observations. We take the log of $\sigma^m_{\text{FE},i,j}$ and estimate
\begin{align}
	\ln(\sigma^m_{\text{FE},i,j}) =  \delta^m +\tilde\delta_{i} + \bar{\delta}_{j} +
\beta \text{Foreign}_{ij} + \varepsilon_{ij}^m  \,, \label{eq:regModelSE_FE}
\end{align}
where  $ \delta^m $, $\tilde\delta_{i} $ and $\bar{\delta}_{j} $ are respectively month-of-year, forecaster and country fixed effects. $\text{Foreign}_{ij} $ is a dummy that takes the value of 1 if forecaster $i$ is foreign to country $j$, and 0 otherwise.

Column (1) of Table \ref{tab:updating_errors_main_small} reports the coefficient $\beta$. The standard deviation of forecast errors is higher when a forecaster produces a foreign forecast than when it produces a local one (14\% higher for inflation, 11\% higher for GDP growth). Since the average standard deviation for local forecasters is 0.63pp for CPI inflation and 0.93 for GDP growth, this implies that the foreign forecasts' extra standard deviation is approximately 0.09-0.1pp for both variables.\footnote{Table \ref{tab:tab_stderr_rob} in the Appendix shows the results for alternative, less rich fixed-effect specifications. The results show that fixed effects are important to reduce endogeneity. For instance, the coefficient of Foreign drops when country fixed effects are added (Column (2)). It is possible, as discussed above, that institutions forecasting small countries, which are also more volatile, are more likely to be foreign.}

{\setstretch{1}
	\begin{table}[H] \centering
\newcolumntype{C}{>{\centering\arraybackslash}X}

\caption{Forecast Errors, Updating, and the Location of the Forecaster - Forecasts on the Current Year}
\label{tab:updating_errors_main_small}
{\footnotesize
\begin{tabularx}{\linewidth}{l l C m{0.01\textwidth} C C m{0.01\textwidth} C C}

\toprule
{}&{}&{$\ln(\sigma^m_{\text{FE},i,j})$}&&\multicolumn{2}{c}{$\ln(|Error_{ijt,t}^m|)$}&&\multicolumn{2}{c}{$\ln(N_{ijt})$} \tabularnewline \cline{3-3} \cline{5-6} \cline{8-9} \tabularnewline &&{(1)}&&{(2)}&{(3)}&&{(4)}&{(5)} \tabularnewline
{Variable}&{Coefficient}&{}&{}&{}&{Distinct updates}&{}&{}&{Distinct updates} \tabularnewline
\midrule \addlinespace[0pt]
\midrule $\text{CPI}_t$&Foreign&0.12**&&0.09***&0.08***&&--0.12***&--0.12*** \tabularnewline
&&(0.05)&&(0.02)&(0.03)&&(0.04)&(0.04) \tabularnewline
&N&6,662&&99,228&54,654&&10,857&10,822 \tabularnewline
&$ R^2 $&0.80&&0.62&0.68&&0.53&0.50 \tabularnewline
$\text{GDP}_t$&Foreign&0.09**&&0.06**&0.06**&&--0.10***&--0.10*** \tabularnewline
&&(0.04)&&(0.02)&(0.02)&&(0.03)&(0.03) \tabularnewline
&N&7,131&&103,866&58,157&&11,240&11,238 \tabularnewline
&$ R^2 $&0.88&&0.66&0.72&&0.54&0.52 \tabularnewline
&Country, For., Month FE&\checkmark&&&&&& \tabularnewline
&Country $ \times $ Year FE&&&&&&\checkmark&\checkmark \tabularnewline
&Forecaster $ \times $ Year FE &&&&&&\checkmark&\checkmark \tabularnewline
&Country $ \times $ Date FE&&&\checkmark&\checkmark&&& \tabularnewline
&Forecaster $ \times $ Date FE &&&\checkmark&\checkmark&&& \tabularnewline
\bottomrule \addlinespace[\belowrulesep]

\end{tabularx}
\begin{flushleft}
\footnotesize \begin{minipage}{1\linewidth} \vspace{-10pt} \begin{tabnote} \textit{Notes:} Column (1) shows the regression of the log standard deviation of the errors on the location of the forecaster. Columns (2) and (3) show the regression of the log absolute forecast error on the location of the forecaster. Columns (4) and (5)) show the results of regression of the number of forecast updates within a year on the location of the forecaster. Standard errors are clustered at the country and forecaster level in columns (1), (4) and (5), and at the country, forecaster and date level in Columns (2) and (3). In Columns (3) and (5), the sample is restricted to the published forecasts that are distinct from the last published one. \end{tabnote} \end{minipage}  
\end{flushleft}
}
\end{table}

}


In this specification, we control for country, forecaster and month-of-year characteristics, but not for the time period. Ignoring time-specific characteristics could bias our results if, for instance, more foreign forecasts are produced in times of turmoil and uncertainty, where all forecasters will make more mistakes. Therefore, as a second measure of the forecast error distribution, we calculate the log absolute value of the forecast error, which is time-varying.\footnote{For absolute forecast errors smaller than 0.001 percentage point, we assign the value of $\ln(0.001)$ to keep all observations in the sample.}  The model we estimate is as follows.
\begin{align}
	\ln(|Error_{ijt,t}^m|)= \delta_{it}^m +\tilde\delta_{jt}^m +  \beta \text{Foreign}_{ij} + \varepsilon_{ij,t}^m  \,, \label{eq:regModelFE}
\end{align}
$ \delta_{it}^m$ are forecaster-date fixed effects and $\tilde\delta_{jt}^m$ are country-date fixed effects. These fixed effects enable us to control for country-specific trends in volatility and forecaster-specific trends in forecasting performance.


Column (2) of Table \ref{tab:updating_errors_main_small} displays the results for CPI and GDP. Foreign forecast errors are significantly larger in absolute value than local forecasts. More precisely, the absolute value of foreign forecast errors is 9\% larger for current inflation and 6\% for current GDP growth. Since the average error of local forecasters regarding CPI inflation (GDP growth) is 0.45pp (0.60pp), this means that the typical extra error on foreign forecasters is on average 0.04pp (0.036pp).\footnote{Table \ref{tab:tab_errors_rob} in the Appendix shows the results for alternative, less rich fixed-effect specifications. The coefficient of Foreign drops when country fixed effects are added (Column (2)). Interestingly, the Foreign coefficient for GDP growth becomes insignificant when introducing country fixed effects. However, this is arguably due to noise, as it becomes significant again when introducing forecaster fixed effects (Column (3)). These results show that fixed effects are not only important to reduce endogeneity, but also to improve the estimates' precision.} %Presumably, as uncertainty is higher when forecasting at a longer time horizon, the informational advantage is lower for local forecasters.


Table \ref{tab:updating_errors_app_small} in the Appendix shows the results for forecasts about future inflation and GDP growth. The gap in forecast precision between foreign and local forecasters appears smaller than for forecasts about current GDP growth and inflation. For inflation, it drops from 14\% to 6\% for the standard deviation of errors (Column (1)) and from 9\% to 6\% for the log of the absolute value of errors (Column (2)). For GDP growth, it even becomes insignificant in Column (2). The extra standard deviation (or extra absolute error) of foreign forecasters still remain sizable at 0.08-10.12pp (0.01-0.05pp), because local forecasters make larger errors when forecasting the future than the current year.


\subsection{Foreign Forecasters Update Their Forecasts Less Often}

One potential source of difference between local and foreign forecasters is sticky information à la \citet{MankiwReis2002}: foreign forecasters may not update their information as frequently. To examine this hypothesis, we compute the number of published forecasts for each year-forecaster-country unit, which we denote $N_{ijt}$. The distribution of these publication frequencies is provided in Panels (a) and (b) of Figure \ref{fig:updates} in the Appendix.
We test formally whether foreign forecasters publish forecasts less often by taking the log of $N_{ijt}$ and estimating
\begin{align}
	\ln(N_{ijt}) = \tilde\delta_{it} + \bar{\delta}_{jt} +
\beta \text{Foreign}_{ij} + \varepsilon_{ijt}  \,, \label{eq:regN}
\end{align}
where  $\tilde\delta_{it} $ and $\bar{\delta}_{jt} $ are respectively forecaster-year and country-year fixed effects.% $\text{Foreign}_{ij} $ is a dummy that takes the value of 1 if forecaster $i$ is foreign to country $j$, and 0 otherwise.

The results are reported in Column (4) of Table \ref{tab:updating_errors_main_small}: foreign forecasters publish their forecasts 10\% to 12\% less often than local forecasters. Since the average publication frequency is 9 publications per year, this means that foreign forecasts would have about one less publication per year. The difference in publication frequency between local and foreign forecasters is smaller when considering GDP growth (as opposed to inflation).\footnote{Table \ref{tab:tab_updating_rob} in the online Appendix shows the results for alternative, less rich fixed-effect specifications. The coefficient of Foreign remains insignificant until we control for forecaster fixed effects (Column (3)). Controlling for country-year fixed effects and forecaster-year fixed effects matters since the coefficients decline slightly when we introduce these fixed effects (Columns (4) and (5)).}

Note that forecasters may publish a forecast without necessarily updating it, so the publication frequency is an imperfect measure of the updating frequency. We thus follow \citet{Andrade2013} and compute the number of yearly forecasts when considering only ``distinct'' forecasts, that is, forecasts that differ from the previous release. Measured in this way, the average updating frequency drops to about 6 times a year.\footnote{An identical forecast does not necessarily reflect the absence of new information, because of rounding, so this measure may understate the updating frequency. However, \citet{Andrade2013} find that rounding actually does not significantly drive infrequent updating.} Panels (c) and (d) in Figure \ref{fig:updates} in the online Appendix provides the distribution of the number of distinct forecasts in a given year. We use this measure to estimate Equation \eqref{eq:regN} and report the results in Column (5) of Table \ref{tab:updating_errors_main_small}. The results, in fact, barely change (10-12\%). Since the average updating frequency is 6 updates per year, this means that foreign forecasts would have about 0.6 less update per year.

The results of Column (2) must be reevaluated under the hypothesis of sticky information. Indeed, one potential source of the foreign excess error could be that the published forecasts are not updated as frequently. We thus report in Column (3) the results of regression \eqref{eq:regModelFE} when restricting the sample to distinct forecasts. The results do not change. This means that the excess error of foreign forecasters is not due to the lower updating frequency: conditional on updating, foreigner still make larger mistakes.\footnote{These results are consistent with \citet{Andrade2013}, who find that disagreement among forecasters does not appear to be entirely driven by infrequent information updating.}

Table \ref{tab:updating_errors_app_small} in the online Appendix shows the results for updates about future inflation and GDP growth. The difference in the frequency of forecast updates is mostly the same for current and future variables (Columns (4) and (5)). This is not surprising since forecasters often produce their forecasts regarding the next year at the same time as they produce forecasts regarding the current year. 
\subsection{Economic Significance of the Foreign Penalty}

Our estimates of a 6-9\% higher standard error can be interpreted as a 12-19\% difference in the conditional variances. Assuming that they are driven by information asymmetries (which we will discuss in the next section), are they economically significant? 

These estimates are not large enough to provide an explanation of the home bias per se. For instance, \citet{Jeske2001} finds that the foreign penalty necessary to explain the home bias varies between 25\% for the US to 80\% for Italy. However, the literature has shown that small information costs can be significantly amplified by investor behavior and market mechanisms. \citet{VanNieuwerburghVeldkamp2009} show that a difference in the variance of priors as small as 10\% can generate empirically plausible levels of home bias when investors can choose what information to learn before they invest. Our estimates are therefore more in line with models that feature amplification effects. \citet{Hatchondo2008} show that, when investors face short-selling constraints, a small information asymmetry can generate a sizable home bias. When returns are correlated, small diversification costs can be enough to generate a home bias. In particular, according to \citet{Wallmeier2022}, a 5\% home ``variance advantage'' can alone explain half of the observed home bias when the return correlation is 0.9, which is the value they document for 9 major economies.

Moreover, other phenomena driven by disagreement between local and foreign agents can be explained by information asymmetries of this magnitude. For instance, \citet{TillevanWincoop2014} show that the degree of information asymmetry that generates a plausible level of gross capital flow volatility implies a very small difference in the conditional variances.\footnote{Despite a 50\% difference in the volatility of individual noise, the difference in the volatility of forecast errors is smaller than 1\%, because the variance of the fundamental is one order of magnitude smaller.} In the international trade literature, \citet{Allen2014} shows that, in a model where firms decide on market entry and investment based on their information sets, small information costs are consistent with the empirical extensive and intensive patterns of trade, but ignoring these costs significantly deteriorates the fit of the model.

In general, more quantitative work is needed to evaluate the quantitative relevance of macroeconomic information frictions in international finance and trade. Our estimates provide a useful conservative benchmark to do so. Indeed, our estimates can be interpreted as a lower bound on the level of information asymmetries that are relevant for decision-making. The Consensus Economics Survey that we use is based on a panel of professional forecasters that are selected because forecasting is part of their business. They are by construction better forecasters than other firms. %Besides, the variables that are the object of the forecast are macroeconomic variables that are easier to forecast because they are less volatile and plenty of public information is available. Economic decisions often needs information on individual firms or markets, for which public information is not as plentiful. We can thus expect asymmetric information on ``microeconomic'' data to be at least as high as our estimates.



\section{Behavioral Biases or Information Asymmetry?}\label{sec:model}

To account for foreigner excess error, we lay down a simple noisy information model. We explore two potential sources of heterogeneity between local and foreign forecasters: behavioral biases and information asymmetry. We rule out differences in behavioral biases using rational expectation tests that are now common in the literature. We then establish the presence of asymmetric information by using two tests that are robust to common behavioral biases and to public signals.

\subsection{A Simple Noisy Information Model}


We consider a set of $N$ professional forecasters indexed by $i=1,..,N$ who form expectations on $J$ countries indexed by $j=1,..,J$. We denote by $x_{jt}$ the variable that is forecasted. Denote by $\textit{S}(j)$ the set of forecasters who form expectations on country $j$. Forecaster $i\in\textit{S}(j)$ can belong either to the group of local forecasters $\textit{S}^l(j)$ or to the group of foreign forecasters $\textit{S}^f(j)$. We denote by $N(j)$, $N^l(j)$ and $N^f(j)$ the number of elements in $\textit{S}(j)$, $\textit{S}^l(j)$ and $\textit{S}^f(j)$ respectively. We assume that $x_{jt}$, the yearly realization of $x_j$, follows an AR(1):
$x_{jt}=\rho_jx_{jt-1}+\epsilon_{jt}\label{eq:ar1}$, 
with $\epsilon_{jt}\sim\textsl{N}(0,\gamma^{-1/2})$.

\subsubsection{Information structure and behavioral biases}

We consider an information structure and behavioral assumptions that are similar to \citet{Angeletosetal2020}, except that we include public signals.

\paragraph{Information structure.}

We assume that the information structure is country, month, and forecaster-specific. Between month $m$ of year $t-1$ and month $m$ of year $t$, forecasters receive two types of signals: a public signal
$\phi_{jt}^m=x_{jt}+(\kappa_{j}^m)^{-1/2}u_{jt}^m$
observed by all forecasters, where $u_{jt}^m\sim\textsl{N}(0,1)$ is an i.i.d. aggregate noise shock and $\kappa_{j}^m>0$ is the precision of the public signal, which is specific to country $j$ and to month $m$, and a private signal
$\varphi_{ijt}^m=x_{jt}+(\tau_{ij}^m)^{-1/2}e_{ijt}^m$
that is observed only by forecaster $i$, where $e_{ijt}^m\sim\textsl{N}(0,1)$ is an i.i.d. idiosyncratic noise shock, $\tau_{ij}^m>0$ is the precision of the private signal, which is specific to country $j$, to month $m$, but also to forecaster $i$. Through the law of large numbers we have $\frac{1}{N^l(j)}\sum_{i\in\textit{S}^l(j)}e_{ijt}^m=0$ and $\frac{1}{N^f(j)}\sum_{i\in\textit{S}^f(j)}e_{ijt}^m=0$.

We assume that, for a given month $m$, $e_{ijt}^m$ and $u_{jt}^m$ are mutually and serially independent. This means, for instance, that the noise shocks in the signals of month $m$ from year $t$ are not correlated with the noise shocks in the signals of month $m$ from year $t-1$. But we do not impose that the noise shocks are serially uncorrelated within a given year.\footnote{This type of information structure would arise if forecasters were receiving independent signals every month. In that case, the information received between month $m$ of year $t-1$ and month $m$ of year $t$ would be represented by a 12-month moving average of the monthly signals, which is serially correlated on a month-on-month basis, but not on a year-on-year basis.}

\paragraph{Behavioral biases.}

Following \citet{Angeletosetal2020}, we consider two behavioral biases that go a long way in explaining survey forecasts: over-extrapolation and over-confidence. We denote forecaster $i$'s belief about the persistence of $x_{jt}$ by $\hat\rho_{ij}$, and her belief about the precision of her private signal by $\hat\tau_{ij}^m$. Over-(under-)extrapolation consists in distorted beliefs about the persistence of shocks $\rho_j$: $\hat\rho_{ij}\neq\rho_j$. Over-(under-)confidence consists in distorted beliefs about the precision of private signals $\hat\tau_{ij}^m\neq\tau_{jk}^m$.

\paragraph{Expectations.}

Between month $m$ of year $t-1$ and month $m$ of year $t$, the forecasters update their expectations in the following way:
\begin{equation}
	E_{ijt}^m(x_{jt})=(1-G_{ij}^m)\hat\rho_{ij}E_{ijt-1}^m(x_{jt-1})+G_{ij}^ms_{ijt}^m\label{eq:update}
\end{equation}
where $G_{ij}^m$ is the Kalman gain that is consistent with forecaster $i$' beliefs about the persistence of $x_{jt}$ and about the precision of their signals, and $s_{ijt}^m$ is a ``synthetic'' signal built out of the public and private signals: $s_{ijt}^m=h_{ij}^m\phi_{jt}^m+(1-h_{ij}^m)\varphi_{ijt}^m$, with $h_{ij}^m=\kappa_{j}^m/(\kappa_{j}^m+\hat\tau_{ij}^m)$. %, so that $E_{ijt}^m(x_{jt}|\phi_{jt}^m,\varphi_{ijt}^m)=(\kappa_{j}^m+\hat\tau_{ij}^m)/(\gamma_j+\kappa_{j}^m+\hat\tau_{ij}^m)s_{ijt}^m$. Note that this synthetic signal can be written as $s_{ijt}^m=x_{jt}+v_{ijt}^m$ with $v_{ijt}^m=h_{ij}^m(\kappa_j^m)^{-1/2}u_{jt}^m+(1-h_{ij}^m)(\tau_{ij}^m)^{-1/2}e_{ijt}^m$ an average of the private and public noises.

%Notice that $E_{ijkt}(x_{jt})$ can be rewritten in its moving-average form as follows:
%\begin{equation}E_{ijkt}(x_{jt})=\frac{G_{jk}}{1-(1-G_{jk})\hat\rho_{jk}L}s_{ijkt}\label{eq:ma}\end{equation}
We define the forecast revisions between month $m$ of year $t-1$ and month $m$ of year $t$ as
$Revision_{ijt}^m=E_{ijt}^m(x_{jt})-E_{ijt-1}^m(x_{jt})$ and the error as $Error_{ijt,t}^m=x_{jt}-E_{ijt}^m(x_{jt})$, as before.
%with $\eta_{ijkt}=h_{jk}\kappa_j^{-1/2}u_{jt}+(1-h_{jk})\tau_{jk}^{-1/2}e_{ijt}$ is the total noise.


%\paragraph{Assumptions} We will assume in our analysis that for a given country $j$, the information and behavioral bias structure differs between local and foreign forecasters, but it is homogeneous within the local and foreign forecaster pools:
%\begin{assumption}[Within-location homogeneity]\label{ass:loc_hom} $\tau_{ij}^m=\tau_{jl}^m$, $\hat\tau_{ij}^m=\hat\tau_{jl}^m$ and $\hat\rho_{ij}=\hat\rho_{jl}$ if $i\in\mathcal{S}^l(j)$ and $\tau_{ij}^m=\tau_{jf}^m$, $\hat\tau_{ij}^m=\hat\tau_{jf}^m$ and $\hat\rho_{ij}=\hat\rho_{jf}$ if $i\in\mathcal{S}^f(j)$, for all $j=1,..J$ and $m=1,..,12$.
%\end{assumption}

%A stricter assumption will consist in assuming that local and foreign forecasters have the same behavioral biases regarding country $j$, and differ only regarding their information structure:
%\begin{assumption}[Homogeneous biases]\label{ass:hom} $\hat\rho_{ij}=\hat\rho_{i'j}=\hat\rho_{j}$ and $(\tau_{ij}^m)^{-1}-(\hat\tau_{ij}^m)^{-1}=(\tau_{i'j}^m)^{-1}-(\hat\tau_{i'j}^m)^{-1}$ for all $(i,i')\in(\textit{S}(j))^2$, $j=1,..J$ and $m=1,..,12$.
%\end{assumption}

%Finally, it will be useful to consider the assumption that there are no behavioral biases:
%\begin{assumption}[No biases]\label{ass:nobias} $\hat\rho_{ij}=\rho_j$ and $\hat\tau_{ij}^m=\tau_{ij}^m$, for all $i\in\mathcal{S}(j)$, $j=1,..J$ and $m=1,..,12$.
%\end{assumption}


\subsubsection{Variance of errors}

Consider the case with no behavioral biases, so that the following assumption is satisfied:
\begin{assumption}[No behavioral biases]\label{ass:nobias} $\hat\rho_{ij}=\rho_j$ and $\hat\tau_{ij}^m=\tau_{ij}^m$, for all $i\in\mathcal{S}(j)$, $j=1,..J$ and $m=1,..,12$.
\end{assumption}
Under this assumption, forecasters with less precise information make more errors on average. This derives from the forecasters' optimal use of information. In fact, the variance of errors can be related to the precision of private signals, as stated in the following proposition (see proof in the online Appendix \ref{proof:variance}):
\begin{prop}\label{prop:variance} Under Assumption \ref{ass:nobias} (no behavioral biases), the variance of forecast errors $V(Error_{ijt,t-1}^m)$ and $V(Error_{ijt,t}^m)]$ are decreasing in $\tau_{ij}^m$.
\end{prop}
The variance of errors decreases with the precision of the private signals. Therefore, the difference in variance between local and foreign forecasters can be explained by local forecasters' more precise private information. But asymmetric information is not the only potential source of difference in variance. The Kalman filter is a minimum mean-square error estimator. Mis-specified parametric inputs to the estimator will increase the variance of errors as compared to the well-specified estimator. Therefore, the difference in variance may be due to differences in behavioral biases. In the remainder of the section, we use model-based tests to detect differences in behavioral biases and differences in information. 

\subsection{Testing for Differences in Behavioral Biases}

\paragraph{BGMS regressions.}

We rely on regressions popularized by \citet{Bordaloetal2020} and \citet{BroerKohlhas2019} to assess the presence of behavioral biases among forecasters:
\begin{equation}Error_{ijt}^m=\beta^{BGMSm}_{ij}Revision^m_{ijt}+\delta_{ij}^m+\lambda_{ijt}^m\label{eq:BGMS}
\end{equation}
where $\beta^{BGMSm}_{ij}$ is a country, forecaster and month-specific coefficient, $\delta_{ij}^m$ are country-forecaster-month fixed effects and $\lambda_{ijt}^m$ is an error term. Following \citet{Angeletosetal2020}, we can show that $\beta^{BGMSm}_{ij}$ is related to the deviations of the beliefs $\hat\rho_{ij}$ and $\hat\tau_{ij}$ from their true counterparts (see the proof in online Appendix \ref{proof:BGMS}):
\begin{prop}\label{prop:BGMS} The coefficient $\beta^{BGMSm}_{ij}$, estimated by OLS, can be approximated at the first-order around $\hat\rho_{ij}=\hat\rho_{j}=\rho_{j}$ and $(\hat\tau_{ij}^m)^{-1}=(\tau_{ij}^m)^{-1}=(\tau_j^m)^{-1}$, where $\tau_j^m$ is the average level of precision, as follows:
$$\beta^{BGMSm}_{ij}\simeq -(\hat\rho_{ij}-\rho_{j})\hat\beta_{1j}^m- [(\tau_{ij}^m)^{-1}-(\hat\tau_{ij}^m)^{-1}]\hat\beta_{2j}^m$$
where $\hat\beta_1^m$ and $\hat\beta_2^m$ are strictly positive and independent of $\hat\rho_{ij}$, $\tau_{ij}^m$ and $\hat\tau_{ij}^m$.
\end{prop}
A negative coefficient reflects an over-reaction of forecasters to their information. This over-reaction can arise from over-confidence ($\hat\tau_{ij}^m-\tau_{ij}^m>0$) or from over-extrapolation ($\hat\rho_{ij} -\rho_j>0$).
%Differing levels of information precision can affect the BGMS coefficient, but these effects are proportional to the behavioral biases. For small levels of biases ($\hat\rho_j-\rho_j$ and $\hat\tau_{jk}^m-\tau_{jk}^m$ small), differences in coefficients across groups must be related to differences in over-confidence $(\hat\tau_{jf}^m-\tau_{jf}^m)-(\hat\tau_{jl}^m-\tau_{jl}^m)$. We make the assumption that the behavioral biases are small enough so that differences in $\hat\tau_{jk}$ do not affect the coefficient significantly.
 Therefore, a more negative BGMS coefficient will be interpreted as reflecting differences in either over-confidence or over-extrapolation.

We estimate Equation \eqref{eq:BGMS} using the mean-group methodology, under the assumption that the $\beta^{BGMS}$ coefficients could differ across each month and each country-forecaster pair. We collect the country-forecaster-month-specific $\beta^{BGMS}$ coefficients and test for significant differences between local and foreign forecasters by regressing the coefficient on the Foreign dummy, controlling for country-month and forecaster-month fixed effects. A significant coefficient for the Foreign dummy would indicate that there are systematic differences in behavioral biases. We restrict the sample to the pairs providing forecasts for at least 10 years and weight observations by the inverse of the coefficient's standard deviation to give more weight to the more precisely estimated coefficients.


The results are displayed in Columns (1) and (2) of Table \ref{tab:tab_main}. There is no systematic difference between local and foreign forecasters. Interestingly, the average $\beta^{BGMS}$ coefficient is positive for both inflation and GDP growth in our most conservative specification, suggesting that forecasters under-react to news on average.\footnote{This might seem in contradiction with previous evidence, which has found over-reaction, especially for inflation \citep{Bordaloetal2020,BroerKohlhas2019,Angeletosetal2020}. However, previous evidence has focused on the Survey of Professional Forecasters, which provides forecasts for the US. Our estimated parameters are in fact heterogeneous, especially across countries (see Figure \ref{app:fig:dist_beta_cty} in the online Appendix). Focusing on the US, we find that the inflation forecasts feature over-reaction on average, which is consistent with previous evidence. GDP growth forecasts do not feature systematic over- or under-reaction, which is also consistent with previous evidence.}

\begin{landscape}
	% TABLE
	\enlargethispage{2em}
	{\setstretch{1}
		\begin{table}[H] \centering
\newcolumntype{C}{>{\centering\arraybackslash}X}

\caption{Behavioral Biases and Information Asymmetries}
\label{tab:tab_main}
{\footnotesize
\begin{tabularx}{\linewidth}{l C C m{0.005\textwidth} C C m{0.005\textwidth} C C m{0.005\textwidth} C C m{0.005\textwidth} C C}

\toprule
&\multicolumn{5}{c}{Behavioral biases}&& & &&\multicolumn{5}{c}{Information asymmetries} \tabularnewline \cline{2-6} \cline{11-15}\tabularnewline &\multicolumn{2}{c}{$\beta^{BGMS}$}&&\multicolumn{2}{c}{$\hat\rho$}&&\multicolumn{2}{c}{$\beta^{CG}$}&&\multicolumn{2}{c}{$\beta^{FE}$}&&\multicolumn{2}{c}{$\beta^{Dis}$} \tabularnewline \cline{2-3} \cline{5-6} \cline{8-9} \cline{11-12} \cline{14-15}\tabularnewline &{(1)}&{(2)}&&{(3)}&{(4)}&&{(5)}&{(6)}&&{(7)}&{(8)}&&{(9)}&{(10)} \tabularnewline
{Coefficient}&{$ \text{CPI}_{t} $}&{$ \text{GDP}_{t} $}&{}&{$ \text{CPI}_{t} $}&{$ \text{GDP}_{t} $}&{}&{$ \text{CPI}_{t} $}&{$ \text{GDP}_{t} $}&{}&{$ \text{CPI}_{t} $}&{$ \text{GDP}_{t} $}&{}&{$ \text{CPI}_{t} $}&{$ \text{GDP}_{t} $} \tabularnewline
\midrule \addlinespace[0pt]
\midrule Average&&&&&&&&&&&&&--0.08***&--0.07*** \tabularnewline
&&&&&&&&&&&&&(0.03)&(0.02) \tabularnewline
Average Locals&0.01**&0.04***&&0.40***&0.38***&&0.04***&0.10***&&--0.26***&--0.32***&&& \tabularnewline
&(0.01)&(0.01)&&(0.01)&(0.01)&&(0.00)&(0.01)&&(0.00)&(0.01)&&& \tabularnewline
$ \text{Foreign} $&--0.01&0.03&&0.03&0.04&&--0.00&--0.01&&--0.04***&--0.02*&&& \tabularnewline
&(0.02)&(0.02)&&(0.02)&(0.02)&&(0.01)&(0.01)&&(0.01)&(0.01)&&& \tabularnewline
N&3,067&3,333&&3,937&4,196&&1,214&1,224&&1,136&1,160&&592&604 \tabularnewline
$ R^2 $&0.65&0.72&&0.63&0.72&&0.87&0.90&&0.85&0.76&&0.00&0 \tabularnewline
Country $\times$ month FE&\checkmark&\checkmark&&\checkmark&\checkmark&&\checkmark&\checkmark&&\checkmark&\checkmark&&& \tabularnewline
Forecaster $\times$ month FE&\checkmark&\checkmark&&\checkmark&\checkmark&&&&&&&&& \tabularnewline
MG by ctry and month&&&&&&&&&&&&&\checkmark&\checkmark \tabularnewline
MG by ctry, loc., and month&&&&&&&\checkmark&\checkmark&&\checkmark&\checkmark&&& \tabularnewline
MG by ctry, for., and month&\checkmark&\checkmark&&\checkmark&\checkmark&&&&&&&&& \tabularnewline
\bottomrule \addlinespace[\belowrulesep]

\end{tabularx}
\begin{flushleft}
\footnotesize \begin{minipage}{1.35\textwidth} \vspace{-10pt} \begin{tabnote} \textit{Notes:} Columns (1) and (2) show the results of a regression of the $\beta^{BGMS}$ coefficients on the Foreign dummy, where the $\beta^{BGMS}$ are estimated using Equation \eqref{eq:BGMS} on different sub-groups of our sample. Columns (3) and (4) show the results of a regression of the perceived autocorrelation coefficients $\hat\rho$ on the Foreign dummy, where the $\hat\rho$ is estimated using Equation \eqref{eq:rhohat} on different sub-groups of our sample. Column (5) and (6) show the results of a regression of the $\beta^{CG}$ coefficients on the Foreign dummy, where the $\beta^{CG}$ are estimated using equation \eqref{eq:consensus} on different sub-groups of our sample. Columns (7) and (8) show the results of a regression of the $\beta^{FE}$ coefficients on the Foreign dummy, where the $\beta^{FE}$ are estimated using Equation \eqref{eq:pooledFE} on different sub-groups of our sample. In columns (1) to (8), \textit{Average locals} corresponds to the constant term (or average fixed effect). \textit{Foreign} corresponds to the coefficient of the Foreign dummy. Columns (9) and (10) show the results of a regression of the $\beta^{Dis}$ coefficients on the constant, where the $\beta^{Dis}$ are estimated using Equation \eqref{eq:disagreement} on different sub-groups of our sample.  corresponds to the constant term. Standard errors are clustered at the country and forecaster levels in Columns (1) to (4). Standard errors are clustered at the country and forecaster levels in Columns (5) to (10). All observations are weighted by the inverse of the estimated standard error of the corresponding $\beta$. \end{tabnote} \end{minipage}  
\end{flushleft}
}
\end{table}

	}
\end{landscape}



\paragraph{Perceived persistence.}

A non-negative BGMS coefficient can arise both from distorted beliefs on the precision of private signals and from distorted beliefs on the persistence of the shocks. We have shown that these BGMS coefficients do not differ systematically between local and foreign forecasters. However, this does not imply that foreign forecasters have similar over-(under-)confidence and over-(under-)extrapolation. A similar result would arise if the relative over-(under-)confidence of foreign forecasters compensates their relative over-(under-)extrapolation.  We examine more directly whether the beliefs on persistence are similar. To do this, we use the relation between the forecasts on current and future variables implied by our model:
\begin{equation}E_{ijt}^m(x_{jt+1})=\hat\rho_{ij}E_{ijt}^m(x_{jt})\label{eq:rhohat}
\end{equation}
We estimate Equation \eqref{eq:rhohat} using the same mean-group methodology. In our model, $\hat\rho_{ij}$ is specific to a country-forecaster pair, but we allow it to differ across months as well.\footnote{In our model, all the innovations to inflation have the same persistence, whereas in reality, there could be some components of inflation that are purely transitory. We cannot exclude that forecasters learn about the transitory component over the year. That would affect the month-specific correlation between the current and the future forecast.}

The results are reported in Columns (3) and (4) of Table \ref{tab:tab_main}. The estimated perceived persistence is not significantly different for foreign forecasters.\footnote{In online Appendix \ref{tab:tab_rob_BGMS} and \ref{tab:tab_rob_overextr}, we provide the results when assuming that the $\beta^{BGMS}$ and the $\hat\rho$ coefficients only differ across countries and between local and foreign forecasters, when assuming that they differ only across country-forecaster pairs, and when using different sets of fixed effects. The results are consistent across specifications.}

In Tables \ref{tab:tab_rob_past_consensus} and \ref{tab:tab_rob_last_vintage} in the online Appendix, following \citet{BroerKohlhas2019} and \citet{GemmiValchev2021}, we examine over-(under-)reaction to public news, by examining regressions of forecast errors on public news, using two different measures of public news: the past consensus and the last vintage of realized outcome. A negative (positive) coefficient implies that forecasters over-react (under-react) to public news. Again, we do not find any systematic difference in behavioral biases.\footnote{Consistently with \citet{BroerKohlhas2019}, who find evidence of both under-reaction and over-reaction to salient public news, we find both under-reaction or over-reaction to public news (in Columns (5) and (6) of both tables, which include the largest set of fixed effects, we can see that forecasters under-react to the last vintage and past consensus on inflation and over-react to the last vintage and past consensus on GDP growth, as the average coefficient is positive for inflation and negative for GDP growth).} Finally, we also show that forecasters do not have a different systematic bias in their forecasts (see Table \ref{tab:tab_rob_bias} in the online Appendix).\footnote{Interestingly, we find evidence of a systematic positive bias in inflation expectations: forecasters systematically overestimate inflation by 0.02 percentage points, which is statistically significant but small (see Column (7) of Table \ref{tab:tab_rob_bias}, our baseline specification). Forecasters tend to do the opposite with GDP growth forecasts, as they tend to underestimate it systematically by 0.04 percentage points (see Column (8)). The Foreign dummy, however, is not significant in both Columns (7) and (8), which shows that these systematic biases are similar across locations.}

All in all, foreign and local forecasters do not have significantly different biases. From now on, we thus assume common behavioral biases across forecasters:
%A stricter assumption will consist in assuming that local and foreign forecasters have the same behavioral biases regarding country $j$, and differ only regarding their information structure:
\begin{assumption}[Homogeneous behavioral biases across locations]\label{ass:hom} $\hat\rho_{ij}=\hat\rho_{i'j}=\hat\rho_{j}$ and $(\tau_{ij}^m)^{-1}-(\hat\tau_{ij}^m)^{-1}=(\tau_{i'j}^m)^{-1}-(\hat\tau_{i'j}^m)^{-1}$ for all $(i,i')\in(\textit{S}(j))^2$, $j=1,..J$ and $m=1,..,12$.
\end{assumption}
Note that Assumption \ref{ass:nobias} (no biases)  is a special case of Assumption \ref{ass:hom}.
In the next sub-section, we examine differences in information frictions under this assumption.


\subsection{Testing for Asymmetric Information}

Consensus regressions that consist in regressing the consensus error (i.e., the average error) on the consensus revision (i.e., the average revision) as in \citet{CoibionGorodnichenko2015} are commonly used to detect information frictions. A positive coefficient indicates deviations from full information. Can we use these regressions to identify differences in information frictions between local and foreign forecasters?  We show here that differences in the coefficient of the consensus regression are not a good indicator of the degree of information asymmetry. We thus propose two alternative tests that are robust to public signals.

\subsubsection{Consensus regressions}

%Consensus regression as in \citet{CoibionGorodnichenko2015} are commonly used to detect information frictions. Can we use these regressions to identify differences in information frictions between local and foreign forecasters?

Suppose that we perform the consensus regression as in \citet{CoibionGorodnichenko2015} on both group of forecasters, that is, using the population of foreign forecasts on the one hand and the population of the local forecasts on the other, and then compare the coefficients. Would that comparison tell us which group is better informed?

In our setup, this regression can be written, for each $j=1,..J$, $m=1,..12$ and $k=l,f$, where $l$ refers to the local forecasters' population and $f$ refers to the foreign forecasters' population:
\begin{equation}Error_{jkt}^m=\beta^{CGm}_{jk}Revision_{jkt}^m+\delta_{jk}^m+\lambda_{jkt}^m\label{eq:consensus}
\end{equation}
$Error_{jkt}^m=\frac{1}{N^k(j)}\sum_{i\in\textit{S}^k(j)}Error_{ijt}^m$, $Revision_{jkt}^m=\frac{1}{N^k(j)}\sum_{i\in\textit{S}^k(j)}Revision_{ijt}^m$, are the consensus error and the consensus revision in location $k=l,f$, $\delta_{jk}^m$ are country-month-location fixed effects and $\lambda_{jkt}^m$ is an error term.

Columns (5) and (6) of Table \ref{tab:tab_main} display the results of the estimation of $\beta^{CGm}_{jk}$ using the mean-group estimator, under the assumption that the coefficients $\beta^{CGm}_{jk}$ differ across countries, months and locations. While the $\beta^{CGm}_{jk}$ coefficient is positive on average, as is expected, there is no significant difference between foreign and local coefficients.\footnote{In Appendix \ref{tab:tab_rob_consensus}, we provide the results when assuming that the $\beta^{CG}$ coefficients only differ across countries and locations. The results are very stable across specifications.}

This does not necessarily mean that there are no information asymmetries between local and foreign forecasters. To show this, it will be useful to assume that for a given country $j$, the information and behavioral bias structure differs between local and foreign forecasters, but it is homogeneous within the local and foreign forecaster pools:
\begin{assumption}[Homogeneous behavioral biases and precision within-location]\label{ass:loc_hom} $\tau_{ij}^m=\tau_{jl}^m$, $\hat\tau_{ij}^m=\hat\tau_{jl}^m$ and $\hat\rho_{ij}=\hat\rho_{jl}$ if $i\in\mathcal{S}^l(j)$ and $\tau_{ij}^m=\tau_{jf}^m$, $\hat\tau_{ij}^m=\hat\tau_{jf}^m$ and $\hat\rho_{ij}=\hat\rho_{jf}$ if $i\in\mathcal{S}^f(j)$, for all $j=1,..J$ and $m=1,..,12$.
\end{assumption}
We additionally assume the absence of behavioral bias (Assumption \ref{ass:nobias}) for tractability. The following proposition shows that, in the presence of public information, the relation between $\beta^{CG}$ and the precision of private information is not monotonic (see the proof in Appendix \ref{proof:consensus}), even in the absence of behavioral biases.

\begin{prop}\label{prop:consensus} Suppose that Assumptions \ref{ass:nobias} and \ref{ass:loc_hom} are satisfied: there are no behavioral biases and the precision parameters are identical within foreign forecasters and within local forecasters. The coefficients $\beta^{CGm}_{jk}$, estimated by OLS, are non-monotonous in $\tau_{jk}^m$.
%\begin{itemize}
%\item[(i)] decreasing in $\tau_{jk}^m$ when $\kappa_{j}^m=0$;
%\item[(ii)] increasing in $\tau_{jk}^m$ around $\tau_{jk}^m=0$.
%\end{itemize}
\end{prop}
To understand, consider first the limit case studied by \citet{CoibionGorodnichenko2015} with no public information ($\kappa_j^m=0$), where $\beta^{CGm}_{jk}=(1-G_{jk}^m)/G_{jk}^m$. The coefficient is directly related to the Kalman gain. A large coefficient implies a small Kalman gain and hence noisier information. Therefore, if foreigners have noisier, less precise information ($\tau_{jl}^m>\tau_{jf}^m$), we would have $\beta^{CGm}_{jl}<\beta^{CGm}_{jf}$. But with public information, this relation can be reversed. Suppose that local forecaster have access to a private signal on top of the public signal, while foreign forecasters only observe the public signal ($\tau_{jf}^m$ goes to zero). For foreign forecasters, the public signal becomes the only valid signal, so that all foreign forecasters share the same information. In that case, rational expectations imply that $\beta^{CGm}_{jf}=0$. Since $\beta^{CGm}_{jl}$ is strictly positive, we would observe this time $\beta^{CGm}_{jl}>\beta^{CGm}_{jf}$, while foreigners still have less precise information.

%However, is the presence of a public signal, that is, when $\kappa_j^m>0$, $\beta^{CGm}_{jk}$ is not a straightforward function of the information structure, so it is not clear what to infer from $\beta^{CGm}_{jl}<\beta^{CGm}_{jf}$. This is due to the presence of aggregate noise. This aggregate noise, as discussed in \citet{CoibionGorodnichenko2015}, introduces a negative bias in the estimation of $\beta_{jk}^{CGm}$. While the correlation between the error and the revision driven by the fundamental $x_{jt}$ is positive, the public noise introduces a negative correlation. CG argue that because the bias is negative, a positive coefficient is still a sign of noisy information. However, in order to test for \emph{differences} in the quality of private information by comparing $\beta^{CGm}_{jl}$ and $\beta^{CGm}_{jf}$, we need $\beta^{CGm}_{jk}$ to be a monotonic function of $\tau_{jk}^m$.

%But the case described in (ii) shows that, in fact, $\beta^{CGm}_{jl}$ is not always decreasing in $\tau_{jk}^m$. Take the limit case where the precision of the private signal is vanishing ($\tau_{jk}^m$ goes to zero). In that case, the public signal becomes the only valid signal, so that all forecasters share the same information, which corresponds to the public information. In that case, rational expectations imply that the $\beta^{CGm}_{jl}$ coefficient goes to zero. Since $\beta^{CGm}_{jl}$ is otherwise strictly positive, this means that $\beta^{CGm}_{jl}$ is locally increasing in $\tau_{jk}^m$ in the vicinity of $\tau_{jk}^m=0$. In that case, $\beta^{CGm}_{jl}<\beta^{CGm}_{jf}$ would imply that foreigners have more precise information ($\tau_{jf}^m<\tau_{jl}^m$).

We thus need tests that identify the degree of information frictions and that are robust to public information. We propose two such tests.

\subsubsection{Fixed-effect regressions}

For our first test of asymmetric information, we use fixed-effect regressions. We use the following pooled regression, for each $j=1,..J$, $m=1,..,12$ and $k=l,f$:
\begin{equation}Error_{ijkt}^m=\beta^{FEm}_{jk}Revision_{ijkt}^m+\delta_{jkt}^m+\lambda_{ijkt}^m\label{eq:pooledFE}
\end{equation}
where $\delta_{jkt}^m$ are country-location-time fixed effects and $\lambda_{ijkt}^m$ is an error term. The difference between this regression and the BGMS regression is that it controls for time fixed effects. These fixed effects control for aggregate shocks ($\epsilon_{jt}$ and $u_{jt}$), which are not observed by forecasters when they revise their forecasts. The coefficient $\beta^{FEm}_{jk}$ is thus not an indicator of the deviation from rational expectations, but rather captures the cross-sectional covariance between the errors and the revisions. This coefficient is necessarily negative: optimistic forecasters make a more negative error than pessimistic forecasters. The following proposition shows that, if the biases are homogeneous across groups, then a more negative $\beta^{FEm}_{jk}$ indicates noisier information (see proof in Appendix \ref{proof:pooledFE}).
\begin{prop}\label{prop:pooledFE} Suppose that Assumptions \ref{ass:hom} and \ref{ass:loc_hom} are satisfied: forecasters have identical behavioral biases and the precision parameters are homogeneous within foreign forecasters and within local forecasters.
Consider the $\beta^{FEm}_{jk}$ coefficients, estimated by OLS. If $0<\hat\rho_j<1$, then $\beta^{FEm}_{jf}<\beta^{FEm}_{jl}$ if and only if $\tau_{jl}^m>\tau_{jf}^m$.
\end{prop}
If the foreign and local forecasters have similar behavioral biases and if forecasters believe that there is some persistence in the process, then $\beta^{FEm}_{jf}<\beta^{FEm}_{jl}$ reflects an informational advantage for locals.\footnote{Note that adding time fixed effects to the regression is equivalent to subtracting the cross-forecaster average from each side of the equation:
$$-\left(E_{ijkt}^m(x_{jt})-E_{jkt}^m(x_{jt})\right)=\beta^{FEm}_{jk}(Revision_{ijkt}^m-Revision_{jkt}^m)+\lambda_{ijkt}^m$$
In that sense, this test is similar in spirit to \citet{Goldstein2021}, who proposes to measure information frictions by estimating the persistence of a forecaster's deviation from the mean:
$$\left(E_{ijkt}^m(x_{jt})-E_{jkt}^m(x_{jt})\right)=\beta^{Gm}_{jk}\left(E_{ijkt-1}^m(x_{jt})-E_{jkt-1}^m(x_{jt})\right)+\lambda_{ijkt}^m$$
$\beta^{Gm}_{jk}=1-G^{m}_{jk}$ is also directly and monotonically related to the degree of information frictions.}

%In columns (1) and (2), we assume that $\beta^{CGm}_{jk}$ differs across countries and locations. In columns (3) and (4), we assume that $\beta^{CGm}_{jk}$ can also differ across months. There does not appear to be any significant difference between foreign and local coefficients.

We estimate Equation \eqref{eq:pooledFE} under the assumption that the $\beta^{FE}$ coefficients differ across countries, locations, and months. We then regress these coefficients on the Foreign dummy and report the results in Columns (7) and (8) of Table \ref{tab:tab_main}. Note first that the estimated coefficients are negative on average, as predicted. Second, the coefficient for Foreign dummy is significantly negative for inflation. For GDP growth, it is negative as well, but smaller in magnitude and not significant. This is consistent with the preliminary evidence of Section \ref{sec:mistakes} where we have shown that foreign forecasters made relatively larger errors on inflation than on GDP growth.\footnote{In Appendix \ref{tab:tab_rob_FE}, we provide the results when assuming that the $\beta^{FE}$ coefficients only differ across countries and locations. The results are very stable across specifications.}

\subsubsection{Foreign-local disagreement}

Our second test of asymmetric information is based on how disagreement between local and foreign forecasters reacts to public information.
We define the disagreement between the local and foreign forecasters as follows:
\begin{equation}
Disagreement_{jt}^m=E_{jlt}^m(x_{jt})-E_{jft}^m(x_{jt})
\label{eq:dis}
\end{equation}
where $E_{jkt}^m(x_{jt})=\frac{1}{N(j)^k}\sum_{i\in\textit{S}^k(j)}E_{ijkt}(x_{jt})$, $k=l,f$, is the location-specific average expectation.

Consider now the following regression:
\begin{equation}Disagreement_{jt}^m=\beta^{DISm}_{j}Revision_{jt}^m+\beta^{0m}_{j}x_{jt}+\beta^{2m}_jE_{jlt-1}^m(x_{jt})+\beta^{3m}_jE_{jft-1}^m(x_{jt})+\delta_{j}^m+\lambda_{jt}^m\label{eq:disagreement}\end{equation}
where $Revision_{jt}^m=\frac{1}{2}(Revision_{jlt}^m+Revision_{jft}^m)$ is the average of local and foreign consensus revisions for country $j$ in year $t$ and month $m$.

We can show that the sign of $\beta^{DISm}_{j}$ depends on the precision of local forecasters' information relative to foreign forecasters when the behavioral biases are homogeneous across locations (see proof in Appendix \ref{proof:disagreement}).
\begin{prop}\label{prop:disagreement} Suppose that Assumptions \ref{ass:hom} and \ref{ass:loc_hom} are satisfied: forecasters have identical behavioral biases and the precision parameters are homogeneous within foreign forecasters and within local forecasters. Consider the coefficients $\beta^{DISm}_{j}$, estimated by OLS. Then $\beta^{DISm}_{j}<0$ if and only if $\tau_{jl}^m>\tau_{jf}^m$.
\end{prop}
Intuitively, because we control for the fundamental $x_{jt}$, $\beta^{DISm}_{j}$ captures the reaction of disagreement to the public noise. As a consequence, $\beta^{DISm}_{j}$ is negative if the foreign expectations are more sensitive to the public signal. This is the case if the foreign forecasters' private information is less informative than that of local forecasters.

We estimate Equation \eqref{eq:disagreement} under the assumption that the $\beta^{Dis}$ coefficients differ across countries and across months.\footnote{In Appendix \ref{tab:tab_rob_disag}, we provide the results when assuming that the $\beta^{DIS}$ coefficients only differ across countries. The results are very stable across specifications} We then test whether the coefficients are different from zero on average and report the results in columns (9) and (10) of Table \ref{tab:tab_main}. The disagreement coefficients are significantly negative on average for both inflation and GDP growth. The coefficient is smaller for GDP growth, which is consistent with our previous results.\footnote{Note that because $\beta^{FE}$ and $\beta^{Dis}$ are based on country-level regressions, we cannot control for forecaster fixed effects in Columns (7) to (10). The Foreign coefficient could then be driven by the fact that the local forecasts are more likely to be produced by multinational firms. To circumvent this endogeneity issue, we replicate columns (7) to (10) using only forecasts produced by non-multinational firms. The results are provided in Table \ref{tab:tab_r3} in the Appendix and remain very similar.}

\subsection{Robustness analysis}

We perform robustness tests where (i) we use alternative vintage series to compute the forecast errors, (ii) we include only forecasters who produce forecasts for both local and foreign forecasts, (iii) we use alternative trimming strategies, (iv) we exclude forecasts that are identical to their previous release. We replicate the results of Table \ref{tab:updating_errors_main_small} and Table \ref{tab:tab_main} under these different specifications.  Finally, (v) we use \citet{Driscoll1998} standard errors for our our main panel estimation of Table \ref{tab:updating_errors_main_small}, Columns (2) and (3). The details of the analysis is available in Section \ref{sec:robustness} in the Appendix. The results remain robust.

%The only exception is when we use an alternative definition of foreign forecasters. In our baseline analysis, a foreign forecaster is defined as a forecaster that has neither its headquarters nor any subsidiary located in the country. This definition suggests that local subsidiaries play a role in forecast formation. In this robustness check, we ignore the role of subsidiaries and deem a forecaster to be foreign if its headquarters are located in another country, whether a subsidiary is present in the country or not. Compared to the 28\% of foreign forecasters in the baseline results, 64\% of the forecasters are defined to be foreign according to this alternative definition. Overall, our results remain robust, even though they are less pronounced and more imprecisely estimated. Ignoring the role of subsidiaries weakens the results, which suggests that the local information gathered by subsidiaries matters. We investigate this issue further in Section \ref{sec:heterogeneity}, where we explore heterogeneity.


\section{The Geography of Information}\label{sec:geography}

The preceding sections have established that asymmetric information between local and foreign forecasters explains foreigners' excess errors. In this section, we use our multi-country, multi-forecaster panel to explore whether well-identified barriers to information acquisition drive the foreign penalty. We first rely on the multinationals' multiple locations to identify the role of having a local economic activity in the country in reducing information asymmetries. We then draw from the literature on the geography of trade and finance to identify other important drivers of information asymmetries.

\subsection{The Relevance of Local Subsidiaries}

In our analysis, we have chosen to consider a forecast to be ``foreign'' if the forecasting firm has neither their headquarters nor a subsidiary located in the country. While this assumption is innocuous for non-multinational firms, whose subsidiaries (if there are any) are all located in the same country as the headquarters, it is not the case for multinational firms. This relies on the implicit assumption that a forecaster with its headquarters in the country and a forecaster with only a subsidiary in the country are equally good at forecasting that country. We reconsider this assumption here, in order to also further understand the origin of the foreign penalty. In other words, is it important to have some activity in the country to gain an informational advantage, or is it mainly the location of the headquarters that matters?

To test for the relevance of local subsidiaries, we use an alternative definition of ``foreign'', where a forecaster is foreign if its headquarters are located in another country, whether a subsidiary is present or not.  Compared to the 28\% of foreign forecasters with the baseline definition, 64\% of the forecasters are defined to be foreign according to the alternative definition. The corresponding dummy is \textit{Foreign HQ}.

{\setstretch{1}`
\begin{table}[H] \centering
\newcolumntype{C}{>{\centering\arraybackslash}X}

\caption{Multinational and Non-Multinational Forecasters}
\label{tab:error_reg_labs_mult}
{\footnotesize
\begin{tabularx}{\linewidth}{l C C C C C}

\toprule
&\multicolumn{5}{c}{$\ln(|Error_{ijt,t}^m|)$}\tabularnewline\cline{2-6} &{(1)}&{(2)}&{(3)}&{(4)}&{(5)} \tabularnewline
{Coefficient}&{}&{Non-Mult.}&{Mult.}&{Mult.}&{Mult.} \tabularnewline
\midrule \addlinespace[0pt]
\midrule Foreign&0.06***&0.25***&0.04**&& \tabularnewline
&(0.02)&(0.07)&(0.02)&& \tabularnewline
Foreign HQ&&&&0.04&0.08** \tabularnewline
&&&&(0.02)&(0.03) \tabularnewline
Local Subsidiary&&&&&--0.04** \tabularnewline
&&&&&(0.02) \tabularnewline
N&389,295&57,888&313,197&313,197&313,197 \tabularnewline
$ R^2 $&0.70&0.77&0.71&0.71&0.71 \tabularnewline
Country $ \times $ Date $ \times $ Var. $ \times $ Hor. FE&\checkmark&\checkmark&\checkmark&\checkmark&\checkmark \tabularnewline
For. $ \times $ Date $ \times $ Var. $ \times $ Hor. FE &\checkmark&\checkmark&\checkmark&\checkmark&\checkmark \tabularnewline
\bottomrule \addlinespace[\belowrulesep]

\end{tabularx}
\begin{flushleft}
\footnotesize \begin{minipage}{1\textwidth} \vspace{-10pt} \begin{tabnote} \textit{Notes:}   The table shows the regression of the log absolute forecast error of current and future CPI and GDP on regressors on different sub-samples. All standard errors are clustered at the country, forecaster and date levels. \end{tabnote} \end{minipage}  
\end{flushleft}
}
\end{table}

}

%{\setstretch{1}
%	\input{Tables/error_reg_labs_mult_rob}
%}

We construct a single sample of observations by stacking observations of inflation and GDP growth errors at different horizons. We first estimate our baseline regression:
\begin{align}
	\ln(|Error_{ijt,t+h}^{m,x}|)= \delta_{it,h}^{m,x} +\tilde\delta_{jt,h}^{m,x} +  \beta \text{Foreign}_{ij} +\varepsilon_{ij,t}^m  \,, \label{eq:main}
\end{align}
where $x=growth, inflation$ and $h=0,1$ stand respectively for the forecast variable and the horizon, and $ \delta_{it,h}^{m,x}$ and $\tilde\delta_{jt,h}^{m,x}$ are forecaster-date-variable-horizon fixed effects and country-date-variable-horizon fixed effects. $\beta$ identifies the average effect of the \textit{Foreign} dummy across variables and horizons. The results with the baseline definition of \textit{Foreign} are reported in Column (1) of Table \ref{tab:error_reg_labs_mult}. The coefficient is highly significant and equal to 0.06. This reflects the average impact of \textit{Foreign} across variables and horizons.

We then report the estimates with our baseline definition when distinguishing multinational from non-multinational firms. By comparing Columns (2) and (3), we can see that non-multinational firms have a higher foreign penalty than multinational firms (0.25 against 0.04). This is not surprising, as multinational firms are likely to have more resources and departments dedicated to forecasting. The coefficient of multinational firms is closer to our main estimate, because, as we have seen below, multinationals account for the bulk of our sample. Note, however, that because the coefficient of \textit{Foreign} is imprecisely estimated for non-multinational firms, we cannot statistically distinguish it from the \textit{Foreign} coefficient of multinational firms. Panel (c) of Figure \ref{fig:heterogeneity} in the Appendix shows the distribution of the foreign penalty across forecasters. The foreign penalty is indeed heterogeneous across forecasters, and even more so across non-multinational forecasters.

In Column (4), we replace the \textit{Foreign} dummy with \textit{Foreign HQ}. \textit{Foreign HQ} is not significant. It is expected that, if our assumption that the location of subsidiaries matter is true, then \textit{Foreign HQ} would include a heterogeneous mix of purely foreign forecasters and forecasters who are present through their subsidiaries. To further test this, we add an additional dummy variable that accounts for the presence of a local subsidiary: \textit{Local Subsidiary} is equal to one if the forecaster has a subsidiary in the country, but no headquarters. In Column (5), now that we control for the presence of a local subsidiary, \textit{Foreign HQ} becomes significantly positive. Interestingly, the sum of the two coefficients is not statistically different from zero. This means that a forecaster with no headquarters, but with a subsidiary located in the country, is just as good at forecasting as a forecaster with headquarters located in the country. It is thus enough to have \emph{some} activity in the country to benefit from a significant ``local'' advantage. This also confirms that our main assumption is valid.\footnote{In Section \ref{sec:robustness} in the Appendix, we perform an additional robustness test where we use this alternative definition of foreign forecasters and replicate our main results. The results are weaker, which confirms that taking into account subsidiaries is important to identify the foreign penalty.}

%While this difference is not significant in our sample, this suggests that having headquarters located in the country tends to provide an informational advantage as compared to having only a subsidiary. This could be due to the fact that the specialized forecasting departments within a multinational are typically located in the headquarters, or because often headquarters concentrates a higher share of the multinational's operations and profits (which would be consistent with our evidence on incentives)

%Note that this analysis also rules out a potential endogeneity bias. Suppose that a firm opens a subsidiary in a country only if it has good quality information on this country. Then, making lower forecasting mistakes would be the cause and not the consequence of the presence of a subsidiary in the country. In our main analysis, we consider the firm to be ``local'' in this case. This would thus lead to a positive Foreign dummy coefficient, without the effect being causal. The alternative dummy ForeignHQ is not endogenous (at least as long as the headquarter decision is orthogonal to information), but it is significant and of the same magnitude as our main estimate.

%This analysis motivates us to replicate the main results of the paper with this richer specification. In Section \ref{sec:robustness} in the Appendix, we thus perform an additional robustness test where we use this alternative definition of foreign forecasters, and control for the presence of local subsidiaries. (TBD). %In our baseline analysis, a foreign forecaster is defined as a forecaster that has neither its headquarters nor any subsidiary located in the country. This definition suggests that local subsidiaries play a role in forecast formation. In this robustness check, we ignore the role of subsidiaries and deem a forecaster to be foreign if its headquarters are located in another country, whether a subsidiary is present in the country or not. Compared to the 28\% of foreign forecasters in the baseline results, 64\% of the forecasters are defined to be foreign according to this alternative definition. Overall, our results remain robust, even though they are less pronounced and more imprecisely estimated. Ignoring the role of subsidiaries weakens the results, which suggests that the local information gathered by subsidiaries matters. We investigate this issue further in Section \ref{sec:heterogeneity}, where we explore heterogeneity.

\subsection{Barriers to Information}


To assess the role of the barriers to information, we use the same single sample and now estimate equations of the following form:
\begin{align}
	\ln(|Error_{ijt,t+h}^{m,x}|)= \delta_{it,h}^{m,x} +\tilde\delta_{jt,h}^{m,x} +  \beta \text{Foreign}_{ij} +\gamma X_{ijt}^{m,x}+ \varepsilon_{ij,t}^m  \,, \label{eq:geography}
\end{align}

where $X_{ijt}^{m,x}$ is a set of variables that have a forecaster-country dimension.

We build on the trade and capital flow literature that has identified information as one of the impediments to the cross-border circulation of products and capital. This literature has shown that geography, and in particular distance, retains a high explanatory power for both bilateral trade and bilateral holdings of financial assets. While in the case of trade, information is only one of the many reasons why distance plays a role, the main one being transportation costs,\footnote{See \citet{AndersonvanWincoop2004,HeadMayer2013,Allen2014}.} in the case  of capital flows,  the cost of information has been the main interpretation of the role of distance.\footnote{See \citet{Ghosh2000,GrinblattKeloharju2001,DiGiovanni2005,PortesRey2005}.}

Following \citet{Pellegrino2021}, we consider a parsimonious set of geographical variables: physical, cultural, and linguistic distance. \textit{Physical Distance} measures the geodesic distance between two countries, based on a population-weighted average of the distances between capital cities. \textit{Cultural Distance} captures distance in contemporary values and beliefs, introduced by \citet{Spolaore2016}, and \textit{Linguistic Distance} captures distance in spoken languages, introduced by \citet{Fearon2003} and constructed by \citet{Spolaore2016}.\footnote{See Appendix \ref{app:descvarbarriers} for a detailed description and source of the variables.} Note that, since forecasters are better at forecasting their domestic variables, then they could also be better at forecasting a country whose business cycles are correlated with the domestic business cycles. We thus test whether forecasting is facilitated by business cycle comovement. To do so, we use the squared correlation between the GDP growth (or inflation) of the forecaster country and the GDP growth (or inflation) of country $j$. Squaring the correlation gives the same weight to a large positive correlation and to a large negative correlation (in tables, we abbreviate the variable as \textit{BC comovement}).\footnote{Note that this corresponds simply to the $R^2$ of a regression of the GDP growth rate (or inflation) of country $j$ on the GDP growth rate (or inflation) of the forecaster country.} We expect a higher comovement to have a negative impact on the error. Finally, we add \textit{Migration}, the share of the forecaster country population that was born in country $j$. For all these variables, the forecaster country is defined as the country of the forecaster's closest subsidiary.

%\clearpage\begin{landscape}
%	{\setstretch{1}
%		\begin{table}[H] \centering
\newcolumntype{C}{>{\centering\arraybackslash}X}

\caption{The Geography of Information}
\label{tab:error_reg_labs_gravity}
{\footnotesize
\begin{tabularx}{\linewidth}{l C C C C C C C m{0.005\textwidth} C C C m{0.005\textwidth} C C}

\toprule
&\multicolumn{14}{c}{$\ln(|Error_{ijt,t}^m|)$}\tabularnewline\cline{2-15} \tabularnewline  \cline{2-8} \cline{10-12} \cline{14-15}\tabularnewline  &{(1)}&{(2)}&{(3)}&{(4)}&{(5)}&{(6)}&{(7)}&&{(8)}&{(9)}&{(10)}&&{(11)}&{(12)} \tabularnewline
{Coefficient}&{}&{}&{}&{}&{}&{}&{}&{}&{Finance}&{Finance}&{Finance}&{}&{}&{} \tabularnewline
\midrule \addlinespace[0pt]
\midrule Foreign&.0577***&.054***&.0356&.0262&.0727**&.059**&.0189&&.0389*&.0348*&.026&&.024&.079*** \tabularnewline
&(.0158)&(.0148)&(.022)&(.0249)&(.0295)&(.029)&(.024)&&(.0208)&(.0206)&(.0186)&&(.0287)&(.0256) \tabularnewline
\textbf{\emph{W.r.t. closest subs.:}} &&&&&&&&&&&&&& \tabularnewline
Physical dist.&&.0053&&&&&&&&&&&& \tabularnewline
&&(.0078)&&&&&&&&&&&& \tabularnewline
Cultural dist.&&&.0115&&&&&&&&&&& \tabularnewline
&&&(.0085)&&&&&&&&&&& \tabularnewline
Linguistic dist.&&&&.0189*&&&.0193*&&&&&&.022*& \tabularnewline
&&&&(.0108)&&&(.0103)&&&&&&(.0125)& \tabularnewline
BC comovement&&&&&.007&&&&&&&&& \tabularnewline
&&&&&(.0106)&&&&&&&&& \tabularnewline
Migration&&&&&&.015&&&&&&&& \tabularnewline
&&&&&&(.0287)&&&&&&&& \tabularnewline
Trade&&&&&&&&&&&&&&.0133 \tabularnewline
&&&&&&&&&&&&&&(.0094) \tabularnewline
\textbf{\emph{W.r.t. headquarters:}} &&&&&&&&&&&&&& \tabularnewline
Linguistic dist.&&&&&&&&&&&&&.0144& \tabularnewline
&&&&&&&&&&&&&(.0101)& \tabularnewline
Trade&&&&&&&-.0233**&&&&&&&-.0214** \tabularnewline
&&&&&&&(.0105)&&&&&&&(.0105) \tabularnewline
Foreign $\times$ Low Cap. Controls&&&&&&&&&&-.0776***&-.0811**&&& \tabularnewline
&&&&&&&&&&(.0264)&(.0325)&&& \tabularnewline
Foreign $\times$ Institutions&&&&&&&&&&&-.0116&&& \tabularnewline
&&&&&&&&&&&(.0082)&&& \tabularnewline
N&389,295&388,415&349,093&373,980&378,911&290,630&373,066&&235,608&214,216&205,122&&348,250&382,051 \tabularnewline
$ R^2 $&.698&.6977&.7056&.6995&.7004&.7065&.6995&&.7196&.7177&.7178&&.6993&.6976 \tabularnewline
Cty $ \times $ Date $ \times $ Var. $ \times $ Hor. FE&\checkmark&\checkmark&\checkmark&\checkmark&\checkmark&\checkmark&\checkmark&&\checkmark&\checkmark&\checkmark&&\checkmark&\checkmark \tabularnewline
For. $ \times $ Date $ \times $ Var. $ \times $ Hor. FE &\checkmark&\checkmark&\checkmark&\checkmark&\checkmark&\checkmark&\checkmark&&\checkmark&\checkmark&\checkmark&&\checkmark&\checkmark \tabularnewline
\bottomrule \addlinespace[\belowrulesep]

\end{tabularx}
\begin{flushleft}
\footnotesize \begin{minipage}{1\linewidth} \vspace{-10pt} \begin{tabnote} \textit{Notes:}   The table shows the regression of the log absolute forecast error on regressors accounting for the geography of information. All standard errors are clustered at the country, forecaster and date levels. \end{tabnote} \end{minipage}  
\end{flushleft}
}
\end{table}

%	}
%\end{landscape}

\begin{sidewaystable}
	\centering
	{\setstretch{1}
		\begin{table}[H] \centering
\newcolumntype{C}{>{\centering\arraybackslash}X}

\caption{The Geography of Information}
\label{tab:error_reg_labs_gravity}
{\footnotesize
\begin{tabularx}{\linewidth}{l C C C C C C C m{0.005\textwidth} C C C m{0.005\textwidth} C C}

\toprule
&\multicolumn{14}{c}{$\ln(|Error_{ijt,t}^m|)$}\tabularnewline\cline{2-15} \tabularnewline  \cline{2-8} \cline{10-12} \cline{14-15}\tabularnewline  &{(1)}&{(2)}&{(3)}&{(4)}&{(5)}&{(6)}&{(7)}&&{(8)}&{(9)}&{(10)}&&{(11)}&{(12)} \tabularnewline
{Coefficient}&{}&{}&{}&{}&{}&{}&{}&{}&{Finance}&{Finance}&{Finance}&{}&{}&{} \tabularnewline
\midrule \addlinespace[0pt]
\midrule Foreign&.0577***&.054***&.0356&.0262&.0727**&.059**&.0189&&.0389*&.0348*&.026&&.024&.079*** \tabularnewline
&(.0158)&(.0148)&(.022)&(.0249)&(.0295)&(.029)&(.024)&&(.0208)&(.0206)&(.0186)&&(.0287)&(.0256) \tabularnewline
\textbf{\emph{W.r.t. closest subs.:}} &&&&&&&&&&&&&& \tabularnewline
Physical dist.&&.0053&&&&&&&&&&&& \tabularnewline
&&(.0078)&&&&&&&&&&&& \tabularnewline
Cultural dist.&&&.0115&&&&&&&&&&& \tabularnewline
&&&(.0085)&&&&&&&&&&& \tabularnewline
Linguistic dist.&&&&.0189*&&&.0193*&&&&&&.022*& \tabularnewline
&&&&(.0108)&&&(.0103)&&&&&&(.0125)& \tabularnewline
BC comovement&&&&&.007&&&&&&&&& \tabularnewline
&&&&&(.0106)&&&&&&&&& \tabularnewline
Migration&&&&&&.015&&&&&&&& \tabularnewline
&&&&&&(.0287)&&&&&&&& \tabularnewline
Trade&&&&&&&&&&&&&&.0133 \tabularnewline
&&&&&&&&&&&&&&(.0094) \tabularnewline
\textbf{\emph{W.r.t. headquarters:}} &&&&&&&&&&&&&& \tabularnewline
Linguistic dist.&&&&&&&&&&&&&.0144& \tabularnewline
&&&&&&&&&&&&&(.0101)& \tabularnewline
Trade&&&&&&&-.0233**&&&&&&&-.0214** \tabularnewline
&&&&&&&(.0105)&&&&&&&(.0105) \tabularnewline
Foreign $\times$ Low Cap. Controls&&&&&&&&&&-.0776***&-.0811**&&& \tabularnewline
&&&&&&&&&&(.0264)&(.0325)&&& \tabularnewline
Foreign $\times$ Institutions&&&&&&&&&&&-.0116&&& \tabularnewline
&&&&&&&&&&&(.0082)&&& \tabularnewline
N&389,295&388,415&349,093&373,980&378,911&290,630&373,066&&235,608&214,216&205,122&&348,250&382,051 \tabularnewline
$ R^2 $&.698&.6977&.7056&.6995&.7004&.7065&.6995&&.7196&.7177&.7178&&.6993&.6976 \tabularnewline
Cty $ \times $ Date $ \times $ Var. $ \times $ Hor. FE&\checkmark&\checkmark&\checkmark&\checkmark&\checkmark&\checkmark&\checkmark&&\checkmark&\checkmark&\checkmark&&\checkmark&\checkmark \tabularnewline
For. $ \times $ Date $ \times $ Var. $ \times $ Hor. FE &\checkmark&\checkmark&\checkmark&\checkmark&\checkmark&\checkmark&\checkmark&&\checkmark&\checkmark&\checkmark&&\checkmark&\checkmark \tabularnewline
\bottomrule \addlinespace[\belowrulesep]

\end{tabularx}
\begin{flushleft}
\footnotesize \begin{minipage}{1\linewidth} \vspace{-10pt} \begin{tabnote} \textit{Notes:}   The table shows the regression of the log absolute forecast error on regressors accounting for the geography of information. All standard errors are clustered at the country, forecaster and date levels. \end{tabnote} \end{minipage}  
\end{flushleft}
}
\end{table}

	}
\end{sidewaystable}

Column (1) of Table \ref{tab:error_reg_labs_gravity} reports the baseline estimates for Foreign. Column (2) to (4) add the distance variables. \textit{Physical} and \textit{Cultural Distance} have the expected signs but are statistically insignificant, whereas \textit{Linguistic Distance} is significant and fully absorbs the impact of Foreign. This finding suggests that language is a major barrier to information, and one of the major causes of the foreign penalty. \textit{Business cycle comovement} and \textit{Migration} do not seem to play a role (Columns (5) and (6)). Overall, we identify \textit{Linguistic Distance} as the main driver of the foreign penalty.


Note that, as widely shown by the above-mentioned literature, barriers to information are also correlated with trade and financial ties, which in turn may increase the incentives to acquire information. In line with rational inattention models \citep{Sims2003}, agents can choose how much effort and resources to allocate to acquiring and processing information. Consequently, the observed effect of geography on the foreign penalty could actually be the indirect result of these choices. In that case, information asymmetries are not an exogenous information processing constraint, but rather the result of an endogenous choice to devote less resources to information acquisition.



% In this view, information asymmetries arise from deliberate choices, not fixed limitations.


 %As in rational inattention models \citep{Sims2003,Mackowiak2009}, agents can choose to devote more resources to the acquisition and processing of information. Therefore, the impact of geography on the foreign penalty could actually be the indirect result of these choices. In that case, information asymmetries are not an exogenous information processing constraint, but rather the result of an endogenous choice to devote less resources to information acquisition.

To test this incentives' channel, we introduce trade linkages in the regression. \textit{Trade linkages} are measured by the exports from the country where the forecaster's headquarters is located to country $j$, normalized by the GDP of the headquarters' country. The idea is that, controlling for other drivers of the errors, a forecaster will devote more resources to forecasting a country if that country is an important market for the producers of the country where the forecaster's headquarters are located. In that case, the coefficient should be negative. Here, we assume that the headquarters concentrate the largest part of the firm's operations and investor base. Column (7) shows that the coefficient is negative, which suggests that the incentives' channel is at play.

Columns (8) to (10) focus on financial forecasters (banks, mutual funds, hedge funds, etc.), who are more likely to have investments at stake in foreign countries and therefore should have stronger incentives to acquire information about them. Column (8) shows that the foreign penalty is lower on average among financial forecasters (0.04 against 0.06 for the whole sample), although this difference is not statistically significant. To further test for the existence of an incentive channel among these financial forecasters, we follow \citet{Baeetal2008} and add the interaction between \textit{Foreign} and a measure of financial openness. As long as financial openness is unrelated to information asymmetries, a negative coefficient on the interaction can be interpreted as reflecting incentives: financial investors will devote more resources to information acquisition in more open foreign markets, resulting in a lower foreign penalty. In Column (9), Foreign is interacted with a dummy that is equal to one if country $j$ has a low level of capital controls: we use the measure of \emph{de jure} capital controls from \citet{Fernandez2016} and define \textit{Low-capital control} countries as the quartile with the lowest level of capital controls. This coefficient is significantly negative, suggesting that capital market openness does increase the incentives to acquire information for financial forecasters. Since financial openness is likely to be correlated with the quality of institutions, which could be related to transparency, we check in Column (10) that our results are robust to the inclusion of the interaction between \textit{Foreign} and a measure of the \textit{Quality of institution}s. We conclude that incentives influence the performance of forecasters and determine the foreign penalty.

This challenges the interpretation of the results in Columns (2) to (6) as being driven directly by barriers to information. An alternative and equally plausible explanation is that \textit{Linguistic Distance} leads to weaker economic linkages between countries, which in turn result in poorer forecast performance. %This casts doubts on the interpretation of the results of Column (2) to (6) as arising directly from barriers to information. Indeed, another valid interpretation would be that linguistic distance indirectly generates larger forecasting errors, because it reduces economic linkages between countries. %A first argument against this interpretation is that physical distance does not appear to be a significant driver of the forecasters' errors. If this indirect channel were at play, physical distance would have been significant. Indeed, because physical distance is a good proxy for transportation costs, it is a robust driver of bilateral trade, and hence of financial linkages. Instead, linguistic distance, which is more directly related to information, plays an important role. This is evidence that information asymmetries are not the mere result of incentives.
An argument against this interpretation is that the geography of incentives differs from the geography of barriers to information. Indeed, while we have assumed so far that trade linkages are relevant at the headquarters level, we can test whether this assumption is valid by performing a horse race between two measures of trade linkages: a measure based on trade with the headquarters' country and a measure based on trade with the closest subsidiary's country. Column (12) confirms that bilateral \textit{Trade} is significant only if it is computed as bilateral trade between the country and the forecaster's headquarters. Now, if barriers to information were only relevant because they indirectly generated incentives to acquire information, then we would expect that only information barriers between the country and the headquarters' country are relevant. However, we find that information barriers are relevant at the level of the closest subsidiary, not at the headquarters' level: \textit{Linguistic Distance} is significant only if computed between the country and the forecaster's closest subsidiary, as is apparent in Column (11). We thus interpret the evidence as being consistent with the coexistence of both the information barriers channel and the incentives channel.


%we follow their approach and interact the Foreign dummy with some measures of \emph{de jure} economic protectionism. If this channel is at play, then investors and exporters will devote less resources to acquire information on a protectionist foreign market, and the foreign penalty will be larger for protectionist countries. As long as protectionism is unrelated to information asymmetries, we can interpret a positive interaction term as arising from incentives. In Column (5), we include the interaction between Foreign and the average tariff applied to the most favored nation from the World Trade Organization database (https://stats.wto.org/). This coefficient is positive, but insignificant. Column (6) shows that the Foreign penalty is lower on average among financial forecasters, which are more likely to have investments at stake in foreign countries (0.04 against 0.06 for the whole sample), but this difference is not statistically significant. In Column (7), we add the interaction between Foreign and a measure of \emph{de jure} capital controls from \citet{Fernandez2016}. This coefficient is significantly positive, suggesting that capital controls lower the incentives to acquire information for financial forecasters. However, note that economic protectionism can be correlated with the degree of transparency, which can affect information asymmetries. To control for this, we add the interaction between Foreign and a measure of institutional quality in Column (8). The coefficient now becomes insignificant. The foreign penalty is represented in Figure \ref{fig:interaction} over the range of tariffs and capital controls that is relevant in our sample. We do see a positive slope, but this slope is not steep enough to statistically tell apart the value of the foreign penalty across the distribution. We thus do find \emph{some} evidence that incentives lower the foreign penalty, but this evidence is weak.%\footnote{Note that the fact that the foreign dummy does not mean that the Foreign penalty vanishes, because the average impact of Foreign depends on the combnation of Foreign and the interaction term. For instance, the sum of Foreign and the interaction term, evaluated at the median value of tariffs, which is 5\%, is significant.}



\section{Heterogeneity}\label{sec:heterogeneity}

In our main specification, foreign forecasters make, on average, forecasting errors that are 6\% larger. We have already shown that the foreign penalty is lower—though not significantly so—for financial and multinational forecasters. We now further examine how the foreign penalty varies with the forecast variable, the forecast horizon, country characteristics, and whether it is state-dependent. To do so, we estimate a regression similar to equation \eqref{eq:geography}:
%In our main specification, foreign forecasters make a 6\% higher error on average. We have already shown that the foreign penalty is lower for financial and multinational forecasters, although not significantly so. We now explore further the extent to which the Foreign penalty varies across the forecast variable, the horizon, country characteristics, and whether it is state-dependent. We do so by estimating a regression that is similar to \eqref{eq:geography}:

\begin{align}
	\ln(|Error_{ijt,t+h}^{m,x}|)= \delta_{it,h}^{m,x} +\tilde\delta_{jt,h}^{m,x} +  \beta \text{Foreign}_{ij} +\gamma \text{Foreign}_{ij}\times X_{ijt,h}^{m,x}+ \varepsilon_{ij,t}^m  \,, \label{eq:heterogeneity}
\end{align}
Here, $X_{ijt,h}^{m,x}$ is a set of variables that varies either across the forecast variable (growth and inflation), horizon (current or future), time, country, or time and country, so that the levels are already absorbed by the fixed effects. If the $\gamma$ coefficients are significant, it means that the foreign penalty is heterogeneous across that dimension. It will be useful though to also examine how the linear term $X_{ijt,h}^{m,x}$ affects the errors to answer the question: is the foreign penalty larger in situations where the errors are also larger? To identify the role of $X_{ijt,h}^{m,x}$, we will need to remove some fixed effects. In these specifications, we keep our \textit{Foreign} dummy as a control, but we do not interpret its estimated effect, since we will be neglecting some of the potential confounding factors and noise that we are controlling for in our main specification with richer fixed effects.

We begin by examining the role of the forecast variable and the forecast horizon. Specifically, we define \textit{GDP} as a dummy variable equal to 1 if the forecast targets GDP growth, and 0 if it targets inflation. \textit{Future} is a dummy equal to 1 if the forecast refers to the next calendar year, and 0 if it refers to the current year. Finally, \textit{Month-of-year} is a categorical variable ranging from 1 to 12, indicating the month in which the forecast is made.


The results are presented in Table \ref{tab:error_reg_labs_cs}. Column (1) examines how forecast errors vary with \textit{GDP}, \textit{Future}, and \textit{Month-of-year}. This specification includes only country-year and forecaster-year fixed effects to preserve sufficient variation in the explanatory variables. The estimates show that forecast errors are larger for GDP growth forecasts and for forecasts targeting the future year. Notably, forecast errors decline over the course of the year, which suggests that information flows continuously during the year. Column (5) explores how the foreign penalty interacts with these variables. It includes forecaster-date-variable-horizon and country-date-variable-horizon fixed effects, which absorb the main effects of \textit{GDP}, \textit{Future}, and \textit{Month-of-year}. The results indicate that the foreign penalty is significantly lower for GDP growth and for forecasts of the future year. Interestingly, the penalty increases over time within the year. As shown in Panel (a) of Figure \ref{fig:heterogeneity} in the Appendix, the average foreign penalty rises across calendar months. This pattern suggests that, somewhat paradoxically, the foreign penalty is larger when overall forecast uncertainty is lower.



%The results are reported in Table \ref{tab:error_reg_labs_cs}. Column (1) shows how the forecast error depends on $GDP$, $Future$ and $Month-of-year$. This regression includes only country-year and forecaster-year fixed effects to avoid absorbing all the variation in the explanatory variables. It shows that forecast errors are higher for GDP growth and for the future year. Noticeably, the forecast errors diminish over time within a given year, which suggests that information flows continuously during the year. 


%Column (5) shows the impact of the interaction of Foreign with these variables. It includes forecaster-date-variable-horizon and country-date-variable-horizon fixed effects, so the linear impact of $GDP$, $Future$ and $Month-of-year$ are fully absorbed with this fixed-effect specification. It shows that the foreign penalty is significantly lower for GDP growth and for the future year. Interestingly, the penalty increases over time within a given year.\footnote{Panel (a) of Figure \ref{fig:heterogeneity} in the Appendix shows the average foreign penalty per month, which appear to be increasing over the course of the year.}  This evidence shows that, somehow paradoxically, the foreign penalty is higher when there is less forecasting uncertainty.

As shown in Column (6), the foreign penalty does not exhibit state dependence: it does not increase significantly during recessions or periods of global uncertainty, measured by the \textit{VIX}, while Column (2) shows that forecast errors are on average larger during periods of heightened uncertainty and in recessions. This suggests that while both local and foreign forecasters make larger errors in adverse times, the relative difference between them remains stable.

We also consider several country-specific characteristics: an \textit{Emerging} economy dummy, institutional quality (from the World Development Indicators, in the table referred to as \textit{Institutions}), and country size (\textit{log of GDP }evaluated at purchasing power parity).\footnote{The data sources are the following: country size \citep{cpigravity22} and quality of institutions \citep{wdi22}. The list of emerging economies is given in the Appendix Table \ref{tab:app_emerging_advanced}.}



In Column (7), the foreign penalty does not differ between emerging and advanced economies, nor does it depend on institutional quality, while Column (3) shows that, on average, errors are larger for emerging economies, and that stronger institutions are associated with a lower error. However, the foreign penalty is higher for larger countries, for which errors are lower on average (Column (4)). As with the forecast variable and horizon, lower average uncertainty is linked to a higher foreign penalty. The main picture is unchanged when putting together all the interaction terms (Column (8)).\footnote{In the Appendix Table \ref{tab:error_reg_labs_cs_ind}, we show that the results do not change either when we interact the Foreign dummy with one variable at a time. Additionally, Figure \ref{fig:heterogeneity} in the Appendix displays the Foreign coefficients per year, month and country. Panel (a) shows the average foreign penalty per month, which appear to be increasing over the course of the year. Panel (b) shows the average foreign penalty per year. No systematic pattern appears, which is consistent with our results. Panel (c) shows the average country penalty per country. The estimates are heterogeneous across countries but no systematic difference between Emerging and Advanced economies appears, confirming our results. We also conduct a similar analysis, using the estimated coefficients from our asymmetric information tests, $\beta^{FE}$ and $\beta^{DIS}$. The results, which are shown in Table \ref{tab:FE_reg_mg}, are broadly consistent with the evidence on the errors, except that they are less precisely estimated.}

%\footnote{Information frictions are more prevalent in the later months of the year ($\beta^{FE}$ is less negative), and the foreign penalty is stronger for these months (the interaction between the month variable and the Foreign dummy has a negative effect on $\beta^{FE}$ and $\beta^{DIS}$ depends negatively on the month variable). Similarly, forecasters forecast larger countries better on average ($\beta^{FE}$ is less negative), but foreign forecasters typically have a larger disadvantage when forecasting larger countries ($\beta^{DIS}$ is significantly more negative for larger countries).}

{\setstretch{1}
	\begin{table}[H] \centering
\newcolumntype{C}{>{\centering\arraybackslash}X}

\caption{Variable, Horizon, Time and Country Dependence}
\label{tab:error_reg_labs_cs}
{\scriptsize
\begin{tabularx}{\linewidth}{l C C C C C C C C}

\toprule
&\multicolumn{8}{c}{$\ln(|Error_{ijt,t}^m|)$}\tabularnewline\cline{2-9} &{(1)}&{(2)}&{(3)}&{(4)}&{(5)}&{(6)}&{(7)}&{(8)} \tabularnewline
{Coefficient}&{}&{}&{}&{}&{}&{}&{}&{} \tabularnewline
\midrule \addlinespace[0pt]
\midrule Foreign&0.05***&0.09***&0.18***&0.03&0.06***&0.05*&--0.39**&--0.43*** \tabularnewline
&(0.02)&(0.02)&(0.06)&(0.05)&(0.02)&(0.02)&(0.17)&(0.16) \tabularnewline
GDP&0.30***&&&&&&& \tabularnewline
&(0.07)&&&&&&& \tabularnewline
Future&0.94***&&&&&&& \tabularnewline
&(0.05)&&&&&&& \tabularnewline
Month-of-year&--0.09***&&&&&&& \tabularnewline
&(0.00)&&&&&&& \tabularnewline
VIX&&0.02***&&&&&& \tabularnewline
&&(0.00)&&&&&& \tabularnewline
Recession&&0.26***&&&&&& \tabularnewline
&&(0.07)&&&&&& \tabularnewline
Emerging&&&0.27***&0.08&&&& \tabularnewline
&&&(0.09)&(0.10)&&&& \tabularnewline
Institutions&&&--0.06***&--0.11***&&&& \tabularnewline
&&&(0.02)&(0.02)&&&& \tabularnewline
ln(GDP)&&&&--0.13***&&&& \tabularnewline
&&&&(0.03)&&&& \tabularnewline
Foreign $\times$ GDP&&&&&--0.04**&&&--0.04* \tabularnewline
&&&&&(0.02)&&&(0.02) \tabularnewline
Foreign $\times$ Future&&&&&--0.03**&&&--0.03** \tabularnewline
&&&&&(0.01)&&&(0.02) \tabularnewline
Foreign $\times$ Month-of-year&&&&&0.01**&&&0.00** \tabularnewline
&&&&&(0.00)&&&(0.00) \tabularnewline
Foreign $\times$ VIX&&&&&&0.00&&0.00 \tabularnewline
&&&&&&(0.00)&&(0.00) \tabularnewline
Foreign $\times$ Recession&&&&&&0.02&&0.04 \tabularnewline
&&&&&&(0.02)&&(0.03) \tabularnewline
Foreign $\times$ Emerging&&&&&&&0.03&0.04 \tabularnewline
&&&&&&&(0.03)&(0.03) \tabularnewline
Foreign $\times$ Institutions&&&&&&&0.01&0.01 \tabularnewline
&&&&&&&(0.01)&(0.01) \tabularnewline
Foreign $\times$ ln(GDP)&&&&&&&0.02**&0.02*** \tabularnewline
&&&&&&&(0.01)&(0.01) \tabularnewline
N&602,113&601,907&375,846&366,653&389,295&389,218&366,401&366,401 \tabularnewline
$ R^2 $&0.36&0.25&0.37&0.37&0.70&0.70&0.70&0.70 \tabularnewline
Cty. $ \times $ Year FE&\checkmark&&&&&&& \tabularnewline
For. $ \times $ Year FE&\checkmark&&&&&&& \tabularnewline
Cty $ \times $ Var. $ \times $ Hor. FE&&\checkmark&&&&&& \tabularnewline
For. $ \times $ Var. $ \times $ Hor. FE&&\checkmark&&&&&& \tabularnewline
For. $ \times $ Date $ \times $ Var. $ \times $ Hor. FE&&&\checkmark&\checkmark&\checkmark&\checkmark&\checkmark&\checkmark \tabularnewline
Cty $ \times $ Date $ \times $ Var. $ \times $ Hor. FE &&&&&\checkmark&\checkmark&\checkmark&\checkmark \tabularnewline
\bottomrule \addlinespace[\belowrulesep]

\end{tabularx}
\begin{flushleft}
\footnotesize \begin{minipage}{1\textwidth} \vspace{-10pt} \begin{tabnote} \textit{Notes:}   The table shows the regression of the log absolute forecast error of current and future CPI and GDP on regressors with different fixed-effects specifications. All standard errors are clustered at the country, forecaster and date levels. \end{tabnote} \end{minipage}  
\end{flushleft}
}
\end{table}

}

\paragraph{Discussion}


The asymmetry of information between local and foreign forecasters is unaffected by the development status of the economy or the quality of institutions. This aligns with existing evidence: \citet{Baeetal2008}, who study whether local analysts are better at forecasting local firms' earnings, find that investor protection and the country's development status do not influence the foreign penalty.


%The asymmetry of information between local and foreign forecasters is not affected by the development status of the economy that is being forecasted, or by the quality of institutions. This is not surprising with regards to existing evidence. Indeed, \citet{Baeetal2008}, who examine whether local analysts are better at forecasting local firms' earnings, find that the protection of investors' rights does not influence the foreign penalty, nor does the development status of the country where the firms are located.%\footnote{In their paper, \citet{Baeetal2008} show that variables that improve the functioning of the local stock market lower the local advantage (for instance, business disclosure). However, we show that these variables are not relevant when it comes to forecast aggregate outcomes.}

We do find that variables such as country size, the forecast variable, and the forecast horizon influence the foreign penalty. However, the penalty is generally higher when forecasting uncertainty is lower, suggesting that local forecasters are better at finding and exploiting available macroeconomic information. These results are consistent with the locals' better access to locally-produced information (by knowing when and where relevant information is released). The stronger information asymmetry for current-year GDP growth or inflation forecasts, and its increase over the year (greater in December than January), is consistent with the assumption that local forecasters are exposed to the regular releases of partial GDP growth and inflation figures and integrate this information faster. Interestingly, inflation is typically available at a higher frequency and with a shorter lag than GDP, making the access to that information an even greater advantage. This is consistent with Table \ref{tab:updating_errors_main_small}, where we can see that the difference in updating frequency is 2\% larger for inflation forecasts than for GDP growth forecasts.



\section{Conclusion}\label{sec:conclusion}


We provide direct evidence of asymmetric information between domestic and foreign forecasters. Using professional forecaster expectation data, in which we determine the location of each forecaster-country pair, we show that foreign forecasters update their information less frequently compared to local forecasters and produce less precise forecasts, even conditional on updating their forecasts. We rule out over-confidence and over-extrapolation, and behavioral biases in general, as drivers of the foreigners' excess mistakes, and identify differences in information asymmetries between foreign and local forecasters. Our results have implications for the modeling and calibration of international trade and finance models with heterogeneous information, since (1) we provide estimates of the excess errors of foreign forecasters and their relative updating frequency and (2) we prove that the source of asymmetry between local and foreign forecasters is informational. Finally, we show that both exogenous barriers to information and incentives to acquire information drive the foreign penalty. In general, the foreign penalty is not stronger when forecasting is more uncertain.



\bibliographystyle{econ}
\bibliography{references}

\end{document}
