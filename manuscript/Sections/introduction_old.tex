
The informational advantage of locals over foreigners regarding macroeconomic fundamentals has far-reaching consequences. Information asymmetries are a primary explanation for the tendency of investors to prefer domestic assets in their investment portfolios, known as the home bias in asset holdings.\footnote{The home bias in asset holdings was originally documented by  \citet{FrenchPoterba1991}. \citet{Gehrig1993} and \citet{BrennanCao1997} use the workhorse model of rational expectations equilibrium developed by \citet{Grossman1976} and \citet{Admati1985}. See also \citet{Lewis1999}, \citet{TesarWerner1995}, \citet{Ahearneetal2004}, \citet{PortesRey2005}, \citet{CoeurdacierRey2013} and \citet{Hu2020}. Work on asymmetric information and the home bias includes \citet{Admati1985}, \citet{Pastor2000}, \citet{Portesetal2001}, \citet{VanNieuwerburghVeldkamp2009}, \citet{Mondria2010}, \citet{DeMarcoetal2021}.} Information asymmetries are also a potential source of capital flow volatility, since disagreement between foreign and domestic investors can generate cross-border asset trade.\footnote{See \citet{Yuan2005}, \citet{Albuquerqueetal2007}, \citet{Albuquerqueetal2009}, \citet{BrennanCao1997}, \citet{Broneretal2013}, \citet{TillevanWincoop2010}, \citet{TillevanWincoop2014}, \citet{BenhimaCordonier2022}.} Beyond their impact on international asset markets, they also constitute a barrier to the international trade in goods, as highlighted by \citet{AndersonvanWincoop2004}.\footnote{See also \citet{HeadMayer2013}, \citet{Allen2014}, \citet{DasguptaMondria2018}, \citet{Eatonetal2021}. \citet{Baleyetal2020} show that cross-border uncertainty may sometimes increase trade.} Finally, recent papers highlight their role in international business cycle comovement.\footnote{See \citet{Iliopulosetal2021} and \citet{Buietal2021}.} However, there remains a lack of direct evidence regarding the existence of information asymmetry regarding macroeconomic fundamentals and quantitative estimates of the extent of this asymmetry.

We fill this gap using a unique dataset of inflation and GDP growth forecasts for the current and the next year by local and foreign forecasters. Unlike previous studies, the forecaster and country dimensions of the panel allows us to control for a rich set of fixed effects. We first show that foreign forecasters update their forecasts about 10\% less frequently than local forecasters. Since forecasters in our dataset update their forecasts on average 9 times a year, this amounts to about one less update per year. They also make more mistakes than local forecasters, and foreign forecasters' absolute error is on average 6-9\% higher than local forecasters, which corresponds to 0.035-0.04 percentage points. The local advantage is especially large when predicting inflation as opposed to GDP and it is stronger for shorter forecasting horizons.

We then investigate the role of information frictions and behavioral biases in explaining our results about errors. We do this in two steps. First, we rule out behavioral biases such as over-reaction to new information as explanations of the foreigners' excess mistakes, by showing that the local and foreign behavioral biases do not differ systematically. Second, we test for the relative precision of local and foreign forecasters' private information, and find that local forecasters have more precise private information. To do so, we build on and extend the fast-growing literature that uses model-based tests to identify frictions in the expectation formation of survey respondents \citep{CoibionGorodnichenko2015,Bordaloetal2020,BroerKohlhas2019,Angeletosetal2020,Goldstein2021}. In particular, we provide tests of asymmetric information that are robust to the presence of public signals. These tests show that foreign forecasters have less precise information.

Finally, we explore some determinants of the information asymmetry between local and foreign forecasters. First, we show that the location of subsidiaries plays an important role for the information produced by multinationals. Having operations in a country through local subsidiaries is associated with the same informational advantage as having headquarters based in the country. Second, geography matters. Linguistic distance in particular appear to be a major barrier to information, as it is the only variable that we find that completely absorbs the Foreign coefficient. Foreign forecasters also forecast better when the economic ties (trade and financial) between the country where their headquarter is located and the country they are forecasting are stronger. This evidence is consistent both with the existence of exogenous barriers to information and with an endogenous emergence of a foreign penalty due to smaller incentives to acquire information. While the limitations of our our data prevents us from fully disentangling these two channels, we find some arguments that they are both are at play.

We also examine whether information asymmetries are related to factors that drive forecasting uncertainty. Interestingly, the asymmetry is not reduced when forecasting is less uncertain. If anything, it is increased. Indeed, the local advantage is higher for short horizons and for inflation (as opposed to GDP growth), but also for large countries. In all these situations, the forecasting uncertainty (measured by the average forecast error) happens to be \emph{smaller}. However, we find no evidence that the difference in forecast errors between local and foreign forecasters is linked to the development status of the country, to institutional quality, or to macroeconomic volatility, despite the fact that these variables do affect the average forecasting uncertainty.\footnote{These findings echo the weak link between uncertainty and disagreement that has been documented in the literature \citep{LahiriSheng2010,RichTracy2010}.}% These results should help further discipline the link between uncertainty and information asymmetry in models of international finance and trade.

This evidence suggests that when information becomes available, it flows to local forecasters, and sometimes, but not always, to foreign forecasters. These results would be consistent with better access to locally-produced information (by knowing when and where relevant information is released). We show that the information asymmetry is stronger for nowcasting, and that it increases in the course of a year (the asymmetry is higher in December than in January). This is consistent with the idea that local forecasters are exposed to the regular releases of partial GDP growth and inflation figures and integrate this information faster. Interestingly, inflation figures are typically available at a higher frequency and with a shorter lag than GDP, making the access to that information an even greater advantage. This is consistent with our finding that the difference in updating frequency is larger for inflation forecasts than for GDP growth forecasts.

Our estimates of a 6-9\% foreign penalty can be interpreted as a 12-19\% difference in the conditional variances. Are these estimates economically significant? The picture is mixed. These estimates are not large enough to provide an explanation of the home bias. For instance, \citet{VanNieuwerburghVeldkamp2009} show that a 10\% difference in the variance of priors would not enough to generate a home bias of the magnitude that is observed in the data. \citet{Jeske2001} finds that the foreign penalty necessary to explain the home bias varies between 25\% for the US to 80\% for Italy.\footnote{See also \citet{Glassman2001}.} However, other phenomenon driven by disagreement between local and foreign agents can be explained by information asymmetries of this magnitude. For instance, \citet{TillevanWincoop2014}, the degree of information asymmetry that generates a plausible level gross capital flow volatility implies very small differences in conditional variance (less than 1\%).\footnote{Despite a 50\% difference in the volatility of individual noise, the difference in the volatility of forecast errors is very small, because the variance of the fundamental is one order of magnitude smaller.}

However, there are two arguments in favor of the economic relevance of these estimates. First of all, these estimates have to be interpreted as an upper bound on the level of information asymmetries that are relevant for decision-making. The Consensus Economics Survey that we use is based on a panel of professional forecasters that are selected into the survey because forecasting is part of their business. Compared to other firms, this is a subset of high-performing information producers. Besides, the variables that are the object of the forecast are macroeconomic variables that are easier to forecast because (i) they are less volatile and (ii) plenty of public information is available. Economic decisions often needs information on individual firms or markets, for which public information is not as plentiful. We can thus expect asymmetric information on ``microeconomic'' data to be at least as high as our estimates. In our empirical analysis, we do find a significant level of heterogeneity between forecasters. 

Second, the literature has shown that small information costs can be significantly amplified by investor behavior and market mechanisms. \citet{VanNieuwerburghVeldkamp2009} also show that the same 10\% difference in the variance of priors can generate empirically plausible levels of home bias when investors can choose what information to learn before they invest. Our estimates are therefore more in line with models that rely on amplification effects. \citet{Hatchondo2008} show that, when investors face short-selling constraints, a small information asymmetry can generate a sizable home bias. When returns are correlated, small diversification costs can be enough to generate a home bias. In particular, according to \citet{Wallmeier2022}, a 5\% home ``variance advantage'' can alone explain half of the observed home bias when the return correlation is 0.9, which is the value for 9 major economies.\footnote{See also \citet{Bhamra2014} and \citet{Levy2014}.} In the international trade literature, \citet{Allen2014} shows that, in a model where firms decide on market entry and investment based on their information sets, small information costs are consistent with the empirical extensive and intensive patterns of trade, but ignoring these costs significantly deteriorates the fit of the model. In general, more quantitative work is needed to evaluate the quantitative relevance of information frictions in international finance and trade. Our estimates provide a useful conservative benchmark to do so.

%Information advantages have been used to explain exchange rate variations (Evans and Lyons (2005), Bacchetta and van Wincoop (2006), international consumption correlation puzzle (Coval (2003)), international capital flows (Brennan and Cao (1997)), a bias towards investing in local firms (Coval and Moskowitz (2001)), and the own-company stock puzzle (B and Wang (2003)). Information asymmetry is also the basis for other home bias explanations, such as ambiguity aversion (Uppal and Wang). All these explanations are bolstered by our finding that information are not only sustainable when information is mobile, but by the fact that asymmetric information can be amplified when investors can choose what to learn.

%The canonical reference on asymmetric information with multiple assets is Admati (1985). Work on asymmetric information and the home bias, in particular, includes Pastor (2000), Brennan and Cao (1997), and Portes, Rey, and Oh (2001).

%Ahearne, Griever, and Warnock (2004)


%ChatGPT thrid version: Are our estimates economically significant? Consider the impact of information asymmetries on international trade. While there are no precise quantitative estimates provided by Rauch (2001) on the percentage difference in information precision required to fully explain trade barriers, his qualitative analysis emphasizes the role of networks in reducing information frictions and facilitating trade, particularly for differentiated products. This suggests that our estimates can be informative in contexts where detailed empirical estimates are lacking. \footnote{See \citet{Rauch2001} for an analysis of the role of networks and information asymmetries in trade. Although Rauch does not provide specific quantitative estimates, his work highlights the significant impact of these frictions on trade. See also \citet{AndersonvanWincoop2004}, \citet{NunnTrefler2013}, and \citet{HeadMayer2013} for discussions on the broader implications of information asymmetries in trade.} Our estimates are more in line with models that rely on amplification effects. For instance, \citet{Allen2014} shows that small differences in information precision can lead to significant trade imbalances when firms decide on market entry and investment based on their information sets. \footnote{See also \citet{Chaney2008}, who discusses how minor information differences can influence trade patterns and volumes through amplification effects.} Specifically, \citet{Chaney2008} finds that information frictions can significantly affect trade flows by influencing firms' decisions to enter foreign markets. Similarly, \citet{NunnTrefler2013} discuss how relational contracts and trust, which are closely tied to information asymmetries, impact international trade. Therefore, while our estimates of information asymmetries alone may not fully explain the observed trade barriers, they are consistent with theoretical models where small initial differences in information can be amplified through market dynamics and firm behaviors.

This paper contributes to the recent literature that uses professional forecasters' expectations to identify information frictions and behavioral biases. This literature has used reduced-form estimations as indicators of deviations from Full-Information Rational Expectations (FIRE). \citet{CoibionGorodnichenko2015} (CG henceforth) use the estimated coefficient in the regression of the consensus error on the consensus revision as an indicator of deviations from Full Information (FI). \citet{Bordaloetal2020}, (BGMS henceforth) \citet{BroerKohlhas2019} (BK henceforth) and \citet{Angeletosetal2020} (AHS henceforth), use the estimated coefficient in the individual pooled regression as an indicator of deviations from Rational Expectations (RE).\footnote{An earlier literature has previously identified deviations from rationality by studying the joint behavior of actual on predicted values, the auto-correlation of forecasts revisions and the predictability of errors. See, for example, \citet{MincerZarnowitz1969}, \citet{Zarnowitz1983}, \citet{Nordhaus1987}, \citet{Clements1997}, \citet{LahiriSheng2008}.} We borrow this test directly from the literature to assess whether domestic and foreign behavioral biases differ. 

However, CG's Full Information (FI) test, which has been commonly used in the literature, is not adapted to our purpose. In the presence of public information, the CG coefficient, which is a common measure of information frictions, is biased. Importantly, the bias depends on the precision of the public signal and is not a monotonic function of the precision of private signals. Comparing the CG coefficient across local and foreign forecasters cannot indicate which group faces more frictions.\footnote{Both CG and \citet{Goldstein2021} have emphasized that the CG coefficient is biased, but have not highlighted the implied non-monotonicity.} We thus provide two tests that are robust to the presence of public information. The first relies on individual regressions in the spirit of BGMS but with country-time fixed effects to capture aggregate shocks and the public signals. This test is similar in spirit to \citet{Goldstein2021}, who proposes to use forecasters' deviations from the mean to measure information frictions robustly. The second test infers the relative precision of private information from the relative reaction of expectations to public signals.

This paper also belongs to the empirical literature documenting the local informational advantage. Many studies provide indirect evidence of asymmetric information between domestic and foreign investors by showing that location matters for portfolio composition and for portfolio returns.\footnote{See for instance \citet{KangStulz1997}, \citet{GrinblattKeloharju2001}, \citet{Dvorak2003}, \citet{PortesRey2005}, \citet{Ahearneetal2004}, \citet{Hamaomei2001}, \citet{Hau2001}, \citet{Choeetal2005}, \citet{Baiketal2010} and \citet{Clemensetal2020}.} However, based on investor choices and returns, some papers find that foreign investors perform better than local investors (e.g. \citet{GrinblattKeloharju2000}).\footnote{This could be explained by the specialization of some investors in some specific markets where they have an initial informational advantage. This informational advantage can be due to location, but not only. 
Therefore, information heterogeneity can also lead to specialization in non-domestic assets (see \citet{VanNieuwerburghVeldkamp2010} and \citet{DeMarcoetal2021}).} In contrast to these studies, we investigate whether location affects the quality of forecasters' information, thus providing direct evidence of information asymmetries. Closest to our study is the paper by \citet{Baeetal2008}, which focuses on the performance of local and foreign analysts in forecasting earnings for firms. Our focus is different since we examine whether locals outperform foreigners in forecasting aggregate variables. Moreover, we not only document the foreign forecasters' excess errors, but we also investigate whether these excess mistakes come from information frictions or behavioral biases. Finally, other studies document foreigners' lack of attention to domestic information.\footnote{See for instance \citet{Leuzetal2009}, \citet{Mondriaetal2010} \citet{Huang2015} and \citet{Czirakietal2021}.}

The paper is structured as follows. Section \ref{sec:data} describes our dataset. Section \ref{sec:updating} focuses on the updating frequency of forecasts. Section \ref{sec:mistakes} documents the foreign forecasters' excess mistakes. Section \ref{sec:model} lays down a model of expectation formation and tests for the sources of the foreigners' excess mistakes. Section \ref{sec:drivers} investigates drivers of forecast errors and asymmetric information. Section \ref{sec:robust} provides several robustness checks. Finally, section  \ref{sec:conclusion} concludes. 