
\paragraph{Forecasts.} We use data from Consensus Economics, a survey firm polling individual economic forecasters on a monthly frequency. The survey covers 51 advanced and emerging countries and we focus on observations between 1998 and 2021.\footnote{For an overview of the data coverage of all advanced and emerging economies in our sample see Table \ref{tab:app_data_countryrange} in the online Appendix. Note that the survey provides forecasts as of 1989 for some countries. However, our sample period is limited by the GDP and inflation vintage series of realized outcomes provided by the IMF. The development status retained for the countries of our sample is available in Table \ref{tab:app_emerging_advanced} in the online Appendix.} Each month, forecasters provide estimates of several macroeconomic indicators for the current and the following year. %An advantage of this dataset is that it allows for  meaningful comparisons across both countries and forecasters.\footnote{Consensus Economics clearly defines each macroeconomic indicator surveyed. }
 In this paper, we focus on two indicators, namely CPI inflation and GDP growth. The dataset discloses the name of the individual forecasters. There are \DTLfetch{valuesData}{indicator}{N}{value} unique forecasters from which \DTLfetch{valuesData}{indicator}{Nctry}{value} conduct forecasts for at least 2 distinct countries.  For each forecaster-country pair, the average (median) number of observations is \DTLfetch{valuesData}{indicator}{meanIdct}{value} (\DTLfetch{valuesData}{indicator}{medianIdct}{value}), which corresponds to approximately \DTLfetch{valuesData}{indicator}{meanIdctYr}{value} (\DTLfetch{valuesData}{indicator}{medianIdctYr}{value}) years. This leads to an unbalanced panel dataset.


\paragraph{Realized Outcomes.} Following the literature, we use first-release data to compare forecast precision across forecasters. For each survey year, we use the realized outcome for yearly inflation and real GDP growth from the International Monetary Fund World Economic Outlook (IMF WEO) published in April of the subsequent year. This allows us to avoid forecast errors that are due to data revisions. For example, to assess the accuracy of the 2013 real GDP growth forecast for Brazil from the January 2013 survey, we use the yearly GDP growth reported in the April 2014 IMF WEO as realized outcome. To assess the accuracy of the 2014 real GDP growth forecast for Brazil from the same January 2013 survey, we use the yearly GDP growth reported in April 2015. %We conduct robustness checks with alternative vintages using IMF WEO published in September or in subsequent years.\footnote{Using alternative vintage series ensures that differences in forecasting precisions are not solely due to individual forecasters that anticipate revisions in actual GDP or inflation and therefore have a different forecasting target.}
 Archived IMF WEO vintage data are available from 1998 onwards.

As is common in the literature, we trim observations, removing forecasts that are more than 5 interquartile ranges away from the median. The quantiles are calculated first on the whole sample, but separately for emerging and advanced countries, and second, for each country and date. This trimming ensures that our results are not driven by extreme outcomes, such as periods of hyperinflation, or by typos. It reduces the number of forecasts for current inflation and GDP by \DTLfetch{valuesData}{indicator}{fcastTrimLosscpi}{value} and \DTLfetch{valuesData}{indicator}{fcastTrimLossgdp}{value} \% respectively.%\DTLfetch{valuesData2}{indicator}{fcastTrimLossgdp}{value} and \DTLfetch{valuesData2}{indicator}{fcastTrimLosscpi}{value}, respectively.

\paragraph{Forecast errors.} We use this information to construct forecast errors. The forecast errors with respect to the current year and the future year are defined as
$Error_{ijt,t}^m=x_{jt}-E_{ijt}^m(x_{jt})$ and $Error_{ijt,t-1}^m=x_{jt}-E_{ijt-1}^m(x_{jt})$, 
where $t$ refers to the year, $i$ is the forecaster, $j$ is the country, $m=1,..,12$ is the month of the year when the forecast is produced, and $x$ is either inflation or GDP growth.

\paragraph{Location of Forecasters.} Consensus Economics discloses the name of the forecasting institution. Eikon (Refinitiv) provides the company tree structure of most forecasters in our dataset. The tree structure includes information about the countries in which the headquarters, the subsidiaries and the affiliates are located. If the forecaster was not listed in the Eikon database, we manually searched for this information on the Internet. In the main analysis, we consider a forecaster to be foreign if neither its headquarter nor any of its subsidiaries are located in the country of the forecast.\footnote{Note that the location information is not time-varying and corresponds to the information accessed in 2021. This amounts to a measurement error that could bias the magnitude of the location effect downward.} %Later, we will examine whether the forecasts produced by a forecaster headquartered in the country are different from the forecasts produced by a forecaster with only a subsidiary located in the country.


\paragraph{Forecasters' Scope.} Furthermore, we identify the scope of the forecasters. We categorize forecasters with subsidiaries and headquarters all located in the same country as non-multinational forecasters. In contrast, we categorize forecasters with at least one subsidiary located in a country outside that of their headquarters as multinationals.\footnote{The scope variable is also based on information from 2021 and is therefore not time-varying.}

\paragraph{Descriptive Statistics} Some descriptive statistics, shown in the online Appendix, are particularly relevant for our analysis. According to Panel a) of Figure \ref{fig:hq_sub_obs}, almost two thirds of the forecasts come from multinational forecasters, and almost three quarters are made by ``local'' forecasters, that is, forecasters who have either their headquarters or a subsidiary located in the country. As compared to non-multinational forecasters, a higher proportion of forecasts by multinational forecasters is local, thanks to their subsidiaries. Since multinationals are also more likely to have well-endowed forecasting departments and to make smaller forecasting errors, it will be important to control for forecaster-level characteristics.

According to Figure \ref{fig:hq_sub_obs_by_cty}, the share of forecasts provided by foreign forecasters (that have neither their headquarters, nor a subsidiary located in the country) varies a lot across countries. Countries that have a lower share of foreign forecasts are large advanced economies where many national and multinational forecasters have their headquarters (United States, Japan, Germany...), or large emerging economies where many multinational forecasters have subsidiaries (China, South Korea, Brazil, Chile...). Countries with a high share of foreign forecasts tend to be small advanced or emerging economies, as they are less likely to host the headquarters or a subsidiary of a forecaster (e.g., Bulgaria, Latvia, Greece, Nigeria). Since smaller countries have more volatile business cycles that are more difficult to forecast, it will be important to control for country-level characteristics in our analysis.

Because we will be controlling for forecaster and country characteristics with fixed effects, the identification of the role of the foreign nature of forecasts will come from the forecasters that provide both local and foreign forecasts. Panel a) of Figure \ref{fig:loc_for} shows that these forecasters are a small minority of forecasters (11\%). Indeed, only 86 forecasters among 749 provide forecasts for both local and foreign countries. However, this minority of forecasters accounts for 60\% of observations in our sample, as Panel b) shows.\footnote{This is consistent with Figure \ref{fig:hist}, which shows that most forecasters provide forecasts for only one country, and only a small proportion of forecasters provide forecasts for 5 countries or more. Even though the remaining 40\% of observations will not contribute to the identification of the role of the forecaster's location, they will contribute to the identification of the role of the country and forecaster fixed effects and improve the precision and power of our estimates.} %Besides, most of these observations come from multinational firms.
%especially non-multinational firms,
%(22\% of multinational forecasters, 5\% of non-multinational forecasters). The difference in forecasting behavior or performance that we will identify will thus come from the most ``expert'' among forecasters: the multinational forecasters and those that cover many countries. The role of these characteristics will be explored in more details later.
