\subsection{Economic Significance of the Foreign Penalty}

Our estimates of a 6-9\% higher standard error can be interpreted as a 12-19\% difference in the conditional variances. Assuming that they are driven by information asymmetries (which we will discuss in the next section), are they economically significant? 

These estimates are not large enough to provide an explanation of the home bias per se. For instance, \citet{Jeske2001} finds that the foreign penalty necessary to explain the home bias varies between 25\% for the US to 80\% for Italy. However, the literature has shown that small information costs can be significantly amplified by investor behavior and market mechanisms. \citet{VanNieuwerburghVeldkamp2009} show that a difference in the variance of priors as small as 10\% can generate empirically plausible levels of home bias when investors can choose what information to learn before they invest. Our estimates are therefore more in line with models that feature amplification effects. \citet{Hatchondo2008} show that, when investors face short-selling constraints, a small information asymmetry can generate a sizable home bias. When returns are correlated, small diversification costs can be enough to generate a home bias. In particular, according to \citet{Wallmeier2022}, a 5\% home ``variance advantage'' can alone explain half of the observed home bias when the return correlation is 0.9, which is the value they document for 9 major economies.

Moreover, other phenomena driven by disagreement between local and foreign agents can be explained by information asymmetries of this magnitude. For instance, \citet{TillevanWincoop2014} show that the degree of information asymmetry that generates a plausible level of gross capital flow volatility implies a very small difference in the conditional variances.\footnote{Despite a 50\% difference in the volatility of individual noise, the difference in the volatility of forecast errors is smaller than 1\%, because the variance of the fundamental is one order of magnitude smaller.} In the international trade literature, \citet{Allen2014} shows that, in a model where firms decide on market entry and investment based on their information sets, small information costs are consistent with the empirical extensive and intensive patterns of trade, but ignoring these costs significantly deteriorates the fit of the model.

In general, more quantitative work is needed to evaluate the quantitative relevance of macroeconomic information frictions in international finance and trade. Our estimates provide a useful conservative benchmark to do so. Indeed, our estimates can be interpreted as a lower bound on the level of information asymmetries that are relevant for decision-making. The Consensus Economics Survey that we use is based on a panel of professional forecasters that are selected because forecasting is part of their business. They are by construction better forecasters than other firms. %Besides, the variables that are the object of the forecast are macroeconomic variables that are easier to forecast because they are less volatile and plenty of public information is available. Economic decisions often needs information on individual firms or markets, for which public information is not as plentiful. We can thus expect asymmetric information on ``microeconomic'' data to be at least as high as our estimates.
