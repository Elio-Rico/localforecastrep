
Consensus regressions that consist in regressing the consensus error (i.e., the average error) on the consensus revision (i.e., the average revision) as in \citet{CoibionGorodnichenko2015} are commonly used to detect information frictions. A positive coefficient indicates deviations from full information. Can we use these regressions to identify differences in information frictions between local and foreign forecasters?  We show here that differences in the coefficient of the consensus regression are not a good indicator of the degree of information asymmetry. We thus propose two alternative tests that are robust to public signals.

\subsubsection{Consensus regressions}

%Consensus regression as in \citet{CoibionGorodnichenko2015} are commonly used to detect information frictions. Can we use these regressions to identify differences in information frictions between local and foreign forecasters?

Suppose that we perform the consensus regression as in \citet{CoibionGorodnichenko2015} on both group of forecasters, that is, using the population of foreign forecasts on the one hand and the population of the local forecasts on the other, and then compare the coefficients. Would that comparison tell us which group is better informed?

In our setup, this regression can be written, for each $j=1,..J$, $m=1,..12$ and $k=l,f$, where $l$ refers to the local forecasters' population and $f$ refers to the foreign forecasters' population:
\begin{equation}Error_{jkt}^m=\beta^{CGm}_{jk}Revision_{jkt}^m+\delta_{jk}^m+\lambda_{jkt}^m\label{eq:consensus}
\end{equation}
$Error_{jkt}^m=\frac{1}{N^k(j)}\sum_{i\in\textit{S}^k(j)}Error_{ijt}^m$, $Revision_{jkt}^m=\frac{1}{N^k(j)}\sum_{i\in\textit{S}^k(j)}Revision_{ijt}^m$, are the consensus error and the consensus revision in location $k=l,f$, $\delta_{jk}^m$ are country-month-location fixed effects and $\lambda_{jkt}^m$ is an error term.

Columns (5) and (6) of Table \ref{tab:tab_main} display the results of the estimation of $\beta^{CGm}_{jk}$ using the mean-group estimator, under the assumption that the coefficients $\beta^{CGm}_{jk}$ differ across countries, months and locations. While the $\beta^{CGm}_{jk}$ coefficient is positive on average, as is expected, there is no significant difference between foreign and local coefficients.\footnote{In Appendix \ref{tab:tab_rob_consensus}, we provide the results when assuming that the $\beta^{CG}$ coefficients only differ across countries and locations. The results are very stable across specifications.}

This does not necessarily mean that there are no information asymmetries between local and foreign forecasters. To show this, it will be useful to assume that for a given country $j$, the information and behavioral bias structure differs between local and foreign forecasters, but it is homogeneous within the local and foreign forecaster pools:
\begin{assumption}[Homogeneous behavioral biases and precision within-location]\label{ass:loc_hom} $\tau_{ij}^m=\tau_{jl}^m$, $\hat\tau_{ij}^m=\hat\tau_{jl}^m$ and $\hat\rho_{ij}=\hat\rho_{jl}$ if $i\in\mathcal{S}^l(j)$ and $\tau_{ij}^m=\tau_{jf}^m$, $\hat\tau_{ij}^m=\hat\tau_{jf}^m$ and $\hat\rho_{ij}=\hat\rho_{jf}$ if $i\in\mathcal{S}^f(j)$, for all $j=1,..J$ and $m=1,..,12$.
\end{assumption}
We additionally assume the absence of behavioral bias (Assumption \ref{ass:nobias}) for tractability. The following proposition shows that, in the presence of public information, the relation between $\beta^{CG}$ and the precision of private information is not monotonic (see the proof in Appendix \ref{proof:consensus}), even in the absence of behavioral biases.

\begin{prop}\label{prop:consensus} Suppose that Assumptions \ref{ass:nobias} and \ref{ass:loc_hom} are satisfied: there are no behavioral biases and the precision parameters are identical within foreign forecasters and within local forecasters. The coefficients $\beta^{CGm}_{jk}$, estimated by OLS, are non-monotonous in $\tau_{jk}^m$.
%\begin{itemize}
%\item[(i)] decreasing in $\tau_{jk}^m$ when $\kappa_{j}^m=0$;
%\item[(ii)] increasing in $\tau_{jk}^m$ around $\tau_{jk}^m=0$.
%\end{itemize}
\end{prop}
To understand, consider first the limit case studied by \citet{CoibionGorodnichenko2015} with no public information ($\kappa_j^m=0$), where $\beta^{CGm}_{jk}=(1-G_{jk}^m)/G_{jk}^m$. The coefficient is directly related to the Kalman gain. A large coefficient implies a small Kalman gain and hence noisier information. Therefore, if foreigners have noisier, less precise information ($\tau_{jl}^m>\tau_{jf}^m$), we would have $\beta^{CGm}_{jl}<\beta^{CGm}_{jf}$. But with public information, this relation can be reversed. Suppose that local forecaster have access to a private signal on top of the public signal, while foreign forecasters only observe the public signal ($\tau_{jf}^m$ goes to zero). For foreign forecasters, the public signal becomes the only valid signal, so that all foreign forecasters share the same information. In that case, rational expectations imply that $\beta^{CGm}_{jf}=0$. Since $\beta^{CGm}_{jl}$ is strictly positive, we would observe this time $\beta^{CGm}_{jl}>\beta^{CGm}_{jf}$, while foreigners still have less precise information.

%However, is the presence of a public signal, that is, when $\kappa_j^m>0$, $\beta^{CGm}_{jk}$ is not a straightforward function of the information structure, so it is not clear what to infer from $\beta^{CGm}_{jl}<\beta^{CGm}_{jf}$. This is due to the presence of aggregate noise. This aggregate noise, as discussed in \citet{CoibionGorodnichenko2015}, introduces a negative bias in the estimation of $\beta_{jk}^{CGm}$. While the correlation between the error and the revision driven by the fundamental $x_{jt}$ is positive, the public noise introduces a negative correlation. CG argue that because the bias is negative, a positive coefficient is still a sign of noisy information. However, in order to test for \emph{differences} in the quality of private information by comparing $\beta^{CGm}_{jl}$ and $\beta^{CGm}_{jf}$, we need $\beta^{CGm}_{jk}$ to be a monotonic function of $\tau_{jk}^m$.

%But the case described in (ii) shows that, in fact, $\beta^{CGm}_{jl}$ is not always decreasing in $\tau_{jk}^m$. Take the limit case where the precision of the private signal is vanishing ($\tau_{jk}^m$ goes to zero). In that case, the public signal becomes the only valid signal, so that all forecasters share the same information, which corresponds to the public information. In that case, rational expectations imply that the $\beta^{CGm}_{jl}$ coefficient goes to zero. Since $\beta^{CGm}_{jl}$ is otherwise strictly positive, this means that $\beta^{CGm}_{jl}$ is locally increasing in $\tau_{jk}^m$ in the vicinity of $\tau_{jk}^m=0$. In that case, $\beta^{CGm}_{jl}<\beta^{CGm}_{jf}$ would imply that foreigners have more precise information ($\tau_{jf}^m<\tau_{jl}^m$).

We thus need tests that identify the degree of information frictions and that are robust to public information. We propose two such tests.

\subsubsection{Fixed-effect regressions}

For our first test of asymmetric information, we use fixed-effect regressions. We use the following pooled regression, for each $j=1,..J$, $m=1,..,12$ and $k=l,f$:
\begin{equation}Error_{ijkt}^m=\beta^{FEm}_{jk}Revision_{ijkt}^m+\delta_{jkt}^m+\lambda_{ijkt}^m\label{eq:pooledFE}
\end{equation}
where $\delta_{jkt}^m$ are country-location-time fixed effects and $\lambda_{ijkt}^m$ is an error term. The difference between this regression and the BGMS regression is that it controls for time fixed effects. These fixed effects control for aggregate shocks ($\epsilon_{jt}$ and $u_{jt}$), which are not observed by forecasters when they revise their forecasts. The coefficient $\beta^{FEm}_{jk}$ is thus not an indicator of the deviation from rational expectations, but rather captures the cross-sectional covariance between the errors and the revisions. This coefficient is necessarily negative: optimistic forecasters make a more negative error than pessimistic forecasters. The following proposition shows that, if the biases are homogeneous across groups, then a more negative $\beta^{FEm}_{jk}$ indicates noisier information (see proof in Appendix \ref{proof:pooledFE}).
\begin{prop}\label{prop:pooledFE} Suppose that Assumptions \ref{ass:hom} and \ref{ass:loc_hom} are satisfied: forecasters have identical behavioral biases and the precision parameters are homogeneous within foreign forecasters and within local forecasters.
Consider the $\beta^{FEm}_{jk}$ coefficients, estimated by OLS. If $0<\hat\rho_j<1$, then $\beta^{FEm}_{jf}<\beta^{FEm}_{jl}$ if and only if $\tau_{jl}^m>\tau_{jf}^m$.
\end{prop}
If the foreign and local forecasters have similar behavioral biases and if forecasters believe that there is some persistence in the process, then $\beta^{FEm}_{jf}<\beta^{FEm}_{jl}$ reflects an informational advantage for locals.\footnote{Note that adding time fixed effects to the regression is equivalent to subtracting the cross-forecaster average from each side of the equation:
$$-\left(E_{ijkt}^m(x_{jt})-E_{jkt}^m(x_{jt})\right)=\beta^{FEm}_{jk}(Revision_{ijkt}^m-Revision_{jkt}^m)+\lambda_{ijkt}^m$$
In that sense, this test is similar in spirit to \citet{Goldstein2021}, who proposes to measure information frictions by estimating the persistence of a forecaster's deviation from the mean:
$$\left(E_{ijkt}^m(x_{jt})-E_{jkt}^m(x_{jt})\right)=\beta^{Gm}_{jk}\left(E_{ijkt-1}^m(x_{jt})-E_{jkt-1}^m(x_{jt})\right)+\lambda_{ijkt}^m$$
$\beta^{Gm}_{jk}=1-G^{m}_{jk}$ is also directly and monotonically related to the degree of information frictions.}

%In columns (1) and (2), we assume that $\beta^{CGm}_{jk}$ differs across countries and locations. In columns (3) and (4), we assume that $\beta^{CGm}_{jk}$ can also differ across months. There does not appear to be any significant difference between foreign and local coefficients.

We estimate Equation \eqref{eq:pooledFE} under the assumption that the $\beta^{FE}$ coefficients differ across countries, locations, and months. We then regress these coefficients on the Foreign dummy and report the results in Columns (7) and (8) of Table \ref{tab:tab_main}. Note first that the estimated coefficients are negative on average, as predicted. Second, the coefficient for Foreign dummy is significantly negative for inflation. For GDP growth, it is negative as well, but smaller in magnitude and not significant. This is consistent with the preliminary evidence of Section \ref{sec:mistakes} where we have shown that foreign forecasters made relatively larger errors on inflation than on GDP growth.\footnote{In Appendix \ref{tab:tab_rob_FE}, we provide the results when assuming that the $\beta^{FE}$ coefficients only differ across countries and locations. The results are very stable across specifications.}

\subsubsection{Foreign-local disagreement}

Our second test of asymmetric information is based on how disagreement between local and foreign forecasters reacts to public information.
We define the disagreement between the local and foreign forecasters as follows:
\begin{equation}
Disagreement_{jt}^m=E_{jlt}^m(x_{jt})-E_{jft}^m(x_{jt})
\label{eq:dis}
\end{equation}
where $E_{jkt}^m(x_{jt})=\frac{1}{N(j)^k}\sum_{i\in\textit{S}^k(j)}E_{ijkt}(x_{jt})$, $k=l,f$, is the location-specific average expectation.

Consider now the following regression:
\begin{equation}Disagreement_{jt}^m=\beta^{DISm}_{j}Revision_{jt}^m+\beta^{0m}_{j}x_{jt}+\beta^{2m}_jE_{jlt-1}^m(x_{jt})+\beta^{3m}_jE_{jft-1}^m(x_{jt})+\delta_{j}^m+\lambda_{jt}^m\label{eq:disagreement}\end{equation}
where $Revision_{jt}^m=\frac{1}{2}(Revision_{jlt}^m+Revision_{jft}^m)$ is the average of local and foreign consensus revisions for country $j$ in year $t$ and month $m$.

We can show that the sign of $\beta^{DISm}_{j}$ depends on the precision of local forecasters' information relative to foreign forecasters when the behavioral biases are homogeneous across locations (see proof in Appendix \ref{proof:disagreement}).
\begin{prop}\label{prop:disagreement} Suppose that Assumptions \ref{ass:hom} and \ref{ass:loc_hom} are satisfied: forecasters have identical behavioral biases and the precision parameters are homogeneous within foreign forecasters and within local forecasters. Consider the coefficients $\beta^{DISm}_{j}$, estimated by OLS. Then $\beta^{DISm}_{j}<0$ if and only if $\tau_{jl}^m>\tau_{jf}^m$.
\end{prop}
Intuitively, because we control for the fundamental $x_{jt}$, $\beta^{DISm}_{j}$ captures the reaction of disagreement to the public noise. As a consequence, $\beta^{DISm}_{j}$ is negative if the foreign expectations are more sensitive to the public signal. This is the case if the foreign forecasters' private information is less informative than that of local forecasters.

We estimate Equation \eqref{eq:disagreement} under the assumption that the $\beta^{Dis}$ coefficients differ across countries and across months.\footnote{In Appendix \ref{tab:tab_rob_disag}, we provide the results when assuming that the $\beta^{DIS}$ coefficients only differ across countries. The results are very stable across specifications} We then test whether the coefficients are different from zero on average and report the results in columns (9) and (10) of Table \ref{tab:tab_main}. The disagreement coefficients are significantly negative on average for both inflation and GDP growth. The coefficient is smaller for GDP growth, which is consistent with our previous results.\footnote{Note that because $\beta^{FE}$ and $\beta^{Dis}$ are based on country-level regressions, we cannot control for forecaster fixed effects in Columns (7) to (10). The Foreign coefficient could then be driven by the fact that the local forecasts are more likely to be produced by multinational firms. To circumvent this endogeneity issue, we replicate columns (7) to (10) using only forecasts produced by non-multinational firms. The results are provided in Table \ref{tab:tab_r3} in the Appendix and remain very similar.}

\subsection{Robustness analysis}

We perform robustness tests where (i) we use alternative vintage series to compute the forecast errors, (ii) we include only forecasters who produce forecasts for both local and foreign forecasts, (iii) we use alternative trimming strategies, (iv) we exclude forecasts that are identical to their previous release. We replicate the results of Table \ref{tab:updating_errors_main_small} and Table \ref{tab:tab_main} under these different specifications.  Finally, (v) we use \citet{Driscoll1998} standard errors for our our main panel estimation of Table \ref{tab:updating_errors_main_small}, Columns (2) and (3). The details of the analysis is available in Section \ref{sec:robustness} in the Appendix. The results remain robust.

%The only exception is when we use an alternative definition of foreign forecasters. In our baseline analysis, a foreign forecaster is defined as a forecaster that has neither its headquarters nor any subsidiary located in the country. This definition suggests that local subsidiaries play a role in forecast formation. In this robustness check, we ignore the role of subsidiaries and deem a forecaster to be foreign if its headquarters are located in another country, whether a subsidiary is present in the country or not. Compared to the 28\% of foreign forecasters in the baseline results, 64\% of the forecasters are defined to be foreign according to this alternative definition. Overall, our results remain robust, even though they are less pronounced and more imprecisely estimated. Ignoring the role of subsidiaries weakens the results, which suggests that the local information gathered by subsidiaries matters. We investigate this issue further in Section \ref{sec:heterogeneity}, where we explore heterogeneity.
