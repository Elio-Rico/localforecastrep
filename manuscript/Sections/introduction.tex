
\label{sec:introduction}
The informational advantage of local forecasters over foreign forecasters regarding macroeconomic fundamentals has far-reaching consequences. Information asymmetries are a primary explanation for the tendency of investors to prefer domestic assets in their investment portfolios, known as the home bias in asset holdings, originally documented by  \citet{FrenchPoterba1991}.\footnote{On asymmetric information and the home bias, see for instance \citet{Admati1985}, \citet{Portesetal2001}, \citet{DeMarcoetal2021}.} Information asymmetries are also a potential source of capital flow volatility, since disagreement between foreign and domestic investors generates cross-border asset trade.\footnote{See \citet{BrennanCao1997},  \citet{TillevanWincoop2014}, \citet{BenhimaCordonier2022}.} Beyond their impact on international asset markets, they also constitute a barrier to the international trade in goods (potentially accounting for the ``missing trade''), as highlighted by \citet{AndersonvanWincoop2004}. Finally, recent papers underscore their role in international business cycle comovement.\footnote{See \citet{Buietal2021}.} However, there remains a lack of direct evidence regarding the existence of information asymmetry on macroeconomic fundamentals and quantitative estimates of the extent of this asymmetry.

We fill this gap by exploiting a unique dataset of inflation and GDP growth forecasts for the current and the next year provided by local and foreign forecasters. Unlike previous studies, the forecaster and country dimensions of the panel allows us to control for a rich set of fixed effects. We first show that foreign forecasters publish and update their forecasts about 10\% less frequently than local forecasters. Since forecasters in our dataset update their forecasts on average 6 times a year, this corresponds to about 0.6 fewer updates annually. They also make more mistakes than local forecasters, and foreign forecasters' absolute error is on average 6-9\% higher than local forecasters, which corresponds to 0.035-0.04 percentage points on average. We argue that these estimates are economically relevant.

We then investigate the role of information frictions and behavioral biases in explaining our results about forecast errors. We do this in two steps. First, we rule out behavioral biases such as over-reaction to new information as a key explanation of the foreigners' excess mistakes, by showing that the local and foreign biases do not differ systematically. Second, we show that local forecasters have more precise private information. To do so, we build on and extend the fast-growing literature that uses model-based tests to identify frictions in expectation formation. In particular, we provide tests of asymmetric information that are robust to the presence of public signals.

We then explore the determinants of the information asymmetry between local and foreign forecasters. First, we show that the location of subsidiaries plays an important role for the information produced by multinationals. Having a subsidiary located in a country is associated with the same informational advantage as having headquarters based there. Second, geography matters. Linguistic distance in particular appears to be a major barrier to information. Foreign forecasters also forecast better when the economic ties (trade and financial) between the country where their headquarter is located and the country they are forecasting are stronger. This evidence is consistent both with the existence of exogenous barriers to information and with smaller incentives to acquire information. While the limitations of our data prevents us from fully disentangling these two channels, we provide arguments that they are both at play.

Interestingly, we show that information asymmetries are inversely related to forecasting uncertainty. Indeed, the local advantage is higher for short horizons, for inflation (as opposed to GDP growth), and for large countries. In all these situations, the forecasting uncertainty (measured by the average forecast error) happens to be \emph{smaller}. This evidence suggests that when information is available, local forecasters are better at finding it. Interestingly, the information asymmetry \emph{increases} in the course of a year (the asymmetry is higher in December than in January). This is consistent with the idea that local forecasters are more aware of the regular releases of partial GDP growth and inflation figures and integrate this information faster. Consistently, inflation figures are typically available at a higher frequency and with a shorter lag than GDP, making the access to that information an even greater advantage. However, we find no evidence that the difference in forecast errors between local and foreign forecasters is higher for developing countries, when institutional quality is poor, or when macroeconomic volatility is high.

%Information advantages have been used to explain exchange rate variations (Evans and Lyons (2005), Bacchetta and van Wincoop (2006), international consumption correlation puzzle (Coval (2003)), international capital flows (Brennan and Cao (1997)), a bias towards investing in local firms (Coval and Moskowitz (2001)), and the own-company stock puzzle (B and Wang (2003)). Information asymmetry is also the basis for other home bias explanations, such as ambiguity aversion (Uppal and Wang). All these explanations are bolstered by our finding that information are not only sustainable when information is mobile, but by the fact that asymmetric information can be amplified when investors can choose what to learn.

%The canonical reference on asymmetric information with multiple assets is Admati (1985). Work on asymmetric information and the home bias, in particular, includes Pastor (2000), Brennan and Cao (1997), and Portes, Rey, and Oh (2001).

%Ahearne, Griever, and Warnock (2004)


%ChatGPT thrid version: Are our estimates economically significant? Consider the impact of information asymmetries on international trade. While there are no precise quantitative estimates provided by Rauch (2001) on the percentage difference in information precision required to fully explain trade barriers, his qualitative analysis emphasizes the role of networks in reducing information frictions and facilitating trade, particularly for differentiated products. This suggests that our estimates can be informative in contexts where detailed empirical estimates are lacking. \footnote{See \citet{Rauch2001} for an analysis of the role of networks and information asymmetries in trade. Although Rauch does not provide specific quantitative estimates, his work highlights the significant impact of these frictions on trade. See also \citet{AndersonvanWincoop2004}, \citet{NunnTrefler2013}, and \citet{HeadMayer2013} for discussions on the broader implications of information asymmetries in trade.} Our estimates are more in line with models that rely on amplification effects. For instance, \citet{Allen2014} shows that small differences in information precision can lead to significant trade imbalances when firms decide on market entry and investment based on their information sets. \footnote{See also \citet{Chaney2008}, who discusses how minor information differences can influence trade patterns and volumes through amplification effects.} Specifically, \citet{Chaney2008} finds that information frictions can significantly affect trade flows by influencing firms' decisions to enter foreign markets. Similarly, \citet{NunnTrefler2013} discuss how relational contracts and trust, which are closely tied to information asymmetries, impact international trade. Therefore, while our estimates of information asymmetries alone may not fully explain the observed trade barriers, they are consistent with theoretical models where small initial differences in information can be amplified through market dynamics and firm behaviors.

This paper contributes to the recent literature that uses professional forecasters' expectations to identify information frictions and behavioral biases. This literature has used reduced-form estimations as indicators of deviations from Full-Information Rational Expectations (FIRE). \citet{CoibionGorodnichenko2015} (CG henceforth) use the estimated coefficient in the regression of the consensus error on the consensus revision as an indicator of deviations from Full Information (FI). \citet{Bordaloetal2020} use the estimated coefficient in the individual pooled regression as an indicator of deviations from Rational Expectations (RE).\footnote{An earlier literature has previously identified deviations from rationality by studying the joint behavior of actual on predicted values, the auto-correlation of forecasts revisions and the predictability of errors. See, for example, \citet{MincerZarnowitz1969} and \citet{Nordhaus1987}.} We borrow this test to assess whether domestic and foreign deviations from RE differ.

However, CG's FI test, which has been commonly used in the literature, is not adapted to our purpose. In the presence of public information, the CG coefficient, which is a common measure of information frictions, is biased. Importantly, the bias is not a monotonic function of the precision of private signals. Comparing the CG coefficient across local and foreign forecasters cannot indicate which group faces more frictions. We thus provide two tests that are robust to the presence of public information. The first relies on individual regressions with country-time fixed effects to capture aggregate shocks and the public signals. This test is similar in spirit to \citet{Goldstein2021}, who proposes to use forecasters' deviations from the mean to measure information frictions robustly. The second test infers the relative precision of private information from the relative reaction of expectations to public signals.

This paper also belongs to the empirical literature documenting the informational advantage of locals. Many studies provide indirect evidence of asymmetric information between domestic and foreign investors by showing that location matters for portfolio composition and for portfolio returns.\footnote{See for instance \citet{KangStulz1997}, \citet{GrinblattKeloharju2001}, \citet{PortesRey2005}, \citet{Hau2001}, and \citet{Clemensetal2020}. Based on investor choices and returns, some papers find that foreign investors perform better than local investors (e.g. \citet{GrinblattKeloharju2000}). This could be explained by the specialization of some investors in some specific markets where they have an initial informational advantage. Location can be a source of this informational advantage, but it is not the only one. Therefore, information heterogeneity can also lead to specialization in either domestic or non-domestic assets (see \citet{VanNieuwerburghVeldkamp2010} and \citet{DeMarcoetal2021}).} Other studies document foreigners' lack of attention to domestic information.\footnote{See for instance \citet{Mondriaetal2010} and \citet{Czirakietal2021}.} In contrast to these studies, we investigate whether location affects the quality of forecasters' information, thus providing direct evidence of information asymmetries. Closest to our study is the paper by \citet{Baeetal2008}, which studies the performance of local and foreign analysts in forecasting earnings for firms. Our focus is different since we examine whether locals outperform foreigners in forecasting aggregate variables. Moreover, we investigate whether the foreigners' excess mistakes come from information frictions or behavioral biases.

The paper is structured as follows. Section \ref{sec:data} describes our dataset. Section \ref{sec:mistakes} documents the updating frequency of forecasts and foreign forecasters' excess mistakes. Section \ref{sec:model} lays down a model of expectation formation and tests for the sources of the foreigners' excess mistakes. Sections \ref{sec:geography} and \ref{sec:heterogeneity} investigate respectively the drivers of the foreign penalty and of its heterogeneity. Finally, Section  \ref{sec:conclusion} concludes.

