
\subsection{Foreign Forecasters Update Their Forecasts Less Often}

One potential source of difference between local and foreign forecasters is sticky information à la \citet{MankiwReis2002}: foreign forecasters may not update their information as frequently. To examine this hypothesis, we compute the number of published forecasts for each year-forecaster-country unit, which we denote $N_{ijt}$. The distribution of these publication frequencies is provided in Panels (a) and (b) of Figure \ref{fig:updates} in the Appendix.
We test formally whether foreign forecasters publish forecasts less often by taking the log of $N_{ijt}$ and estimating
\begin{align}
	\ln(N_{ijt}) = \tilde\delta_{it} + \bar{\delta}_{jt} +
\beta \text{Foreign}_{ij} + \varepsilon_{ijt}  \,, \label{eq:regN}
\end{align}
where  $\tilde\delta_{it} $ and $\bar{\delta}_{jt} $ are respectively forecaster-year and country-year fixed effects.% $\text{Foreign}_{ij} $ is a dummy that takes the value of 1 if forecaster $i$ is foreign to country $j$, and 0 otherwise.

The results are reported in Column (4) of Table \ref{tab:updating_errors_main_small}: foreign forecasters publish their forecasts 10\% to 12\% less often than local forecasters. Since the average publication frequency is 9 publications per year, this means that foreign forecasts would have about one less publication per year. The difference in publication frequency between local and foreign forecasters is smaller when considering GDP growth (as opposed to inflation).\footnote{Table \ref{tab:tab_updating_rob} in the online Appendix shows the results for alternative, less rich fixed-effect specifications. The coefficient of Foreign remains insignificant until we control for forecaster fixed effects (Column (3)). Controlling for country-year fixed effects and forecaster-year fixed effects matters since the coefficients decline slightly when we introduce these fixed effects (Columns (4) and (5)).}

Note that forecasters may publish a forecast without necessarily updating it, so the publication frequency is an imperfect measure of the updating frequency. We thus follow \citet{Andrade2013} and compute the number of yearly forecasts when considering only ``distinct'' forecasts, that is, forecasts that differ from the previous release. Measured in this way, the average updating frequency drops to about 6 times a year.\footnote{An identical forecast does not necessarily reflect the absence of new information, because of rounding, so this measure may understate the updating frequency. However, \citet{Andrade2013} find that rounding actually does not significantly drive infrequent updating.} Panels (c) and (d) in Figure \ref{fig:updates} in the online Appendix provides the distribution of the number of distinct forecasts in a given year. We use this measure to estimate Equation \eqref{eq:regN} and report the results in Column (5) of Table \ref{tab:updating_errors_main_small}. The results, in fact, barely change (10-12\%). Since the average updating frequency is 6 updates per year, this means that foreign forecasts would have about 0.6 less update per year.

The results of Column (2) must be reevaluated under the hypothesis of sticky information. Indeed, one potential source of the foreign excess error could be that the published forecasts are not updated as frequently. We thus report in Column (3) the results of regression \eqref{eq:regModelFE} when restricting the sample to distinct forecasts. The results do not change. This means that the excess error of foreign forecasters is not due to the lower updating frequency: conditional on updating, foreigner still make larger mistakes.\footnote{These results are consistent with \citet{Andrade2013}, who find that disagreement among forecasters does not appear to be entirely driven by infrequent information updating.}

Table \ref{tab:updating_errors_app_small} in the online Appendix shows the results for updates about future inflation and GDP growth. The difference in the frequency of forecast updates is mostly the same for current and future variables (Columns (4) and (5)). This is not surprising since forecasters often produce their forecasts regarding the next year at the same time as they produce forecasts regarding the current year. 