

We provide direct evidence of asymmetric information between domestic and foreign forecasters. Using professional forecaster expectation data, in which we determine the location of each forecaster-country pair, we show that foreign forecasters update their information less frequently compared to local forecasters and produce less precise forecasts, even conditional on updating their forecasts. We rule out over-confidence and over-extrapolation, and behavioral biases in general, as drivers of the foreigners' excess mistakes, and identify differences in information asymmetries between foreign and local forecasters. Our results have implications for the modeling and calibration of international trade and finance models with heterogeneous information, since (1) we provide estimates of the excess errors of foreign forecasters and their relative updating frequency and (2) we prove that the source of asymmetry between local and foreign forecasters is informational. Finally, we show that both exogenous barriers to information and incentives to acquire information drive the foreign penalty. In general, the foreign penalty is not stronger when forecasting is more uncertain.

