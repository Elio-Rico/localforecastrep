
In our main specification, foreign forecasters make, on average, forecasting errors that are 6\% larger. We have already shown that the foreign penalty is lower—though not significantly so—for financial and multinational forecasters. We now further examine how the foreign penalty varies with the forecast variable, the forecast horizon, country characteristics, and whether it is state-dependent. To do so, we estimate a regression similar to Equation \eqref{eq:geography}:


\begin{align}
	\ln(|Error_{ijt,t+h}^{m,x}|)= \delta_{it,h}^{m,x} +\tilde\delta_{jt,h}^{m,x} +  \beta \text{Foreign}_{ij} +\gamma \text{Foreign}_{ij}\times X_{ijt,h}^{m,x}+ \varepsilon_{ij,t}^m  \,, \label{eq:heterogeneity}
\end{align}
Here, $X_{ijt,h}^{m,x}$ is a set of variables that varies either across the forecast variable (growth and inflation), horizon (current or future), time, country, or time and country, so that the levels are already absorbed by the fixed effects. If the $\gamma$ coefficients are significant, it means that the foreign penalty is heterogeneous across that dimension. It will be useful though to also examine how the linear term $X_{ijt,h}^{m,x}$ affects the errors to answer the question: is the foreign penalty larger in situations where the errors are also larger? To identify the role of $X_{ijt,h}^{m,x}$, we will need to remove some fixed effects. In these specifications, we keep our \textit{Foreign} dummy as a control, but we do not interpret its estimated effect, since we will be neglecting some of the potential confounding factors and noise that we are controlling for in our main specification with richer fixed effects.

We begin by examining the role of the forecast variable and the forecast horizon. Specifically, we define \textit{GDP} as a dummy variable equal to 1 if the forecast targets GDP growth, and 0 if it targets inflation. \textit{Future} is a dummy equal to 1 if the forecast refers to the next calendar year, and 0 if it refers to the current year. Finally, \textit{Month-of-year} is a categorical variable ranging from 1 to 12, indicating the month in which the forecast is made.


The results are presented in Table \ref{tab:error_reg_labs_cs}. Column (1) examines how forecast errors vary with \textit{GDP}, \textit{Future}, and \textit{Month-of-year}. This specification includes only country-year and forecaster-year fixed effects to preserve sufficient variation in the explanatory variables. The estimates show that forecast errors are larger for GDP growth forecasts and for forecasts targeting the future year. Notably, forecast errors decline over the course of the year, which suggests that information flows continuously during the year. Column (5) explores how the foreign penalty interacts with these variables. It includes forecaster-date-variable-horizon and country-date-variable-horizon fixed effects, which absorb the main effects of \textit{GDP}, \textit{Future}, and \textit{Month-of-year}. The results indicate that the foreign penalty is significantly lower for GDP growth and for forecasts of the future year. Interestingly, the penalty increases over time within the year. As shown in Panel (a) of Figure \ref{fig:heterogeneity} in the online Appendix, the average foreign penalty rises across calendar months. This pattern suggests that, somewhat paradoxically, the foreign penalty is larger when overall forecast uncertainty is lower.


As shown in Column (6), the foreign penalty does not exhibit state dependence: it does not increase significantly during recessions or periods of global uncertainty, measured by the \textit{VIX}, while Column (2) shows that forecast errors are on average larger during periods of heightened uncertainty and in recessions. This suggests that while both local and foreign forecasters make larger errors in adverse times, the relative difference between them remains stable.

We also consider several country-specific characteristics: an \textit{Emerging} economy dummy, institutional quality (from the World Development Indicators, in the table referred to as \textit{Institutions}), and country size (\textit{log of GDP }evaluated at purchasing power parity).\footnote{The data sources are the following: country size \citep{cpigravity22} and quality of institutions \citep{wdi22}. The list of emerging economies is given in the online Appendix Table \ref{tab:app_emerging_advanced}.}



In Column (7), the foreign penalty does not differ between emerging and advanced economies, nor does it depend on institutional quality, while Column (3) shows that, on average, errors are larger for emerging economies, and that stronger institutions are associated with a lower error. However, the foreign penalty is higher for larger countries, for which errors are lower on average (Column (4)). As with the forecast variable and horizon, lower average uncertainty is linked to a higher foreign penalty. The main picture is unchanged when putting together all the interaction terms (Column (8)).\footnote{In the online Appendix Table \ref{tab:error_reg_labs_cs_ind}, we show that the results do not change either when we interact the Foreign dummy with one variable at a time. Additionally, Figure \ref{fig:heterogeneity} in the online Appendix displays the Foreign coefficients per year, month and country. Panel (a) shows the average foreign penalty per month, which appear to be increasing over the course of the year. Panel (b) shows the average foreign penalty per year. No systematic pattern appears, which is consistent with our results. Panel (c) shows the average foreign penalty per country. The estimates are heterogeneous across countries but no systematic difference between Emerging and Advanced economies appears, confirming our results. We also conduct a similar analysis, using the estimated coefficients from our asymmetric information tests, $\beta^{FE}$ and $\beta^{DIS}$. The results, which are shown in Table \ref{tab:FE_reg_mg}, are broadly consistent with the evidence on the errors, except that they are less precisely estimated.}


{\setstretch{1}
	\begin{table}[H] \centering
\newcolumntype{C}{>{\centering\arraybackslash}X}

\caption{Variable, Horizon, Time and Country Dependence}
\label{tab:error_reg_labs_cs}
{\scriptsize
\begin{tabularx}{\linewidth}{l C C C C C C C C}

\toprule
&\multicolumn{8}{c}{$\ln(|Error_{ijt,t}^m|)$}\tabularnewline\cline{2-9} &{(1)}&{(2)}&{(3)}&{(4)}&{(5)}&{(6)}&{(7)}&{(8)} \tabularnewline
{Coefficient}&{}&{}&{}&{}&{}&{}&{}&{} \tabularnewline
\midrule \addlinespace[0pt]
\midrule Foreign&0.05***&0.09***&0.18***&0.03&0.06***&0.05*&--0.39**&--0.43*** \tabularnewline
&(0.02)&(0.02)&(0.06)&(0.05)&(0.02)&(0.02)&(0.17)&(0.16) \tabularnewline
GDP&0.30***&&&&&&& \tabularnewline
&(0.07)&&&&&&& \tabularnewline
Future&0.94***&&&&&&& \tabularnewline
&(0.05)&&&&&&& \tabularnewline
Month-of-year&--0.09***&&&&&&& \tabularnewline
&(0.00)&&&&&&& \tabularnewline
VIX&&0.02***&&&&&& \tabularnewline
&&(0.00)&&&&&& \tabularnewline
Recession&&0.26***&&&&&& \tabularnewline
&&(0.07)&&&&&& \tabularnewline
Emerging&&&0.27***&0.08&&&& \tabularnewline
&&&(0.09)&(0.10)&&&& \tabularnewline
Institutions&&&--0.06***&--0.11***&&&& \tabularnewline
&&&(0.02)&(0.02)&&&& \tabularnewline
ln(GDP)&&&&--0.13***&&&& \tabularnewline
&&&&(0.03)&&&& \tabularnewline
Foreign $\times$ GDP&&&&&--0.04**&&&--0.04* \tabularnewline
&&&&&(0.02)&&&(0.02) \tabularnewline
Foreign $\times$ Future&&&&&--0.03**&&&--0.03** \tabularnewline
&&&&&(0.01)&&&(0.02) \tabularnewline
Foreign $\times$ Month-of-year&&&&&0.01**&&&0.00** \tabularnewline
&&&&&(0.00)&&&(0.00) \tabularnewline
Foreign $\times$ VIX&&&&&&0.00&&0.00 \tabularnewline
&&&&&&(0.00)&&(0.00) \tabularnewline
Foreign $\times$ Recession&&&&&&0.02&&0.04 \tabularnewline
&&&&&&(0.02)&&(0.03) \tabularnewline
Foreign $\times$ Emerging&&&&&&&0.03&0.04 \tabularnewline
&&&&&&&(0.03)&(0.03) \tabularnewline
Foreign $\times$ Institutions&&&&&&&0.01&0.01 \tabularnewline
&&&&&&&(0.01)&(0.01) \tabularnewline
Foreign $\times$ ln(GDP)&&&&&&&0.02**&0.02*** \tabularnewline
&&&&&&&(0.01)&(0.01) \tabularnewline
N&602,113&601,907&375,846&366,653&389,295&389,218&366,401&366,401 \tabularnewline
$ R^2 $&0.36&0.25&0.37&0.37&0.70&0.70&0.70&0.70 \tabularnewline
Cty. $ \times $ Year FE&\checkmark&&&&&&& \tabularnewline
For. $ \times $ Year FE&\checkmark&&&&&&& \tabularnewline
Cty $ \times $ Var. $ \times $ Hor. FE&&\checkmark&&&&&& \tabularnewline
For. $ \times $ Var. $ \times $ Hor. FE&&\checkmark&&&&&& \tabularnewline
For. $ \times $ Date $ \times $ Var. $ \times $ Hor. FE&&&\checkmark&\checkmark&\checkmark&\checkmark&\checkmark&\checkmark \tabularnewline
Cty $ \times $ Date $ \times $ Var. $ \times $ Hor. FE &&&&&\checkmark&\checkmark&\checkmark&\checkmark \tabularnewline
\bottomrule \addlinespace[\belowrulesep]

\end{tabularx}
\begin{flushleft}
\footnotesize \begin{minipage}{1\textwidth} \vspace{-10pt} \begin{tabnote} \textit{Notes:}   The table shows the regression of the log absolute forecast error of current and future CPI and GDP on regressors with different fixed-effects specifications. All standard errors are clustered at the country, forecaster and date levels. \end{tabnote} \end{minipage}  
\end{flushleft}
}
\end{table}

}

\paragraph{Discussion}


The asymmetry of information between local and foreign forecasters is unaffected by the development status of the economy or the quality of institutions. This aligns with existing evidence: \citet{Baeetal2008}, who study whether local analysts are better at forecasting local firms' earnings, find that investor protection and the country's development status do not influence the foreign penalty.

We do find that variables such as country size, the forecast variable, and the forecast horizon influence the foreign penalty. However, the penalty is generally higher when forecasting uncertainty is lower, suggesting that local forecasters are better at finding and exploiting available macroeconomic information. These results are consistent with the locals' better access to locally-produced information (by knowing when and where relevant information is released). The stronger information asymmetry for current-year GDP growth or inflation forecasts, and its increase over the year (greater in December than January), is consistent with the assumption that local forecasters are exposed to the regular releases of partial GDP growth and inflation figures and integrate this information faster. Interestingly, inflation is typically available at a higher frequency and with a shorter lag than GDP, making the access to that information an even greater advantage. This is consistent with Table \ref{tab:updating_errors_main_small}, where we can see that the difference in updating frequency is 2\% larger for inflation forecasts than for GDP growth forecasts.
