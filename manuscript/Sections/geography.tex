
The preceding sections have established that asymmetric information between local and foreign forecasters explains foreigners' excess errors. In this section, we use our multi-country, multi-forecaster panel to explore whether well-identified barriers to information acquisition drive the foreign penalty. We first rely on the multinationals' multiple locations to identify the role of having a local economic activity in the country in reducing information asymmetries. We then draw from the literature on the geography of trade and finance to identify other important drivers of information asymmetries.

\subsection{The Relevance of Local Subsidiaries}

In our analysis, we have chosen to consider a forecast to be ``foreign'' if the forecasting firm has neither their headquarters nor a subsidiary located in the country. While this assumption is innocuous for non-multinational firms, whose subsidiaries (if there are any) are all located in the same country as the headquarters, it is not the case for multinational firms. This relies on the implicit assumption that a forecaster with its headquarters in the country and a forecaster with only a subsidiary in the country are equally good at forecasting that country. We reconsider this assumption here, in order to also further understand the origin of the foreign penalty. In other words, is it important to have some activity in the country to gain an informational advantage, or is it mainly the location of the headquarters that matters?

To test for the relevance of local subsidiaries, we use an alternative definition of ``foreign'', where a forecaster is foreign if its headquarters are located in another country, whether a subsidiary is present or not.  Compared to the 28\% of foreign forecasters with the baseline definition, 64\% of the forecasters are defined to be foreign according to the alternative definition. The corresponding dummy is \textit{Foreign HQ}.

{\setstretch{1}`
\begin{table}[H] \centering
\newcolumntype{C}{>{\centering\arraybackslash}X}

\caption{Multinational and Non-Multinational Forecasters}
\label{tab:error_reg_labs_mult}
{\footnotesize
\begin{tabularx}{\linewidth}{l C C C C C}

\toprule
&\multicolumn{5}{c}{$\ln(|Error_{ijt,t}^m|)$}\tabularnewline\cline{2-6} &{(1)}&{(2)}&{(3)}&{(4)}&{(5)} \tabularnewline
{Coefficient}&{}&{Non-Mult.}&{Mult.}&{Mult.}&{Mult.} \tabularnewline
\midrule \addlinespace[0pt]
\midrule Foreign&0.06***&0.25***&0.04**&& \tabularnewline
&(0.02)&(0.07)&(0.02)&& \tabularnewline
Foreign HQ&&&&0.04&0.08** \tabularnewline
&&&&(0.02)&(0.03) \tabularnewline
Local Subsidiary&&&&&--0.04** \tabularnewline
&&&&&(0.02) \tabularnewline
N&389,295&57,888&313,197&313,197&313,197 \tabularnewline
$ R^2 $&0.70&0.77&0.71&0.71&0.71 \tabularnewline
Country $ \times $ Date $ \times $ Var. $ \times $ Hor. FE&\checkmark&\checkmark&\checkmark&\checkmark&\checkmark \tabularnewline
For. $ \times $ Date $ \times $ Var. $ \times $ Hor. FE &\checkmark&\checkmark&\checkmark&\checkmark&\checkmark \tabularnewline
\bottomrule \addlinespace[\belowrulesep]

\end{tabularx}
\begin{flushleft}
\footnotesize \begin{minipage}{1\textwidth} \vspace{-10pt} \begin{tabnote} \textit{Notes:}   The table shows the regression of the log absolute forecast error of current and future CPI and GDP on regressors on different sub-samples. All standard errors are clustered at the country, forecaster and date levels. \end{tabnote} \end{minipage}  
\end{flushleft}
}
\end{table}

}

%{\setstretch{1}
%	\input{Tables/error_reg_labs_mult_rob}
%}

We construct a single sample of observations by stacking observations of inflation and GDP growth errors at different horizons. We first estimate our baseline regression:
\begin{align}
	\ln(|Error_{ijt,t+h}^{m,x}|)= \delta_{it,h}^{m,x} +\tilde\delta_{jt,h}^{m,x} +  \beta \text{Foreign}_{ij} +\varepsilon_{ij,t}^m  \,, \label{eq:main}
\end{align}
where $x=growth, inflation$ and $h=0,1$ stand respectively for the forecast variable and the horizon, and $ \delta_{it,h}^{m,x}$ and $\tilde\delta_{jt,h}^{m,x}$ are forecaster-date-variable-horizon fixed effects and country-date-variable-horizon fixed effects. $\beta$ identifies the average effect of the \textit{Foreign} dummy across variables and horizons. The results with the baseline definition of \textit{Foreign} are reported in Column (1) of Table \ref{tab:error_reg_labs_mult}. The coefficient is highly significant and equal to 0.06. This reflects the average impact of \textit{Foreign} across variables and horizons.

We then report the estimates with our baseline definition when distinguishing multinational from non-multinational firms. By comparing Columns (2) and (3), we can see that non-multinational firms have a higher foreign penalty than multinational firms (0.25 against 0.04). This is not surprising, as multinational firms are likely to have more resources and departments dedicated to forecasting. The coefficient of multinational firms is closer to our main estimate, because, as we have seen below, multinationals account for the bulk of our sample. Note, however, that because the coefficient of \textit{Foreign} is imprecisely estimated for non-multinational firms, we cannot statistically distinguish it from the \textit{Foreign} coefficient of multinational firms. Panel (c) of Figure \ref{fig:heterogeneity} in the Appendix shows the distribution of the foreign penalty across forecasters. The foreign penalty is indeed heterogeneous across forecasters, and even more so across non-multinational forecasters.

In Column (4), we replace the \textit{Foreign} dummy with \textit{Foreign HQ}. \textit{Foreign HQ} is not significant. It is expected that, if our assumption that the location of subsidiaries matter is true, then \textit{Foreign HQ} would include a heterogeneous mix of purely foreign forecasters and forecasters who are present through their subsidiaries. To further test this, we add an additional dummy variable that accounts for the presence of a local subsidiary: \textit{Local Subsidiary} is equal to one if the forecaster has a subsidiary in the country, but no headquarters. In Column (5), now that we control for the presence of a local subsidiary, \textit{Foreign HQ} becomes significantly positive. Interestingly, the sum of the two coefficients is not statistically different from zero. This means that a forecaster with no headquarters, but with a subsidiary located in the country, is just as good at forecasting as a forecaster with headquarters located in the country. It is thus enough to have \emph{some} activity in the country to benefit from a significant ``local'' advantage. This also confirms that our main assumption is valid.\footnote{In Section \ref{sec:robustness} in the Appendix, we perform an additional robustness test where we use this alternative definition of foreign forecasters and replicate our main results. The results are weaker, which confirms that taking into account subsidiaries is important to identify the foreign penalty.}

%While this difference is not significant in our sample, this suggests that having headquarters located in the country tends to provide an informational advantage as compared to having only a subsidiary. This could be due to the fact that the specialized forecasting departments within a multinational are typically located in the headquarters, or because often headquarters concentrates a higher share of the multinational's operations and profits (which would be consistent with our evidence on incentives)

%Note that this analysis also rules out a potential endogeneity bias. Suppose that a firm opens a subsidiary in a country only if it has good quality information on this country. Then, making lower forecasting mistakes would be the cause and not the consequence of the presence of a subsidiary in the country. In our main analysis, we consider the firm to be ``local'' in this case. This would thus lead to a positive Foreign dummy coefficient, without the effect being causal. The alternative dummy ForeignHQ is not endogenous (at least as long as the headquarter decision is orthogonal to information), but it is significant and of the same magnitude as our main estimate.

%This analysis motivates us to replicate the main results of the paper with this richer specification. In Section \ref{sec:robustness} in the Appendix, we thus perform an additional robustness test where we use this alternative definition of foreign forecasters, and control for the presence of local subsidiaries. (TBD). %In our baseline analysis, a foreign forecaster is defined as a forecaster that has neither its headquarters nor any subsidiary located in the country. This definition suggests that local subsidiaries play a role in forecast formation. In this robustness check, we ignore the role of subsidiaries and deem a forecaster to be foreign if its headquarters are located in another country, whether a subsidiary is present in the country or not. Compared to the 28\% of foreign forecasters in the baseline results, 64\% of the forecasters are defined to be foreign according to this alternative definition. Overall, our results remain robust, even though they are less pronounced and more imprecisely estimated. Ignoring the role of subsidiaries weakens the results, which suggests that the local information gathered by subsidiaries matters. We investigate this issue further in Section \ref{sec:heterogeneity}, where we explore heterogeneity.

\subsection{Barriers to Information}


To assess the role of the barriers to information, we use the same single sample and now estimate equations of the following form:
\begin{align}
	\ln(|Error_{ijt,t+h}^{m,x}|)= \delta_{it,h}^{m,x} +\tilde\delta_{jt,h}^{m,x} +  \beta \text{Foreign}_{ij} +\gamma X_{ijt}^{m,x}+ \varepsilon_{ij,t}^m  \,, \label{eq:geography}
\end{align}

where $X_{ijt}^{m,x}$ is a set of variables that have a forecaster-country dimension.

We build on the trade and capital flow literature that has identified information as one of the impediments to the cross-border circulation of products and capital. This literature has shown that geography, and in particular distance, retains a high explanatory power for both bilateral trade and bilateral holdings of financial assets. While in the case of trade, information is only one of the many reasons why distance plays a role, the main one being transportation costs,\footnote{See \citet{AndersonvanWincoop2004,HeadMayer2013,Allen2014}.} in the case  of capital flows,  the cost of information has been the main interpretation of the role of distance.\footnote{See \citet{Ghosh2000,GrinblattKeloharju2001,DiGiovanni2005,PortesRey2005}.}

Following \citet{Pellegrino2021}, we consider a parsimonious set of geographical variables: physical, cultural, and linguistic distance. \textit{Physical Distance} measures the geodesic distance between two countries, based on a population-weighted average of the distances between capital cities. \textit{Cultural Distance} captures distance in contemporary values and beliefs, introduced by \citet{Spolaore2016}, and \textit{Linguistic Distance} captures distance in spoken languages, introduced by \citet{Fearon2003} and constructed by \citet{Spolaore2016}.\footnote{See Appendix \ref{app:descvarbarriers} for a detailed description and source of the variables.} Note that, since forecasters are better at forecasting their domestic variables, then they could also be better at forecasting a country whose business cycles are correlated with the domestic business cycles. We thus test whether forecasting is facilitated by business cycle comovement. To do so, we use the squared correlation between the GDP growth (or inflation) of the forecaster country and the GDP growth (or inflation) of country $j$. Squaring the correlation gives the same weight to a large positive correlation and to a large negative correlation (in tables, we abbreviate the variable as \textit{BC comovement}).\footnote{Note that this corresponds simply to the $R^2$ of a regression of the GDP growth rate (or inflation) of country $j$ on the GDP growth rate (or inflation) of the forecaster country.} We expect a higher comovement to have a negative impact on the error. Finally, we add \textit{Migration}, the share of the forecaster country population that was born in country $j$. For all these variables, the forecaster country is defined as the country of the forecaster's closest subsidiary.

%\clearpage\begin{landscape}
%	{\setstretch{1}
%		\begin{table}[H] \centering
\newcolumntype{C}{>{\centering\arraybackslash}X}

\caption{The Geography of Information}
\label{tab:error_reg_labs_gravity}
{\footnotesize
\begin{tabularx}{\linewidth}{l C C C C C C C m{0.005\textwidth} C C C m{0.005\textwidth} C C}

\toprule
&\multicolumn{14}{c}{$\ln(|Error_{ijt,t}^m|)$}\tabularnewline\cline{2-15} \tabularnewline  \cline{2-8} \cline{10-12} \cline{14-15}\tabularnewline  &{(1)}&{(2)}&{(3)}&{(4)}&{(5)}&{(6)}&{(7)}&&{(8)}&{(9)}&{(10)}&&{(11)}&{(12)} \tabularnewline
{Coefficient}&{}&{}&{}&{}&{}&{}&{}&{}&{Finance}&{Finance}&{Finance}&{}&{}&{} \tabularnewline
\midrule \addlinespace[0pt]
\midrule Foreign&.0577***&.054***&.0356&.0262&.0727**&.059**&.0189&&.0389*&.0348*&.026&&.024&.079*** \tabularnewline
&(.0158)&(.0148)&(.022)&(.0249)&(.0295)&(.029)&(.024)&&(.0208)&(.0206)&(.0186)&&(.0287)&(.0256) \tabularnewline
\textbf{\emph{W.r.t. closest subs.:}} &&&&&&&&&&&&&& \tabularnewline
Physical dist.&&.0053&&&&&&&&&&&& \tabularnewline
&&(.0078)&&&&&&&&&&&& \tabularnewline
Cultural dist.&&&.0115&&&&&&&&&&& \tabularnewline
&&&(.0085)&&&&&&&&&&& \tabularnewline
Linguistic dist.&&&&.0189*&&&.0193*&&&&&&.022*& \tabularnewline
&&&&(.0108)&&&(.0103)&&&&&&(.0125)& \tabularnewline
BC comovement&&&&&.007&&&&&&&&& \tabularnewline
&&&&&(.0106)&&&&&&&&& \tabularnewline
Migration&&&&&&.015&&&&&&&& \tabularnewline
&&&&&&(.0287)&&&&&&&& \tabularnewline
Trade&&&&&&&&&&&&&&.0133 \tabularnewline
&&&&&&&&&&&&&&(.0094) \tabularnewline
\textbf{\emph{W.r.t. headquarters:}} &&&&&&&&&&&&&& \tabularnewline
Linguistic dist.&&&&&&&&&&&&&.0144& \tabularnewline
&&&&&&&&&&&&&(.0101)& \tabularnewline
Trade&&&&&&&-.0233**&&&&&&&-.0214** \tabularnewline
&&&&&&&(.0105)&&&&&&&(.0105) \tabularnewline
Foreign $\times$ Low Cap. Controls&&&&&&&&&&-.0776***&-.0811**&&& \tabularnewline
&&&&&&&&&&(.0264)&(.0325)&&& \tabularnewline
Foreign $\times$ Institutions&&&&&&&&&&&-.0116&&& \tabularnewline
&&&&&&&&&&&(.0082)&&& \tabularnewline
N&389,295&388,415&349,093&373,980&378,911&290,630&373,066&&235,608&214,216&205,122&&348,250&382,051 \tabularnewline
$ R^2 $&.698&.6977&.7056&.6995&.7004&.7065&.6995&&.7196&.7177&.7178&&.6993&.6976 \tabularnewline
Cty $ \times $ Date $ \times $ Var. $ \times $ Hor. FE&\checkmark&\checkmark&\checkmark&\checkmark&\checkmark&\checkmark&\checkmark&&\checkmark&\checkmark&\checkmark&&\checkmark&\checkmark \tabularnewline
For. $ \times $ Date $ \times $ Var. $ \times $ Hor. FE &\checkmark&\checkmark&\checkmark&\checkmark&\checkmark&\checkmark&\checkmark&&\checkmark&\checkmark&\checkmark&&\checkmark&\checkmark \tabularnewline
\bottomrule \addlinespace[\belowrulesep]

\end{tabularx}
\begin{flushleft}
\footnotesize \begin{minipage}{1\linewidth} \vspace{-10pt} \begin{tabnote} \textit{Notes:}   The table shows the regression of the log absolute forecast error on regressors accounting for the geography of information. All standard errors are clustered at the country, forecaster and date levels. \end{tabnote} \end{minipage}  
\end{flushleft}
}
\end{table}

%	}
%\end{landscape}

\begin{sidewaystable}
	\centering
	{\setstretch{1}
		\begin{table}[H] \centering
\newcolumntype{C}{>{\centering\arraybackslash}X}

\caption{The Geography of Information}
\label{tab:error_reg_labs_gravity}
{\footnotesize
\begin{tabularx}{\linewidth}{l C C C C C C C m{0.005\textwidth} C C C m{0.005\textwidth} C C}

\toprule
&\multicolumn{14}{c}{$\ln(|Error_{ijt,t}^m|)$}\tabularnewline\cline{2-15} \tabularnewline  \cline{2-8} \cline{10-12} \cline{14-15}\tabularnewline  &{(1)}&{(2)}&{(3)}&{(4)}&{(5)}&{(6)}&{(7)}&&{(8)}&{(9)}&{(10)}&&{(11)}&{(12)} \tabularnewline
{Coefficient}&{}&{}&{}&{}&{}&{}&{}&{}&{Finance}&{Finance}&{Finance}&{}&{}&{} \tabularnewline
\midrule \addlinespace[0pt]
\midrule Foreign&.0577***&.054***&.0356&.0262&.0727**&.059**&.0189&&.0389*&.0348*&.026&&.024&.079*** \tabularnewline
&(.0158)&(.0148)&(.022)&(.0249)&(.0295)&(.029)&(.024)&&(.0208)&(.0206)&(.0186)&&(.0287)&(.0256) \tabularnewline
\textbf{\emph{W.r.t. closest subs.:}} &&&&&&&&&&&&&& \tabularnewline
Physical dist.&&.0053&&&&&&&&&&&& \tabularnewline
&&(.0078)&&&&&&&&&&&& \tabularnewline
Cultural dist.&&&.0115&&&&&&&&&&& \tabularnewline
&&&(.0085)&&&&&&&&&&& \tabularnewline
Linguistic dist.&&&&.0189*&&&.0193*&&&&&&.022*& \tabularnewline
&&&&(.0108)&&&(.0103)&&&&&&(.0125)& \tabularnewline
BC comovement&&&&&.007&&&&&&&&& \tabularnewline
&&&&&(.0106)&&&&&&&&& \tabularnewline
Migration&&&&&&.015&&&&&&&& \tabularnewline
&&&&&&(.0287)&&&&&&&& \tabularnewline
Trade&&&&&&&&&&&&&&.0133 \tabularnewline
&&&&&&&&&&&&&&(.0094) \tabularnewline
\textbf{\emph{W.r.t. headquarters:}} &&&&&&&&&&&&&& \tabularnewline
Linguistic dist.&&&&&&&&&&&&&.0144& \tabularnewline
&&&&&&&&&&&&&(.0101)& \tabularnewline
Trade&&&&&&&-.0233**&&&&&&&-.0214** \tabularnewline
&&&&&&&(.0105)&&&&&&&(.0105) \tabularnewline
Foreign $\times$ Low Cap. Controls&&&&&&&&&&-.0776***&-.0811**&&& \tabularnewline
&&&&&&&&&&(.0264)&(.0325)&&& \tabularnewline
Foreign $\times$ Institutions&&&&&&&&&&&-.0116&&& \tabularnewline
&&&&&&&&&&&(.0082)&&& \tabularnewline
N&389,295&388,415&349,093&373,980&378,911&290,630&373,066&&235,608&214,216&205,122&&348,250&382,051 \tabularnewline
$ R^2 $&.698&.6977&.7056&.6995&.7004&.7065&.6995&&.7196&.7177&.7178&&.6993&.6976 \tabularnewline
Cty $ \times $ Date $ \times $ Var. $ \times $ Hor. FE&\checkmark&\checkmark&\checkmark&\checkmark&\checkmark&\checkmark&\checkmark&&\checkmark&\checkmark&\checkmark&&\checkmark&\checkmark \tabularnewline
For. $ \times $ Date $ \times $ Var. $ \times $ Hor. FE &\checkmark&\checkmark&\checkmark&\checkmark&\checkmark&\checkmark&\checkmark&&\checkmark&\checkmark&\checkmark&&\checkmark&\checkmark \tabularnewline
\bottomrule \addlinespace[\belowrulesep]

\end{tabularx}
\begin{flushleft}
\footnotesize \begin{minipage}{1\linewidth} \vspace{-10pt} \begin{tabnote} \textit{Notes:}   The table shows the regression of the log absolute forecast error on regressors accounting for the geography of information. All standard errors are clustered at the country, forecaster and date levels. \end{tabnote} \end{minipage}  
\end{flushleft}
}
\end{table}

	}
\end{sidewaystable}

Column (1) of Table \ref{tab:error_reg_labs_gravity} reports the baseline estimates for Foreign. Column (2) to (4) add the distance variables. \textit{Physical} and \textit{Cultural Distance} have the expected signs but are statistically insignificant, whereas \textit{Linguistic Distance} is significant and fully absorbs the impact of Foreign. This finding suggests that language is a major barrier to information, and one of the major causes of the foreign penalty. \textit{Business cycle comovement} and \textit{Migration} do not seem to play a role (Columns (5) and (6)). Overall, we identify \textit{Linguistic Distance} as the main driver of the foreign penalty.


Note that, as widely shown by the above-mentioned literature, barriers to information are also correlated with trade and financial ties, which in turn may increase the incentives to acquire information. In line with rational inattention models \citep{Sims2003}, agents can choose how much effort and resources to allocate to acquiring and processing information. Consequently, the observed effect of geography on the foreign penalty could actually be the indirect result of these choices. In that case, information asymmetries are not an exogenous information processing constraint, but rather the result of an endogenous choice to devote less resources to information acquisition.



% In this view, information asymmetries arise from deliberate choices, not fixed limitations.


 %As in rational inattention models \citep{Sims2003,Mackowiak2009}, agents can choose to devote more resources to the acquisition and processing of information. Therefore, the impact of geography on the foreign penalty could actually be the indirect result of these choices. In that case, information asymmetries are not an exogenous information processing constraint, but rather the result of an endogenous choice to devote less resources to information acquisition.

To test this incentives' channel, we introduce trade linkages in the regression. \textit{Trade linkages} are measured by the exports from the country where the forecaster's headquarters is located to country $j$, normalized by the GDP of the headquarters' country. The idea is that, controlling for other drivers of the errors, a forecaster will devote more resources to forecasting a country if that country is an important market for the producers of the country where the forecaster's headquarters are located. In that case, the coefficient should be negative. Here, we assume that the headquarters concentrate the largest part of the firm's operations and investor base. Column (7) shows that the coefficient is negative, which suggests that the incentives' channel is at play.

Columns (8) to (10) focus on financial forecasters (banks, mutual funds, hedge funds, etc.), who are more likely to have investments at stake in foreign countries and therefore should have stronger incentives to acquire information about them. Column (8) shows that the foreign penalty is lower on average among financial forecasters (0.04 against 0.06 for the whole sample), although this difference is not statistically significant. To further test for the existence of an incentive channel among these financial forecasters, we follow \citet{Baeetal2008} and add the interaction between \textit{Foreign} and a measure of financial openness. As long as financial openness is unrelated to information asymmetries, a negative coefficient on the interaction can be interpreted as reflecting incentives: financial investors will devote more resources to information acquisition in more open foreign markets, resulting in a lower foreign penalty. In Column (9), Foreign is interacted with a dummy that is equal to one if country $j$ has a low level of capital controls: we use the measure of \emph{de jure} capital controls from \citet{Fernandez2016} and define \textit{Low-capital control} countries as the quartile with the lowest level of capital controls. This coefficient is significantly negative, suggesting that capital market openness does increase the incentives to acquire information for financial forecasters. Since financial openness is likely to be correlated with the quality of institutions, which could be related to transparency, we check in Column (10) that our results are robust to the inclusion of the interaction between \textit{Foreign} and a measure of the \textit{Quality of institution}s. We conclude that incentives influence the performance of forecasters and determine the foreign penalty.

This challenges the interpretation of the results in Columns (2) to (6) as being driven directly by barriers to information. An alternative and equally plausible explanation is that \textit{Linguistic Distance} leads to weaker economic linkages between countries, which in turn result in poorer forecast performance. %This casts doubts on the interpretation of the results of Column (2) to (6) as arising directly from barriers to information. Indeed, another valid interpretation would be that linguistic distance indirectly generates larger forecasting errors, because it reduces economic linkages between countries. %A first argument against this interpretation is that physical distance does not appear to be a significant driver of the forecasters' errors. If this indirect channel were at play, physical distance would have been significant. Indeed, because physical distance is a good proxy for transportation costs, it is a robust driver of bilateral trade, and hence of financial linkages. Instead, linguistic distance, which is more directly related to information, plays an important role. This is evidence that information asymmetries are not the mere result of incentives.
An argument against this interpretation is that the geography of incentives differs from the geography of barriers to information. Indeed, while we have assumed so far that trade linkages are relevant at the headquarters level, we can test whether this assumption is valid by performing a horse race between two measures of trade linkages: a measure based on trade with the headquarters' country and a measure based on trade with the closest subsidiary's country. Column (12) confirms that bilateral \textit{Trade} is significant only if it is computed as bilateral trade between the country and the forecaster's headquarters. Now, if barriers to information were only relevant because they indirectly generated incentives to acquire information, then we would expect that only information barriers between the country and the headquarters' country are relevant. However, we find that information barriers are relevant at the level of the closest subsidiary, not at the headquarters' level: \textit{Linguistic Distance} is significant only if computed between the country and the forecaster's closest subsidiary, as is apparent in Column (11). We thus interpret the evidence as being consistent with the coexistence of both the information barriers channel and the incentives channel.


%we follow their approach and interact the Foreign dummy with some measures of \emph{de jure} economic protectionism. If this channel is at play, then investors and exporters will devote less resources to acquire information on a protectionist foreign market, and the foreign penalty will be larger for protectionist countries. As long as protectionism is unrelated to information asymmetries, we can interpret a positive interaction term as arising from incentives. In Column (5), we include the interaction between Foreign and the average tariff applied to the most favored nation from the World Trade Organization database (https://stats.wto.org/). This coefficient is positive, but insignificant. Column (6) shows that the Foreign penalty is lower on average among financial forecasters, which are more likely to have investments at stake in foreign countries (0.04 against 0.06 for the whole sample), but this difference is not statistically significant. In Column (7), we add the interaction between Foreign and a measure of \emph{de jure} capital controls from \citet{Fernandez2016}. This coefficient is significantly positive, suggesting that capital controls lower the incentives to acquire information for financial forecasters. However, note that economic protectionism can be correlated with the degree of transparency, which can affect information asymmetries. To control for this, we add the interaction between Foreign and a measure of institutional quality in Column (8). The coefficient now becomes insignificant. The foreign penalty is represented in Figure \ref{fig:interaction} over the range of tariffs and capital controls that is relevant in our sample. We do see a positive slope, but this slope is not steep enough to statistically tell apart the value of the foreign penalty across the distribution. We thus do find \emph{some} evidence that incentives lower the foreign penalty, but this evidence is weak.%\footnote{Note that the fact that the foreign dummy does not mean that the Foreign penalty vanishes, because the average impact of Foreign depends on the combnation of Foreign and the interaction term. For instance, the sum of Foreign and the interaction term, evaluated at the median value of tariffs, which is 5\%, is significant.}

